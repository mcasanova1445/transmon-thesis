\documentclass[xetex,mathserif,serif]{beamer}

\begin{document}
  \begin{frame}
    \frametitle{This is the first slide}
    %Content goes here
  \end{frame}
  \begin{frame}
    \frametitle{This is the second slide}
    \framesubtitle{A bit more information about this}
    %More content goes here
  \end{frame}
% etc
\end{document}

%---
%title:
%- Diseño y simulación de procesadores cuánticos que implementen algoritmos cuánticos de busqueda
%author:
%- Miguel Casanova
%header-includes:
%   - \usepackage{physics}
%   - \usepackage{mathtools}
%   - \usepackage{physics}
%   - \usepackage{qcircuit}
%   - \usepackage{graphicx}
%   - \usepackage{float}
%   - \usepackage{dsfont}
%   - \usepackage{tkz-graph}
%   - \usetikzlibrary{arrows}
%   - \usepackage{ragged2e}
%theme:
%- Copenhagen
%---
%
%
%# Estructura de la presentación
%
%# Objetivos
%
%# Información cuántica
%
%# Espacios de Hilbert
%
%Un espacio de Hilbert es un espacio lineal real o complejo con un producto interno que también define un espacio normado completo.
%
%- Producto interno
%- Normado
%- Completo: Todas las sucesiones de Cauchy convergen fuertemente.
%    * Sucesiones de Cauchy: Aquellas en las que el ordenamiento no afecta la convergencia.
%- Separable: Tiene bases contables.
%
%# Operadores
%
%- Operadores hermíticos $U = U^\dagger$
%    * Autovalores reales
%    * Diagonal real
%    * Diagonalizable
%
%- Operadores unitarios: $U U^\dagger = \mathds{1}$
%    * Determinante de módulo igual a la unidad
%    * Preserva normas y trazas
%    * Diagonalizable
%
%# Estados cuánticos
%
%- Estados puros: $\ket{\psi}$
%- Estados mixtos: $\rho$
%
%# Sistemas multipartitos
%
%$\mathcal{H} = \mathcal{H}_1 \mathcal{H}_2$
%
%$\rho_B = Tr_A(\rho_{AB})$
%
%# Qubits y esfera de Bloch
%
%Los qubits son la unidad básica de información cuántica.
%
%$\ket{0} = \begin{pmatrix}1 \\ 0\end{pmatrix}, \quad \ket{1} = \begin{pmatrix}0 \\ 1\end{pmatrix}$
%
%$\ket{\psi} = \alpha \ket{0} + \beta \ket{1}, \quad \alpha, \beta \in \mathds{C}$
%
%Esfera de Bloch:
%
%# Ecuación de Schrödinger
%
%Esta ecuación describe la evolución de los estados cuánticos puros.
%
%$i \hbar \frac{d}{dt} \ket{\psi(t)} = \hat{H} \ket{\psi(t)}$
%
%Para estados mixtos se utiliza la ecuación de Liouville-von Neumann.
%
%$i \hbar \dot{\rho}(t) = [\hat{H}, \rho(t)]$
%
%# Postulados de la mecánica cuántica
%
%# Matrices de Pauli
%
%$\sigma_x = \begin{pmatrix}0 & 1 \\ 1 & 0\end{pmatrix}$
%
%$\sigma_y = \begin{pmatrix}0 & -i \\ i & 0\end{pmatrix}$
%
%$\sigma_z = \begin{pmatrix}1 & 0 \\ 0 & -1\end{pmatrix}$
%
%# Circuitos cuánticos
%
%Circuito de ejemplo
%
%# Compuertas cuánticas
%
%Compuertas con matrices
%
%# Criterios de Di Vincenzo
%
%- Un sistema físico escalable con qubits caracterizados.
%- La habilidad de inicializar el estado de los qubits en un estado fiducial simples.
%- Tiempos de coherencia relevantes largos.
%- Un conjunto universal de compuertas cuánticas.
%- La capacidad de medir qubits en específico.
%
%# Fidelidad
%
%La fidelidad es una medida de distancia entre estados cuánticos. Dos estados idénticos tendrán una fidelidad igual a 1, mientras que dos estados ortogonales tendrán una fidelidad de 0.
%
%$F(\rho, \sigma) = Tr(\sqrt{\rho^{1/2} \sigma \rho^{1/2}})$
%
%# Medidas proyectivas
%
%# Sistemas cuánticos abiertos
%
%En la mecánica cuántica de sistemas abiertos con evolución markoviana, la ecuación de Schrödinger toma la siguiente forma más general, conocida como Lindbladiano.
%
%$\dot{\rho}(t) = -i [\hat{H}, \rho(t)] + \sum_k \gamma_k [V_k \rho(t) V_k^\dagger - \frac{1}{2} \{V_k^\dagger V_k, \rho(t)\}]$
%
%# Superconductividad
%
%# Teorías BCS
%
%# Efecto Josephson
%
%\justify
%Se tienen dos placas superconductoras A y B, separadas por un aislante.
%Las funciones de onda de las placas superconductoras son:
%$\psi_A = \sqrt{\rho_1} e^{i \phi_1}, \psi_B = \sqrt{\rho_2} e^{i \phi_2}$ 
%
%\justify
%En principio, no debería haber conducción entre ambas placas.
%Sin embargo, ese no es el caso. Por el efecto tunel, una supercorriente (corriente
%sin disipación) de pares de Cooper (pares de electrones con spines opuestos)
%pueden pasar de una placa a la otra sin disipación.
%
%$$V_J = \frac{\hbar}{2e} \frac{d\delta}{dt}$$
%
%$$I_J = I_0 \sin(\delta)$$
%
%\justify
%Donde $\delta=\phi_2-\phi_1$ es la diferencia de fase entre las dos placas superconductoras.
%
%# Efecto Josephson DC y AC
%
%* Efecto Josephson DC
%
%\justify
%Si las placas se encuentran sin alimentación, entonces correrá una
%supercorriente constante a través de ellas.
%
%* Efecto Josephson AC
%
%\justify
%Si las placas se alimentan con un voltaje DC externo, entonces la diferencia
%de fase entre ellas variará linealmente con el tiempo y habrá una corriente AC
%a través de ellas.
%
%# Energía e inductancia de Josephson
%
%$$E_J = \int I_0 \sin(\delta) \frac{\hbar}{2e} \frac{d\delta}{dt} dt
% = \frac{\hbar I_0}{2e} \int \sin(\delta) dt
% = - \frac{\hbar I_0}{2e} \cos(\delta)$$
%
%$\frac{dI_J}{dt} = I_0 \cos(\delta) \frac{d \delta}{dt} = I_0 \cos(\delta)
% \frac{2e}{\hbar} V_J$
%
%$$L_J = \frac{\hbar}{2e I_0 \cos(\delta)}$$
%
%$E_C = \frac{(2e)^2}{2C}$
%
%$E_L = \frac{\hbar^2}{(2e)^2L}$
%
%# Circuito LCJ
%
%\justify
%Primero, reescribimos el Hamiltoniano del circuito LC en términos de la
%cantidad de pares de Cooper y de la diferencia de fase en los extremos
%del inductor, en lugar de la carga y el flujo
%
%$$\hat{H} = \frac{(2e)^2}{2C} \hat{n}^2 + \frac{\hbar^2}{(2e)^2L} 
% \frac{\hat{\delta^2}}{2}$$
%
%De aquí $\hat{q}=2e\hat{n}$ y $\hat{\phi}=\frac{\hbar}{2e}\hat{\delta}$
%
%Ahora introducimos el término de la unión Josephson
%
%$\hat{H} = \frac{(2e)^2}{2C} (\hat{n}-n_g)^2 + \frac{\hbar^2}{(2e)^2L} 
% \frac{\hat{(\delta-\delta_e)^2}}{2} - \frac{\hbar I_0}{2e} \cos(\delta)
% = E_C (\hat{n}-n_g)^2 + E_L \frac{(\hat{\delta}-\delta_e)^2}{2}
% - E_{J0} \cos(\hat{\delta})$
%
%# Ecuación de Schrödinger del circuito LCJ
%
%\justify
%Para describir el sistema en términos de la ecuación de Schrödinger en
%función de la diferencia de fase $\delta$, se introduce $\hat{n}=-i\hbar
%\frac{\partial}{\partial \phi}$
%
%$$E_C (-i\hbar \frac{\partial}{\partial\phi}-n_g)^2+U(\phi))\psi = E \psi$$
%$$U(\phi) = -E_{J0} \cos(\phi)+E_L \frac{(\phi-\phi_e)^2}{2}$$
%
%# Arquetipos de qubits superconductores
%
%\justify
%* Qubit de carga: Si $E_L$ tiende a cero, la carga almacenada en la isla 
%    superconductora entre el capacitor y  la unión Josephson se puede usar 
%    como qubit. El potencial de este tipo de qubit es de forma de coseno.
%
%\justify
%* Qubit de flujo: Si $E_L$ es comparable con $E_{J0}$, el flujo a través 
%    del lazo formado por el inductor y la unión Josephson se puede usar como 
%    qubit. El potencial de este tipo de qubit es de forma cuártica.
%
%\justify
%* Qubit de fase: Si se polariza la unión Josephson con una fuente de 
%    corriente, la fase en ambos extermos de la unión Josephson se puede 
%    usar como qubit. El potencial de este tipo de qubit es de forma cúbica.
%
%# Qubits de carga
%
%$[\hat{\delta},\hat{n}]=i \implies e^{\pm i \hat{\delta}} \ket{n} = \ket{n\pm 1}$
%
%$\hat{H} = E_C (\hat{n}-n_g)^2 - E_{J0} \cos(\hat{\delta}) =
% E_C (\hat{n}-n_g)^2 - E_{J0} (e^{i \hat{\delta}} + e^{-i \hat{\delta}}) =
% \sum(E_C(N-N_g)^2 \ketbra{n}{n} - \frac{E_{J0}}{2}(\ketbra{n}{n+1}+
% \ketbra{n+1}{n}))$
%
%# Caja de pares de Cooper 
%
%Dimensiones típicas de la isla: 1000nm x 50nm x 20nm
%
%![Circuito de una caja de pares de Cooper](Avance1/cooperpairbox.png){#id .class width=45%} ![Niveles de energía](Avance1/cooperenergy.png){#id .class width=45%}
%
%# Transmon
%
%Intercambiamos anarmonicidad por independencia de $n_g$
%
%![Circuito de un transmon](Avance1/transmon.png){#id .class width=45%} ![Niveles de energía](Avance1/transmonenergy.png){#id .class width=45%}
%
%# Modelo de Jaynes-Cummings
%
%$$\hat{H} = \hat{H}_c + \hat{H}_q + \hat{H}_{qc} = \hbar \omega_c (a^\dag a + 
%  \frac{1}{2}) + \frac{1}{2} \hbar \omega_q \sigma_z + \hbar g \sigma_x (a+a^\dag)$$
%
%\justify
%De ahora en adelante $\hbar = 1$ y despreciaré los términos constantes, pues 
%sólo contribuyen en fases globales a la evolución del sistema.
%
%# Aproximación de onda rotacional
%
%$$\hat{H}_{qc} = \hat{H}_{qc}^{JC} + \hat{H}_{qc}^{AJC} = g(a \sigma_+ + 
%  a^\dag\sigma_-) + g(a^\dag \sigma_+ + a \sigma_-)$$
%
%$$\hat{H} = \hat{H}_c + \hat{H}_q + \hat{H}_{qc} = \omega_c a^\dag a +
%  \frac{1}{2} \omega_q \sigma_z + g(a \sigma_+ + a^\dag \sigma_-)$$
%
%# Hamiltoniano multiquibit
%
%$$\hat{H} = \hat{H}_q + \hat{H}_{qc} = \frac{1}{2} \sum\limits_i \omega_{qi} 
%  \sigma_{zi} + \sum\limits_i g_i (a \sigma_{+ i} + a^\dagger \sigma_{- i})$$
%
%# Pulsos de microondas
%
%$$\hat{H}_d = \sum\limits_k (a+a^\dagger) (\xi_k e^{-i\omega_d^{(k)}t} + 
%  \xi_k^*e^{i\omega_d^{(k)}t})$$
%
%RWA: $$\hat{H}_d=\sum\limits_k a\xi_k^*e^{i\omega_d^{(k)}t}+
%  a^\dagger\xi_ke^{-i\omega_d^{(k)}t}$$
%
%# Régimen rotacional del pulso
%
%\justify
%Trabajando con un sólo modo a la vez, se aplica la siguiente transformación $U(t) 
%= exp[-i \omega_d t(a^\dagger a + \sum\limits_i \sigma_{z i})]$ para entrar en 
%el régimen rotacional del pulso de control.
%
%$$\hat{H} = U^\dagger (\hat{H}_{syst} + \hat{H}_d) U - i U^\dagger \dot{U}$$
%$$ \hat{H} = \Delta_c a^\dagger a + \frac{1}{2} \sum\limits_i \Delta_{qi} 
%  \sigma_{zi} + \sum\limits_i g_i (a \sigma_{+ i} + a^\dagger \sigma_{- i}) + 
%  (a\xi^*e^{i\omega_d t}+a^\dagger\xi e^{-i\omega_d t})$$
%
%$\Delta_c = \omega_c - \omega_d \qquad \quad \Delta_{qi} = \omega_{qi} - \omega_d$
%
%# Efecto del pulso sobre el qubit
%
%\justify
%Luego se aplica el operador de desplazamineto $D(\alpha) = exp[\alpha a^\dagger - 
%\alpha^* a]$ sobre el campo $a$ con $\dot{\alpha} = -i \Delta_c \alpha -i \xi 
%e^{-i \omega_d t}$ para eliminar el efecto directo del pulso sobre la cavidad.
%
%$$\hat{H} = D^\dagger (\alpha) \hat{H}_{old} D(\alpha) -i D^\dagger(\alpha) \dot{D}(\alpha)$$
%
%$$\hat{H} = \Delta_c a^\dagger a + \frac{1}{2} \sum\limits_i \Delta_{qi} 
%  \sigma_{zi} + \sum\limits_i g_i (a \sigma_{+i} + a^\dagger \sigma_{-i})$$
%$$ + \sum\limits_i g_i (\alpha \sigma_{+i} + \alpha^* \sigma_{-i}) - \Delta_c
%  \alpha \alpha^* $$
%
%El término $-\Delta_c \alpha \alpha^*$ se desprecia, ya que sólo representa 
%una fase global en la evolución del sistema.
%
%# Régimen dispersivo
%
%\justify
%Finalmente, aplicamos la transformación $U = exp[\sum\limits_i \frac{g_i}
%{\Delta_i} (a^\dagger \sigma_{-i} - a \sigma_{+i})]$, donde $\Delta_i = 
%\omega_{qi} - \omega_c$ y realizamos la aproximación de segundo grado sobre
%los términos $\frac{g_i}{\Delta_i} \ll 1$.
%
%$$\hat{H} = U^\dagger \hat{H}_{old} U$$
%$$\hat{H} \approx \tilde{\Delta}_c a^\dagger a + \frac{1}{2} \sum\limits_i
%  \tilde{\Delta}_{qi} \sigma_{zi} + \sum\limits_i (\Omega_i \sigma_{+i} +
%  \Omega_i^* \sigma_{-i})$$
%$$+ \sum\limits_{i \neq j} \frac{g_i g_j}{2 \Delta_i} 
%  (\sigma_{-i} \sigma_{+j}+\sigma_{+i} \sigma_{-j})$$
%
%$\tilde{\Delta}_c = (\omega_c + \sum\limits_i \chi_i \sigma_{zi}) - \omega_d
% \qquad
% \tilde{\Delta}_{qi} = (\omega_{qi} + \chi_i) - \omega_d
% \qquad
% \chi_i = \frac{g_i^2}{\Delta_i}$
%
%# Rotaciones X-Y
%
%\justify
%Tomando $\Omega(t) = \Omega^x(t) \cos(\omega_d t) + \Omega^y \sin(\omega_d t)$,
%donde $\omega_d$ es igual a la frecuencia de resonancia de uno de los qubits
%logramos rotaciones sobre los ejes X e Y. Las amplitudes de estas rotaciones
%vienen dadas por $\int_0^{t_0} \Omega^x(t) dt$ y $\int_0^{t_0} \Omega^y(t)
%dt$, respectivamente, donde $t_0$ es la duración del pulso.
%
%$$\hat{H} \approx \tilde{\Delta}_c a^\dagger a + \frac{1}{2} \tilde{\Delta}_q 
%  \sigma_z + \frac{1}{2} (\Omega^x(t) \sigma_x + \Omega^y(t) \sigma_y)$$
%
%# Compuerta de entrelazamiento
%
%Ejemplo con sólo dos qubits
%
%$$\hat{H} \approx \frac{1}{2} \tilde{\Delta}_{q_1} \sigma_{z_1} + 
%  \frac{1}{2} \tilde{\Delta}_{q_2} \sigma_{z_2} + 
%  \frac{g_1 g_2 (\Delta_1 + \Delta_2)}{2 \Delta_1 \Delta_2} 
%  (\sigma_{-_1} \sigma_{+_2} + \sigma_{+_1} \sigma_{-_2})$$
%
%Variando la frecuencia de resonacia de los qubit, se puede variar el 
%acoplamiento entre estos. 
%
%# Simulador
%
%# Parámetros de los sistemas simulados
%
%# Compuertas compuestas
%
%# Algoritmo de Grover
%
%\begin{align}
%    \Pi_\mathcal{W} &= \sum_k \ketbra{\omega_k} \\
%    \ket{\psi} &= \sin(\theta) \ket{\psi_1} + \cos(\theta) \ket{\psi_0}
%\end{align}
%
%\begin{align}
%    U_\mathcal{W} &= 2 \mathds{1} - 2 \Pi_\mathcal{W}\\
%    U_\psi &= 2 \ketbra{\psi} - \mathds{1}
%\end{align}
%
%
%# Variaciones del algoritmo de Grover
%
%# Simulaciones del algoritmo de Grover
%
%# Transformada cuántica de Fourier
%
%\begin{equation}
%    QFT^\dagger = \frac{1}{\sqrt{N}} \sum_x \sum_k e^{-2 \pi i k x / N} \ketbra{x}{k}
%\end{equation}
%
%# Estimación de fase
%
%# Estimación de orden
%
%# Algoritmo de Shor
%
%# Simulaciones del algoritmo de Shor
%
%# PageRank
%
%# Caminatas cuánticas de Szegedy
%
%# PageRank cuántico
%
%# Circuitos de Loke
%
%# Simulaciones del algoritmo de PageRank cuántico
%
%# Conclusiones
%
%
%
%
%
%
%
%
%
%
%
%
%
%
%
%
%
%
%
%
%
%
%
%
%
%
%
%
%
%
%
%
%
%
%
%
%
%
%
%
%
%
%
%
%
%
%
%
%
%
%
%
%
%
%
%
%
%
%
%
%
%
%
%
%
%
%
%
%
%
