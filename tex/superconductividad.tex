\chapter{Superconductividad}

Gross y Marx, Walther-Meißner-Institut \cite{gross}

\section{Modelos cuánticos macroscópicos}

Uno de los mayores principios de la mecánica cuántica es el hecho de que cantidades fisicas como la energía o el momentum están, bajo ciertas condiciones, cuantizados. Es decir, que sólo tienen valores discretos. Sin embargo, por un largo tiempo se creyó que la cuantización sólo era relevante para sistemas microscópicos, como los nucleos, los átomos o las moléculas. De hecho, considerndo el comportamiento de objetos macroscópicos que consistan de una gran cantidad de átomos, los efectos de la cuantización no pueden ser observados, aunque cada átomo individual obedezca las leyes de la mecánica cuánica. Esto se debe al hecho de que los movimientos térmicos enmascaran las regularidades cuánticas. Sin embargo, para ciertos fénomenos, en particular la superconductividad, ha sido demostrado que es posible observar cuantización macroscópica. Así que podemos observar la cuantización de parámetros que caracterizan a sistemas macroscópicos (por ejemplo, el flujo a través de un anillo superconductor de dimensión macroscópica) muchos órdenes de magnitud más grandes que sistemas como los átomos. Esto se debe a la alta correlación entre los electrones en un superconductor, por efectos de coherencia. Por esto, se deberán considerar todos los electrones superconductores como una única entidad mecánico-cuántica.

\section{Modelo macroscópico de la superconductividad}

El modelo macroscópico de la superconductividad está basado en la hipótesis de que existe una función de onda $\psi(r,t)$ que describe el comportamiento del ensemble completo de electrones superconductores. Por supuesto, esta hipótesis puede ser justificada por la teoría microscópica de la superconductividad (Teoría BCS). Esta teoría se basa en la idea de que en metales superconductores existe una fuerza atractiva entre los electrones cercanos al nivel de Fermi. A temperaturas bajo la temperatura críctica $T_c$, esta fuerza atractiva crea un nuevo estado cuántico diferente del mar de Fermi de un metal normal. Se puede decir que una pequeña porción de los electrones cercanos el nivel de Fermi están ligados a pares de Cooper (pares de electrones con spines opuestos que se comportan como una única entidad). En el caso más simple, el movimiento interno de los pares no tiene momentum angular orbital (estado s simétrico) y consecuentemente el principio de Pauli requiere que los dos spines estén en un estado de spin singleto (antisimétrico). Contrario a ligar dos átomos a una molécula, el estado orbital del par tiene un radio mucho mayor, típicamente entre 10nm y 1um, de manera que pares individuales se sobrelapan fuertemente en espacio y por lo tanto, la ligadura resulta cooperativa. En particular, la energía de ligadura de cualquier par depende de cuantos otros pares se hayan condensado y, más aún, el movimiento del centro de masas de los pares está fuertemente correlacionado tal que cada par reside en el mismo estado con el mismo movimiento de centro de masas. Es este estado el que describimos con una función de onda macroscópica y que le da al sistema sus propiedades superfluídicas. Por ejemplo, el movimiento del centro de masas puede ser descrito por la función de onda $\psi(r,t) = \psi_0 e^{i \theta(r,t)} = \psi_0 e^{i k_s \dot r - i \omega t}$, donde cada par tiene el mismo momentum $\hbar k_s$ o velocidad de par $v_s = \hbar k/m^$.

En la aproximación usual de partículas que interactuan debilmente o que no interactuan, su evolución se puede describir en términos de la ecuación de Schrödinger ordinaria $i \hbar \frac{\partial \psi}{\partial t} = \hat{H} \psi$, donde $\psi(r,t) = \psi_0(r,t) e^{t \theta(r,t)}$ es la función de onda compleja de una partícula. $\abs{\psi}^2$ puede ser interpretado como la densidad de probabilidad de las partículas. En el caso estacionario se puede asumir que $\abs{\psi}^2$ es constante y $\hat{H}$ puede ser reemplazado por la energía $E$ de la partícula. Entonces podemos escribir:

\begin{equation}
    \hbar \frac{\partial}{\partial t} = - E
\end{equation}

Es decir, el carácter cuántico específico se reduce a aquel de la fase $\theta$ de la función de onda.

El punto importante es que en metales normales () no resulta en correlaciones cuánticas para las variables macroscópicas porque los electrones obedecen la estadística de Fermi-Dirac y sus energías nunca pueden ser exactamente iguales. Por lo tante, de acuerdo a (), la evolución temporal de las fases de la función de onda de las partículas difiera para todas ellas. Es decir, que las fases están uniformemente distribuidas, y ya que las cantidades macroscópicas son la suma sobre todas las partículas, las fases se cancelan en estas cantidades.

Este no es el caso en los superconductores. En estos los pares ligados de electrones (pares de Cooper) se forman con momenta y espines opuestos en el caso más simple. Estos pares, con espín neto cero obedecen la estadística de Bose-Einstein y por lo tanto, pueden ocupar el estado de menor energía a temperaturas bajas. Como resultado sus $\partial \theta / \partial t$ son identicos. Además, los pares de Cooper tienen tamaños relativamente grandes, del orden de 10 a 1000nm, los cual es mucho más grande que la distancia típica entre los pares. Por lo que las funciones de onda de cada par individual se sobrelapan en gran medida. Como resultado de estos dos hechos, todos los pares forman un estado de "fase bloqueda" que puede ser descrito por una única función de onda $\psi$, la cual es denotada frecuentemente como el parámetro de orden. En esta situación las fases no se cancelan en la suma sobre todas las partículas y las variables macroscópicas, en particular la corriente, pueden depender de la fase $\theta$, la cual cambia de una manera cuántica bajo la acción de un campo electromagnético. Esta dependencia cuántica lleva no sólo a la resistencia cero de los superconductores y al efecto Meißner-Ochsenfeld, sino también a efectos específicamente coherentes como la cuantización de flujo y el efecto Josephson.

En el caso de la funcion de onda macroscópica, se cumplen las siguientes propiedades:

\begin{align}
    \int \psi^*(r,t) \psi(r,t) dV &= N^*_s \\
    \abs{\psi(r,t)}^2 = \psi^*(r,t) \psi(r,t) &= n^*_s(r,t)
\end{align}

Donde $n^*_s(r,t)$ es la densidad local y $N^*_s$ es el número total de electrones superconductores.

\section{Junciones de Josephson}

Las junciones de Josephson representan el elemento clave en los qubits superconductores. Se podría decir que son para los qubits, lo que los transistores son para los bits.

\section{Efecto Josephson}

EL efecto Josephson, predicho por Brian David Josephson en 1962, consiste en que una corriente fluya indefinidamente a través de una junción de Josephson aun cuando no hay una diferencia de potencial aplicada. Una junción de Josephson consta de dos superconductores acoplados por una conexión debil, la cual puede ser formada por un aislante (superconductor-isolator-superconductor, SIS), un metal normal (superconductor-normal-superconductor, SNS) o cualquier otro material u obstáculo que acople debilmente a los dos superconductores. En principio, no debería haber conducción entre ambas placas.  Sin embargo, ese no es el caso. Por el efecto tunel, una supercorriente (corriente sin disipación) de pares de Cooper (pares de electrones con spines opuestos) pueden pasar de una placa a la otra sin disipación.

Las uniones Josephson son capaces de generar voltajes oscilatorios de alta frecuencia, por lo regular de $10^{10} ~ 10^{11} Hz$ y detectan potenciales eléctricos de un cuatrillón de voltios.

Para comenzar el analisis de una junción de Josephson, se considera un sistema simple y simétrico, tal que el material sea el mismo en ambos extremos de la junción y no exista campo magnético. Se tienen dos placas superconductoras A y B, separadas por un aislante, cuyas funciones de onda son: $\psi_A = \sqrt{\rho_1} e^{i \phi_1}, \psi_B = \sqrt{\rho_2} e^{i \phi_2}$ 


\begin{align}
    V_J &= \frac{\hbar}{2e} \frac{d\delta}{dt} \\
    I_J &= I_0 \sin(\delta)
\end{align}

Donde $\delta=\phi_2-\phi_1$ es la diferencia de fase entre las dos placas superconductoras.

\subsection{Efecto Josephson DC}

Si las placas se encuentran sin alimentación, entonces correrá una supercorriente constante a través de ellas.

\subsection{Efecto Josephson AC}

Si las placas se alimentan con un voltaje DC externo, entonces la diferencia de fase entre ellas variará linealmente con el tiempo y habrá una corriente AC a través de ellas.

\section{Componentes de la corriente en las junciones de Josephson}

Esta corriente tiene tres componentes:

\begin{enumerate}
    \item $I_d$, la corriente de desplazamiento: Como la corriente en un capacitor. La junción de Josephson forma un capacitor de placas parelelas superconductoras con un material aislante o un metal normal entre ellas, entonces podemos hablar de una corriente de desplazamiento $I_d$. La capacitancia $C$ de este dispositivo está definida de la misma manera que en el estado normal: $C = \epsilon_r \frac{A}{4 \pi d}$, donde $\epsilon_r$ es la constante dieléctrica relativa de la capa que separa a los dos superconductores, $d$ la separación de los superconductores y $A$ el área de los mismos.
    \item $I_n$, la correinte ordinaria: Por los electrones individuales. Cuando la temperatura $T \neq 0$, siempre habrá movimiento térmico de cargas cuya energía es del orden de $k_b T$, donde $k_B$ es la constante de Boltzmann. Cuando $T$ es menor, pero cercano a la termperatura crítica $T_c$, la energía de acoplamiento de los pares de Cooper $E_g = 2 \Delta$ es mucho menor a $k_B T$, lo cual resulta en la disminución e los pares de Cooper y el aumento de la concentración de eletrones normales. Si el voltaje a través de la junción es mayor al asociado a la energía de la brecha $V_g = \abs{\Delta_1 + \Delta_2}/e$, los pares de Cooper de un la do de la unión se rompen y uno de los electrones de cada uno de los pres disueltos pasa al otro lado, es decir, se produce un tunelamiento de electrones normales. Si la concentración de electrones individuales aumenta, el comportamiento de la unión tenderá a uno de tipo óhmico, es decir, la junción tenderá a comportarse como una resistencia.
    \item $I_s$, la supercorriente: Por los pares de Cooper. Se puede 
\end{enumerate}

\section{Qubits superconductores}
Los qubits superconductores se basan en circuitos osciladores no lineales, hechos a partir de JJs \cite{wendin}.
\vspace{0.5cm}

El Hamiltoniano de un oscilador armónico LC está dado por 

\begin{equation}
\hat{H} = E_C \hat{n}^2 + E_L \frac{\hat{\phi}^2}{2},
\end{equation}

donde $\hat{n}$ es la cantidad de pares de Cooper inducidos en el capacitor (En otras parabras, la carga inducida en el capacitor, medida en unidades de $2e$), y $\hat{\phi}$ es la diferencia de fase sobre el inductor. La carga $\hat{n}$ y la fase $\hat{\phi}$ no conmutan, $\comm{\hat{\phi}}{\hat{n}}=i$, lo que significa que sus valores esperados no se pueden medir simultaneamente. $E_C=\frac{(2e)^2}{2C}$, $E_L=\frac{\hbar^2}{(2e)^2L}$ y la distancia entre niveles de energía del oscilador armónico $\hbar \omega = \frac{\hbar}{\sqrt{LC}}=\sqrt{2E_LE_C}$.
\vspace{0.5cm}

Para poder servir como qubit, el oscilador debe ser anarmónico, de manera que se pueda operar sobre un par específico de niveles de energía. Al agregar una JJ, el Hamiltoniano del circuito LCJ se convierte en:

\[
\hat{H} = E_C (\hat{n}-n_g)^2 - E_{J0} \cos( \hat{\phi} ) + E_L \frac{(\hat{\phi}-\phi_e)^2}{2},
\]

donde $n_g$ es la carga inducida por voltaje en el capacitor C (isla qubit) y $\phi_e$ es la fase inducida por flujo sobre la JJ. La energía de Josephson $E_{J0}$ está dada por $E_{J0}=\frac{\hbar}{2e}I_0$ en términos de la corriente crítica $I_0$ de la unión. Usualmente, la JJ es del tipo Superconductor-Aislante-Superconductor con corriente crítica fija.

Con el fin de introducir la inductancia no lineal de Josephson, empezamos por 

\[
I_J = I_0 \sin(\phi)
\]

Combinado con la ley de Lenz:

\[
V = \frac{d\Phi}{dt} = \frac{\Phi_0}{2\pi} \frac{d\phi}{dt}, \hspace{20pt} \Phi_0=\frac{h}{2e}
\]

Se encuentra que:

\[
V = \frac{\Phi_0}{2\pi} \frac{1}{I_0\cos(\phi)} \frac{dI_J}{dt}
\]

Definiendo $L_J = V (\frac{dI_J}{dt})^{-1}$, se obtiene finalmente la inductancia de Josephson $L_{J0}$:

\[
L_J = \frac{\Phi_0}{2\pi} \frac{1}{I_0 \cos(\phi)} = L_{J0} \frac{1}{\cos(\phi)}
\]

Esto define la inductancia de Josephson de la JJ aislada y nos permite expresar la energía de Josephson como $E_{J0} = \frac{\hbar^2}{(2e)^2L_{J0}}$

\begin{align*}
[E_C (-i\hbar \frac{\partial}{\partial\phi}-n_g)^2 + U(\phi)] \psi = E \psi \\
U(\phi) = -E_{J0} \cos(\phi) + E_L \frac{(\phi-\phi_e)^2}{2}
\end{align*}

\begin{enumerate}
\item $E_L = 0 \quad (L \sim \infty)$ :
\item $E_L \approx E_{J0}$ :
\end{enumerate}

\section{Arquetipos de qubits superconductores}

\subsection{Qubit de carga}

Si $E_L$ tiende a cero, la carga almacenada en la isla superconductora entre el capacitor y  la unión Josephson se puede usar como qubit. El potencial de este tipo de qubit es de forma de coseno.

\subsection{Qubit de flujo}

Si $E_L$ es comparable con $E_{J0}$, el flujo a través del lazo formado por el inductor y la unión Josephson se puede usar como qubit. El potencial de este tipo de qubit es de forma cuártica.

\subsection{Qubit de fase}

Si se polariza la unión Josephson con una fuente de corriente, la fase en ambos extermos de la unión Josephson se puede usar como qubit. El potencial de este tipo de qubit es de forma cúbica.

\section{Transmones}

Los transmones son un tipo de qubit de carga. Tratando el transmón como un sistema de dos niveles acoplado linealmente a un oscilador monomodo, su Hamiltoniano toma la siguiente forma:

\[
\hat{H} = \hat{H}_q + \hat{H}_{qr} + \hat{H}_r = -\frac{1}{2} \epsilon \sigma_z + g \sigma_x (a+a^\dag) + \hbar \omega (a^\dag a + \frac{1}{2})
\]

donde $\epsilon$ es la energía de excitación del qubit, $g$ es el acoplamiento qubit-oscilador y $\omega$ es la frecuencia del oscilador.
\vspace{0.5cm}

Introduciendo los operadores escalera del qubit, $\sigma_\pm = \frac{1}{2}(\sigma_x \pm i \sigma_y)$, el término de interacción $\hat{H}_{qr}$ se puede dividir en dos términos, el de Jaynes-Cummings (JC) y el anti-Jaynes-Cummings (AJC):

\[
\hat{H}_{qr} = \hat{H}_{qr}^{JC} + \hat{H}_{qr}^{AJC} = g(\sigma_+ a + \sigma_- a^\dag) + g(\sigma_+ a^\dag + \sigma_- a)
\]

Este Hamiltoniano describe el modelo cuántico canónico de Rabi (canonical quantum Rabi model - QRM). Las ecuaciones ()() son completamente generales y aplicables a cualquier sistema qubit-oscilador. Mantener sólo el término JC correponde a realizar la aproximación de onda rotativa (rotating wave approximation - RWA).

\section{Hamiltonianos multiqubit de transmones}
Omitiendo el término del oscilador, el Hamiltoniano toma la siguiente forma general:

\[
\hat{H} = \hat{H}_q + \hat{H}_{qr} + \hat{H}_{qq} = -\frac{1}{2} \sum\limits_i \epsilon_i \sigma_{zi} + \sum\limits_i g_i \sigma_{xi} (a+a^\dag) + \frac{1}{2} \sum\limits_{i,j;\nu} \lambda_{\nu,ij} \sigma_{\nu i} \sigma_{\nu j}
\]

Por simplicidad, se considera que el término $\hat{H}_{qr}$ se refiere sólo a la lectura y las operaciones de bus, dejando la interacción indirecta qubit-qubit via el resonador ser incluidas en $\hat{H}_{qq}$ via la constante de acoplamiento $\lambda_{\nu,ij}$.

\subsection{Acoplamiento capacitivo}
\begin{align*}
\hat{H}_{qq} = \lambda_{1 2} \sigma_{x1} \sigma_{x2} \\
\lambda_{1 2} = \frac{1}{2} \sqrt{E_{1 0, 1} E_{1 0, 2}} \frac{\sqrt{E_{E_{C1}} E_{E_{C2}}}}{E_{Cc}} = \frac{1}{2} \sqrt{E_{1 0, 1} E_{1 0, 2}} \frac{Cc}{\sqrt{C_1 C_2}} \approx \frac{1}{2} E_{1 0} \frac{C_c}{C} \\
\hat{H}_{qq} = \lambda_{1 2} (\sigma_{+1} \sigma_{-2}  + \sigma_{-1} \sigma_{+2})
\end{align*}

\subsection{Acoplamiento por el resonador}
\begin{align*}
\hat{H}_{qq} = \lambda_{1 2} \sigma_{x1} \sigma_{x2} \\
\lambda{1 2} = \frac{1}{2} g_1 g_2 (\frac{1}{\Delta_1} + \frac{1}{\Delta_2} \equiv g_1 g_2 \frac{1}{\Delta}) \\
\Delta_i = \epsilon_i - \hbar \omega
\end{align*}

\subsection{Acoplamiento de JJ}
\begin{align*}
\hat{H}_{qq} = \lambda_{1 2} \sigma_{y1} \sigma_{y2} \\
\lambda_{1 2} \approx \frac{1}{2} E_{1 0} \frac{L_c}{L_J} \frac{\cos(\delta_c)}{2L_c \cos(\delta_c) + L_{J c}}
\end{align*}

\subsection{Acoplamiento afinable/calibrable}

\section{Compuertas cuánticas en transmones}

\subsection{El operador de evolución temporal}
La evolución temporal de un sistema complejo (many-body) puede ser descrita por la ecuación de Schrödinger para el vector de estado $\ket{\psi(t)}$:

\[
i \hbar \frac{\partial}{\partial t} \ket{\psi(t)} = \hat{H}(t) \ket{\phi(t)}
\]

en términos del operador evolución $\hat{U}(t,t_0)$

\[
\ket{\psi(t)} = \hat{U}(t,t_0) \ket{\psi(t_0)}
\]

determinado a partir del Hamiltoniano complejo (many-body) dependiente del tiempo del sistema:

\[
\hat{H} = \hat{H}_{syst} + \hat{H}_{ctrl}(t)
\]

describiendo el sistema intrínseco y las operaciones de control aplicadas. Las compuertas son el resultado de aplicar pulsos de control específicos a partes selectas de un circuito físico. Esto afecta varios términos del Hamiltoniano, haciéndolos dependientes del tiempo.

Para el transmón, el Hamiltoniano del sistema bajo la RWA toma la forma:

\[
\hat{H}_{syst} = -\frac{1}{2} \sum\limits_{\nu i} \epsilon_i \sigma_{z i} + \sum\limits_{i} g_i (\sigma_{+ i} a + \sigma_{- i} a^\dag) + \hbar \omega a^\dag a + \frac{1}{2} \sum\limits_{i,j;\nu} \lambda_{\nu, ij} (\sigma_{+ i} \sigma_{- j} + \sigma_{- i} \sigma_{+ j})
\]

y el término de control se puede escribir como:

\[
\hat{H}_{ctrl} = \sum\limits_{i; \nu} f_{\nu i}(t) \sigma_{\nu i} + \frac{1}{2} \sum\limits_{i,j;\nu} h_{\nu, ij}(t) \sigma_{\nu i} \sigma_{\nu j} + k(t) a^\dag a
\]

\subsection{Pulsos de microondas}

$$\hat{H}_d = \sum\limits_k (a+a^\dagger) (\xi_k e^{-i\omega_d^{(k)}t} + \xi_k^*e^{i\omega_d^{(k)}t})$$

RWA: $$\hat{H}_d=\sum\limits_k a\xi_k^*e^{i\omega_d^{(k)}t}+ a^\dagger\xi_ke^{-i\omega_d^{(k)}t}$$

\subsection{Régimen rotacional del pulso}

Trabajando con un sólo modo a la vez, se aplica la siguiente transformación $U(t) = exp[-i \omega_d t(a^\dagger a + \sum\limits_i \sigma_{z i})]$ para entrar en el régimen rotacional del pulso de control.

$$\hat{H} = U^\dagger (\hat{H}_{syst} + \hat{H}_d) U - i U^\dagger \dot{U}$$
$$ \hat{H} = \Delta_c a^\dagger a + \frac{1}{2} \sum\limits_i \Delta_{qi} \sigma_{zi} + \sum\limits_i g_i (a \sigma_{+ i} + a^\dagger \sigma_{- i}) + (a\xi^*e^{i\omega_d t}+a^\dagger\xi e^{-i\omega_d t})$$

$\Delta_c = \omega_c - \omega_d \qquad \quad \Delta_{qi} = \omega_{qi} - \omega_d$

\subsection{Efecto del pulso sobre el qubit}

Luego se aplica el operador de desplazamineto $D(\alpha) = exp[\alpha a^\dagger - \alpha^* a]$ sobre el campo $a$ con $\dot{\alpha} = -i \Delta_c \alpha -i \xi e^{-i \omega_d t}$ para eliminar el efecto directo del pulso sobre la cavidad.

$$\hat{H} = D^\dagger (\alpha) \hat{H}_{old} D(\alpha) -i D^\dagger(\alpha) \dot{D}(\alpha)$$

$$\hat{H} = \Delta_c a^\dagger a + \frac{1}{2} \sum\limits_i \Delta_{qi} \sigma_{zi} + \sum\limits_i g_i (a \sigma_{+i} + a^\dagger \sigma_{-i})$$
$$ + \sum\limits_i g_i (\alpha \sigma_{+i} + \alpha^* \sigma_{-i}) - \Delta_c \alpha \alpha^* $$

El término $-\Delta_c \alpha \alpha^*$ se desprecia, ya que sólo representa una fase global en la evolución del sistema.

\subsection{Régimen dispersivo}

Finalmente, aplicamos la transformación $U = exp[\sum\limits_i \frac{g_i} {\Delta_i} (a^\dagger \sigma_{-i} - a \sigma_{+i})]$, donde $\Delta_i = \omega_{qi} - \omega_c$ y realizamos la aproximación de segundo grado sobre los términos $\frac{g_i}{\Delta_i} \ll 1$.

$$\hat{H} = U^\dagger \hat{H}_{old} U$$
$$\hat{H} \approx \tilde{\Delta}_c a^\dagger a + \frac{1}{2} \sum\limits_i \tilde{\Delta}_{qi} \sigma_{zi} + \sum\limits_i (\Omega_i \sigma_{+i} + \Omega_i^* \sigma_{-i})$$
$$+ \sum\limits_{i \neq j} \frac{g_i g_j}{2 \Delta_i} (\sigma_{-i} \sigma_{+j}+\sigma_{+i} \sigma_{-j})$$

$\tilde{\Delta}_c = (\omega_c + \sum\limits_i \chi_i \sigma_{zi}) - \omega_d
 \qquad
 \tilde{\Delta}_{qi} = (\omega_{qi} + \chi_i) - \omega_d
 \qquad
 \chi_i = \frac{g_i^2}{\Delta_i}$

\subsection{Rotaciones X-Y}

Tomando $\Omega(t) = \Omega^x(t) \cos(\omega_d t) + \Omega^y \sin(\omega_d t)$, donde $\omega_d$ es igual a la frecuencia de resonancia de uno de los qubits logramos rotaciones sobre los ejes X e Y. Las amplitudes de estas rotaciones vienen dadas por $\int_0^{t_0} \Omega^x(t) dt$ y $\int_0^{t_0} \Omega^y(t) dt$, respectivamente, donde $t_0$ es la duración del pulso.

$$\hat{H} \approx \tilde{\Delta}_c a^\dagger a + \frac{1}{2} \tilde{\Delta}_q \sigma_z + \frac{1}{2} (\Omega^x(t) \sigma_x + \Omega^y(t) \sigma_y)$$

\subsection{Compuerta de entrelazamiento}

Ejemplo con sólo dos qubits

$$\hat{H} \approx \frac{1}{2} \tilde{\Delta}_{q_1} \sigma_{z_1} + \frac{1}{2} \tilde{\Delta}_{q_2} \sigma_{z_2} + \frac{g_1 g_2 (\Delta_1 + \Delta_2)}{2 \Delta_1 \Delta_2} (\sigma_{-_1} \sigma_{+_2} + \sigma_{+_1} \sigma_{-_2})$$

Variando la frecuencia de resonacia de los qubit, se puede variar el acoplamiento entre estos. 

\subsection{Compuertas compuestas}

Con los transmones se pueden realizar rotaciones X-Y y la compuerta iSWAP. Sin embargo, los algoritmos no se escriben en función de sólo estas compuertas, también se necesitan H, CNOT, entre otras. Entonces, debemos construir estas otras compuertas en función de Rx, Ry e iSWAP. Esto es posible, ya que con secuencias de rotaciones en X e Y se puede realizar cualquier compuerta de un sólo qubit, ya que estas consisten de rotaciones en la esfera de Bloch y con rotaciones sobre dos ejes ortogonales se pueden lograr rotaciones sobre cualquier eje en la esfera. Luego, con un set universal de compuertas de un sólo qubit y una compuerta de entrelazamiento, se tiene un conjunto universal de compuertas cuánticas.

Las otras compuertas que necesitaremos, se construyen de la siguiente manera:


