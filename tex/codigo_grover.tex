\chapter{Códigos de la simulación del algoritmo de Grover}
\label{ch:grovercod}

\section{Wolfram Mathematica}

El siguiente fragmento de código genera una lista de números entre 1 y 16, en orden aleatorio, donde ningún número se repite.

\begin{doublespace}
\noindent\(\pmb{\text{data}=\{\};}\\
\pmb{\text{While}[\text{Dimensions}[\text{data}][[1]]<2^4,}\\
\pmb{\qquad\text{AppendTo}[\text{data},\text{RandomInteger}[\{1,2^4\}]];}\\
\pmb{\qquad\text{data}=\text{DeleteDuplicates}[\text{data}];]}\)
\end{doublespace}

Ahora declaramos el número que se buscará en la lista.

\begin{doublespace}
\noindent\(\pmb{\text{buscar}=16;}\)
\end{doublespace}

Se procede a definir las matrices necesarias para ejecutar el algoritmo. Es decir: Los kets de la base computacional, la compuerta y la transformada de Hadamard y los operadores de reflexión.

\begin{doublespace}
\noindent\(\pmb{\text{ket0}=\{\{1\},\{0\}\};}\\
\pmb{\text{ket1}=\{\{0\},\{1\}\};}\\
\pmb{\text{Id}=\text{IdentityMatrix}[2];}\\
\pmb{H=\text{HadamardMatrix}[2];}\\
\pmb{\omega =\text{Position}[\text{data},\text{buscar}][[1,1]];}\\
\pmb{\text{ket$\omega $}=\text{Table}\left[\{0\},\left\{2^4\right\}\right];}\\
\pmb{\text{ket$\omega $}[[\omega ,1]]=1;}\\
\pmb{\text{kets}=\text{KroneckerProduct}[H,H,H,H].\text{KroneckerProduct}[\text{ket0},\text{ket0},\text{ket0},\text{ket0}];}\\
\pmb{\text{U$\omega $}=\text{KroneckerProduct}[\text{Id},\text{Id},\text{Id},\text{Id}]-2\text{ket$\omega $}.\text{ConjugateTranspose}[\text{ket$\omega
$}];}\\
\pmb{\text{Us}=2\text{kets}.\text{ConjugateTranspose}[\text{kets}]-\text{KroneckerProduct}[\text{Id},\text{Id},\text{Id},\text{Id}];}\\
\pmb{G=\text{Us}.\text{U$\omega $};}\)
\end{doublespace}

Finalmente, se prepara el estado inicial y se realizan las reflexiones para ejecutar el algoritmo. Se realizan $\pi N /2$ iteraciones, el doble de las necesarias, para poder estudiar lo que ocurre si sigue rotando el estado, pasado el máxima probabilidad de medir el estado deseado.

\begin{doublespace}
\noindent\(\pmb{\text{ket$\psi $}=\text{KroneckerProduct}[\text{ket0},\text{ket0},\text{ket0},\text{ket0}];}\\
\pmb{\text{ket$\psi $}=\text{KroneckerProduct}[H,H,H,H].\text{ket$\psi $};}\\
\pmb{\text{ket$\psi $}_0=\text{ket$\psi $};}\\
\pmb{\text{For}\left[i=1,i<2\frac{\pi }{4}\sqrt{\text{Dimensions}[\text{ket$\psi $}][[1]]}+1,i\text{++},\text{ket$\psi $}_i=G.\text{ket$\psi $}_{i-1};\right]}\)
\end{doublespace}

\begin{doublespace}
\noindent\(\pmb{n=\text{Floor}\left[\frac{\pi }{4}\sqrt{\text{Dimensions}[\text{ket$\psi $}][[1]]}\right];}\\
\pmb{\text{toplot}_{\text{j$\_$}}\text{:=}\text{Table}\left[\text{Abs}\left[\text{ket$\psi $}_i[[j,1]]\right]{}^2,\{i,0,2n+1\}\right];}\\
\pmb{\text{ListLinePlot}\left[\text{Table}\left[\text{toplot}_i,\{i,1,16\}\right],\text{PlotRange}\to \text{All}\right]}\)
\end{doublespace}

%\subsubsection*{Medir y probar resultado}
%
%\begin{doublespace}
%\noindent\(\pmb{\text{RandomChoice}\left[\text{Flatten}\left[\text{Abs}\left[\text{ket$\psi $}_n\right]{}^2\right]\to \text{Table}\left[i,\left\{i,1,\text{Dimensions}\left[\text{ket$\psi
%$}_n\right][[1]]\right\}\right],20\right]}\\
%\pmb{\text{Position}[\text{data},\text{buscar}]}\\
%\pmb{\text{data}[[\text{Position}[\text{data},\text{buscar}][[1,1]]]]}\)
%\end{doublespace}
%
%\begin{doublespace}
%\noindent\(\{1,1,1,1,1,1,1,14,1,1,1,1,1,1,1,1,1,1,1,1\}\)
%\end{doublespace}
%
%\begin{doublespace}
%\noindent\(\{\{1\}\}\)
%\end{doublespace}
%
%\begin{doublespace}
%\noindent\(16\)
%\end{doublespace}

\section{Python}

Primero, se importan todos los módulos necesarios. Entre ellos, Numpy para variables y funciones numéricas, tgates para las compuertas de transmones.

\begin{verbatim}
import matplotlib.pyplot as plt
import matplotlib as mpl
import numpy as np
from scipy.stats import norm
from qutip import *
import tgates
import time
\end{verbatim}

Ahora se definen las operaciones que son explícitas para este algoritmo. Ellas son la transformada de Hadamard y los operadores de reflexión.

\begin{verbatim}
def Htrans(psi0):
    res = tgates.H(psi0, 0)
    res = tgates.H(res.states[-1], 1)
    res = tgates.H(res.states[-1], 2)
    return tgates.H(res.states[-1], 3)

def Us(psi0):
    res = Htrans(psi0)
    res = tgates.CCCP(res.states[-1], 1, 2, 3, 0, np.pi, b = 0b00)
    return Htrans(res.states[-1])

def Uomega(psi0):
    return tgates.CCCP(psi0, 0, 1, 2, 3, np.pi, b = 0b11)

qN = 2**4
\end{verbatim}

Finalmente, se define el estado inicial y se ejecuta el algoritmo. Se realizan el doble de interaciones de las necesarias, igual que en el código de Wolfram Mathematica, para analizar lo que ocurre si se realizan más rotaciones de las necesarias.

\begin{verbatim}
psi0 = tensor(basis(2,0), basis(2,0), basis(2,0), basis(2,0))

res = Htrans(psi0)

Nit = 2*int(np.pi*np.sqrt(qN)/4)+1

for i in range(Nit):
    
    res = Uomega(res.states[-1])
    res = Us(res.states[-1])
    
    qsave(res, 'it_{}'.format(i+1))

\end{verbatim}

