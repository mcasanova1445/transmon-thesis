\chapter{S}

Uno de los efectos más espectaculares de la física del estado sólido es el efecto superconductor. La superconductividad produce efectos cuánticos macroscópicos, siendo un campo perfecto para estudiar aspectos básicos en la física cuántica.

Empezaremos este capítulo describiendo los fenómenos que caracterizan un superconductor.

En 1911 Kamerlingh Onnes descubre que el Hg conduce sin resistencia cuando se enfría a temperatura del He líquido (4.2 K). Kamerlingh Onnes fue el primer físico que licuó el He, poco tiempo después de esto, midiendo resistividades de distintos metales a estas bajas temperaturas, se encontró, en su laboratorio, que el Hg a 4.2K conduce sin resistencia, no con una resistencia despreciablemente pequeña, sino con resistencia cero.

Una primera característica de la superconductividad es la nula resistividad por debajo de una cierta temperatura.

Aunque se pudiera pensar que esto es una excepción resulta que el número de elementos y materiales superconductores es muy grand. El sistema periódico está lleno de elementos que, si se disminuye suficientemente la temperatura, se convierten en superconductores. En realidad la pregunta no sería por que un elemento es superconductor, más bien hay que preguntarse por que no lo es. Ahora bien, las temperaturas a las cuales ocurre el efecto son extraordinariamente bajas. La mayoría de los superconductores tienen temperaturas de transición que están en el rango del He liquido (4.2K), lo cual quiere decir que el efecto es muy débil. Esto hay que matizarlo a la vista de los superconductores recientemente descubiertos (1986) y llamados de alta temperatura. En este capítulo no haremos ninguna referencia  estos nuevos superconductores.

Ahora bien, un superconductor es bastante más que un conductor perfecto. En 1913 se comprobó que si se aplica un campo magnético exterior al material en estado superconductor, se terminaba por destruir el estado de conducción perfecta. Este campo, cuyo valor depende de la temperatura, se llama campo crítico. Tenemos así el primer indicio de que la interacción superconductividad-magnetismo juega un papel primordial en los fenómenos superconductores.

La siguiente característica fundamental de un superconductor consiste en que cuando el campo magnético aplicado no es mayor que el campo  crítico un superconductor actúa como un diamagnético perfecto. Esto es $\chi = -1$. Por lo tanto, no existen líneas de campo magnético en el interior de un material superconductor, salvo en una pequeña zona próxima a la superficie. Este efecto se conoce con el nombre de efecto Meissner-Ochsenfeld.

Así la segunda característica del efecto superconductor es el diamagntismo perfecto (efecto Meissner).

Otra característica típica de la superconductividad es que el flujo del campo magnético que atraviesa un anillo superconductor está cuantizado. Es decir, el flujo que atraviesa un superconductor vale un número entero de veces una unidad de flujo elemental llamada el fluxoido $\Phi_0$ y cuyo valor es $\Phi_0 = h /2 e$, donde $h$ es la constante de Planck, c la velocidad de la luz y e la carga del electrón.

Por lo tanto, la tercera característica de la superconductividad es que el flujo magnético qu eatravieas un superconductor está cuantizado.

Existe otro efecto cuántico macroscópico asociado a la superconductividad, conocido como el efecto Josephson. Este efecto es un efecto túnel muy especial. Si se separan dos superconductores distintos, o bien el mismo superconductor por una barrera, por ejemplo un aislante o un estrechamiento en el superconductor (lo que se conoce como una unión débil), se tiene paso de corriente por la barrear sin que aparezca caída de potencial a ambos lados de la barrera.

La cuarta característica de la superconductividad es el efecto Josephson, esto es, la existencia de un efecto túnel superconductor.

Existen otros efectos que completan el panorama experimental de la superconductividad, pero los báscios que aparecen siempre son los señalados unas líneas arriba. El resto de fenómenos experimentales superconductores se detallan en la siguiente sección.

\subsection{La teoría BCS}

En 1957 John Bardeen, Leon Cooper y J. Robert Schieffer, encontraron la llave para poder explicar, desde un punto de vista microscópico, la superconductividad. Realmente existen bastantes materiales que se desvian de la teoría estricta BCS, como pueden ser aleaciones, compuestos y algunos elementos (el mismo Hg, donde se descubrió la superconductividad, es un ejemplo de superconductor que no cumple los estrictos requisitos de la teoría BCS).

Esto no significa que la teoría sea incorecta, tan sólo que es algo incompleta y para tratar algunas situaciones hay que recurrir a levantar alguna de las aproximaciones que esta complicada teoría lleva consigo.

Empezaremos por repasar alguno de los hechos experimentales en que se basa la teoría y señalar algún otro que hasa el momento no ha sido mencionado.

En principio lo más característico de la superconductividad, como ya hemos repetido varias veces, es que por debajo de una cierta temperatura y en ciertas condiciones de densidad de corriente eléctrica y campo magnético aplicado, no se tiene resistencia, es decir que los electrones de conducción no cambian su momento, su $k$: pero no es esta la única característica fundamental de un superconductor, también sabemos que un superconductor es un diamagnético perfecto y estte efecto conocido como efecto Meissner nos indica que un superconductor es mucho más que un conductor perfecto. si recordamos en qué consiste el efecto Meissner, tenemos que cuando enfriamos un superconductor por debjo de la temperatura crítica y en presencia de un campo magnético menor que el crítico, tenemos que el caampo magnético es excluido del superconductor y este estado es de equilibrio termodinámico. En un conductor perfecto el campo magnético quedaría atrapado en el material sin que haya ninguna razón para que sea explulsado, basta con recordar las ecuaciones de Maxwell para verlo.

Por lo tanto estamos en una situación bastante especial. Como veremos en su momento a un superconductor lo que le caracteriza es una función de onda macroscópica que da lugar a una corriente cuántica macroscópica y el efecto Meissner es justamente la expulsión del flujo magnético por esta corriente cuántica. No vamos a decir nada más acerca del efecto Meissner desde el punto de vista microscópico, dado lo complicado que es. Por el contrario, la conducción sin resistencia eléctrica se puede deducir de una manera mucho más sencilla con le formalismo de la teoría microscópica BCS.

Como se ha señalado unas lineas más arriba aparte de estos dos hechos fundamenales ddel efecto superconductor (conducción eléctrica sin resistencia y diamagnetismo perfecto) hay otros fenómenos experimentales propios de los superconductores que son cruciales y marcan la pauta a la teoría BCS, como es el efecto isotópico, efecto con el que empezaremos.

Se observa experimentalmente que la temperatura crítica $T_c$ superconductora, por ejemplo en el Hg (primer superconductor y primer efecto isotópico encontrado), depende de la masa de los iones en el metal. Dado que existe varios isótopos del Hg se pudieron preparar varias muestras y se comprobó experimentalmente que si $M$ es la masa del ion se tiene que $T_c proportional M^{-\alpha}$ donde el exponente $\alpha$ es del orden de 0.5 para metales que no son de tansición y puede ser emnor que este valor para metales de transición; incluso existen casos especiales como es el Os donde vale cer, no existiendo efecto isotópico para esta excepción. Este efecto experimental tiene una interpretación bien sencilla: los iones de la red juegan un papel fundamental en la superconductividad.

Existe otro grupo de fenómenos experimentales asociados al estado superconductor, que pasaremos a describir a continuación, que tienen todos ellos el mismo origen.

\begin{enumerate}
    \item El calor específico electrónico de un metal no superconductor a muy baja temperatura varía linealmente con la temperatura. En un metal existen dos contribuciones al calor específico, una de ellas es la normal de la red. Esta contribución varía como $T^3$ y prácticamente desaparece a bajas temperaturas. Exite otra contribución al calor específico propia de los metales. A esta contribución, que es la predominante a bajas temperaturas, es a la que nos referimos y varía como $T$.
    \item Un metal en estado normla no es transparente a la radiación visible e infrarroja.
\end{enumerate}

Por el contrario, en un metal en el estado superconductor se tiene:

\begin{enumerate}
    \item La variación del calor específico en función de la temperatura sigue una ley exponencial.
    \item En un metal en estado superconductor existe, en el rango del infrarrojo lejano, una frecuencia de corte para la cual el metal es transparente a esa radiación.
\end{enumerate}

Estos dos efectos en superconductores son la clara señal de que en un superconductor existe una zanja prohibida de energía, que, como veremos en su momento, no es del mismo tipo que la que existe en semiconductores. Otra comprobación experimental de este hecho son las anomalías en el efecto túnel conocido com otúnel Giaver.

Resumiendo, existen dos hechos experimentales cruciales para entender la superconductividad, uno de ellos es que la masa de los iones juega un papel fundamental (efecto isotópico). El segundo es que en un superconductor existe una zanja prohibida de energía, esto es que mientras que en un metal en estado normal no se necesita una energía umbral ara tener estados excitados por encima del estado fundamental, en un metal en estado superconductor se necesita comunicar una cierta energía por encima de un cierto valor, el valro de la zanja, para tener a los electrones en estado excitados de energía.

Por lo tanto, simplificando, cualquier teoría microscópica de la superconductividad tiene que dar cuenta de:

\begin{enumerate}
    \item Condctividad infinita, esto es ausencia de resistividad.
    \item Intervención de los iones de la red cristalina, más precisamente, de sus masas. Esto es de los fonones, es decir de las diferentes vibraciones de la red cristalina.
    \item La existencia de una zanja de energía en la banda de conduccion.
\end{enumerate}

A partir de ahora solamente vamos a cnetrarno en descubrir la posible interacción responsable de que un metal conduzca sin resistencia eléctrica por debajo de un acierta temperatura.

Lo primero de todo es recordar que los electrones de conducción, por su movimiento al azar en una red cristalina, donde los iones están vibrando (fonones) cambian su momento, $k$, esto es la interacción electrón-fonón es la que produce la resistividad. Que un metal conuzca sin resistencia eléctrica querrá decir que los electrones de conucción en su movimineto en el cristal no cambian su vector $k$.

Se trata por lo tanto de encontrar una interacción en los electrones de conducción que nos produzca este efecto. En un cristal contamos aparentemente con pocos recursos, por un lado enemos electrones de conducción e iones que forman la red cristalina y que están fibrando, de otra forma con electrones y fonones y con una interacción fundamental entre estas cargas eléctricas, que es la interacción coulombiana. Pues bien, son estos únicos ingreidnetes los necesarios para construir la teoría BCS, electrones, fonones e interacción coulombiana.

La interaccion coulombiana electrón de conducción-red cristalina (iones) se puede ver esquemáticmanete de la siguiente forma. Supongamos un electrón de conducción que se mueve por el cristal y fijémonos en un punto concreto de la red, formada por iones positivos (cationes). Al pasar el electrón por ese punto la red, por interacción coulombiana, se sentirá atraido por ese electrón y se deformará localmente. Ya uq elas frecuencias de vibración de la red (de los fonones) son del orden de $10^{-13} s^{-1}$ y las velocidades de los electrones de conducción son del orden de $10^{16} Å/s$, ocurrirá que cuando la red vuelva a su posicíón de equilibrio, el electrón que la ha deformdo se encontrará muy lejos, del orden de $1-^3 Å$, esto es del orden de varios cientos de parámetros de la red del ion que ha dejdo fuera de su posición de equilibiro. Esto es así porque las velocidades de los electrones de conducción en su movimeinto al azar, velocidades de Fermi, son muy grandes comparadas con los teimpos de relajación de l red, ligaods a las frecuencias de los fonones. El segundo paso en este mecanismo es muy sencillo: Mientras que la red está deformada por sus cercanías pasan uchos otros electrones de conducción, que se sienten más <<cómodos>>, más atraídos por la red, que sin éßta no estuviera deformada. En resumen, se puede establecer una cierta conexión, una cierta interacción atraactiva entre los electrones de conducción vía los fonones (vibraciones de la red). Esto es, un electrón de conducción deforma la red y un segundo electrón de conducción se siente atraido por esta deformación y en cierta manera <<ligado>> al primer electrón, electrón que puede estar físicamente muy alejado del primero. Esta idea de cómo una interacción de corto alcance electrón-fonón, puede dar lugar a una interacción de largo alcance electrón-electrón, se debe a Fröhlich (1950) y es la base de la teoría de la superconductividad.

En principio tenemos un posible mecanismo de interacción atractiva enter electrones de doncucción vía fonones, que de momento no parece que teng mucho que ver con la superconductividad. Antes de seguir avanzando se tiene que recordar el papel que juega la interacción coulombiana repulsiva directa entre los electrones de coonducción. Ahora bien esta interacción coulombiana repulsiva va como $1/r^2$, siendo $r$ la distancia entre los electrones, es deccir que disminuye rápidamente con la distancia. Luego se tendrá que la interacción neta entre los electones de conducción sera la suma de estas dos interacciones una repulvisa, que esla que podríamos deir normal y otra, vía fonón, que es atractiva. La interacción neta será atractiva solamente cuando estemos haciendo el balance entre electrones cuya interacción coulombiana repulsiv se amuy pequeña, es decir entre electrones muy alejados unos de otros. Este caso se puede dar, como hemos resaltado una líneas más arriba, ya que se puede todavía tner un aligadur vía fonones entre electrones muy alejados, varios cientos de parámetros de red con interaacción repulsiva coulombiana muy pequeña.

Una vez entendido lo que entecede lo que sigue es muy sencillo. El siguiente paso lo da Cooper, que demuestra que si se tienen dos electones que interaccionan con un ainteracción atractiva enta, es decir negativa, aunque ésta sea todo lo peuqeña que se queira, el mar de Fermi de los electrones de conducción es inestrable y se produce un estado ligado con momentos $k$ y espines opuestos, llamdo par de Cooper. Bardeen, Cooper y Schieffer, trabajando juntos, escriben un hamiltoniano y el estado fundamental formado por la condecnsación de parejas de electrones, pares de Cooper, de tal manera que son capaces de desarrollar expresiones para la temperatura crítica superconductora, deducen la existencia de una zanja de energía en los superconductores, tal que para romper una pareja de electrones ligados via la interacción con los fonones de la red, hay que suministrar una energía superior o igual a la de esa zanja, y asimismo deducen el efecto isotópico. Queda por último ver de dónde se obtiene con esta teoría conducción eléctrica sin resistencia, es decir sin que los electrones de conducción cambien de momento (esto es $k$). Hay que señalar que el estado fundamentl propuesto por la teoria BCS no está formado por un conunto de parejas de Cooper cualesqueira, es algo muy complicado de tal manera que la interacción atractiva entre electrones, que produce una inestabilidad en el mar de Fermi hace que se condensen pare de Cooper, estando la función de onda superconductora formada por todos los pares de Cooper. Los pares no actúan independientemente unos de otros y para romper un pr, esto es para convertir a los dos electones en electrones digamos, <<normales>> hay que suministrar energía (la correspondiente a la zanja) de tipo térmico, subiendo la temperatura, o bien de tipo magnético, aplicando un campo magnético, etc. Mientras que no se suminisre una energía superior a la de la zanja los electrones superconductores están formando pares de Cooper, es decir se mueven con momento $k$ bien determinado y no lo cambian. Esto es, conducen sin resistencia a no ser que se rompan los pares, que no olvidemos no son independientes. Ahora bien, estos electrones se están moviendo en un sólido, en una red cristalina, que estará a una cierta temperatura (por debajo, por supuesto, de la temperatura crítica) que no es suficiente para romper los pares. Por lo tanto, verán fonones e interaccionarán con ellos. ¿Cómo es posible que un electrón de una pareja que interacciona con un fonón (de energía menor de la necesaria para romper el par) no cambie su momento? Esto es debido a que los pares están interrelacionados y el resto se acomoda para hacer posible que en su conjunto de $k$ no varíe. Aquí se puede utilizar una imagen debida a Schieffer. Supongamos que tenemos un conjunto de esquiadores, la mitad hombres y la mitad mujeres, que bajan una pendiente cogidos de la mano, hombre con mujer (aquí tenemos los pares de Cooper), pero estas parejas no bajan cada uno por su lado, sino que lo hacen al mismo tiempo y de alguna manera enlazados, de tal manera que si algún miembro de la pareja se encuentra en el camino un obstáculo (un fonón), siempre que no sea muy importante (de energía menor que la de la zanja) el conjunto de pares aguantarán el golpe y harán posible que se siga bajando sin perder velocidad, sin cambiar $k$, sin resistencia.

Resulta llamativo que la interacción electrón fonón sea el origen de la resistividad y que la interacción electrón-fonón-electrón que acabamos de bosquejar resulta en el origen de la superconductividad.

No resulta nada simple el desarrollar, utilizand el formalismo adecuado, las ideas expuestas en las líneas anteriores, pero por ejemplo la función de onda superconductora y algunos otros detalles se pueden tratar explícitamente en el marco de este capítulo.

Sin duda existen aspectos que pueden parecer caprichosos, en la descripción que acabamos de hacer. Uno de ellos es sin duda que las parejas de Cooper estén formadas por electrones con momentos opuestos y con espines también opuestos, esto es muy fácil entenderlo.

El esquema de la teoría BCS, en sus primeros pasos, es bien simple. Se parte de considerar como el núcleo de la teoría la interacción electrón-fonón-electrón. Esto se puede representar de la siguiente forma: El primer electrón llega a un punto de la red y atrae a la red y hace que ésta <<vibre>>, esto en nuestro lenguaje se dice que emite un fonón. El segundo electrón <<ve>> este fonón y lo absorbe acoplándose así con el primer electrón. Es decir, los electronesse acoplan intercambiando un fonón, esto se puede representar con uno de los típicos diagrmas de Geynman, cono se ve en la figura (FIGURA). Este tipo de esquemas se encuentran en otras interacciones en la Naturaleza. Por ejemplo, en el núcleo atómico tenemos en este mismo caso en el intercambio de mesones $\pi$ virtuales que produce atracción entre nucleones.

Ahora, ¿Cuál será el tipo de electrones y fonones que están interaccionando? En principio los electrones claramente son los de la banda de conducción, con energías del orden de unos pocos electrón voltio. Los fonones disponibles tienen unas energías que son como máximo del orden de l aenergía de Debye. Dado que la temperatura de Debye suele ser del orden de algo más de 100K, se tiene que las energías de los fonones disponibles son de unos pocos milielectrón voltio, esto es muy pequeña comparada con la de los electrones. se tiene que la capa de los electrones de conducción en la esfera de Fermi que intercambian fonones, según este esquema es una franja muy estrecha.

%%%%% Diagrama %%%%

Si llamamos $\hbar k, \hbar k^\prime$ y $\hbar k_1, \hbar k_1^\prime$, los momentos de los electrones antes y después de la itneracción la ley de conservación del momento nos dice que

\begin{equation*}
    \hbar k_1 + \hbar k_1^\prime = \hbar k = \hbar k^\prime = \hbar K
\end{equation*}

donde hemos llamado $K$ al momento del centro de masas de los electrones. Conviene aprovechar este punto para indicar el convenio habitual en cuanto al signo de la energía de los electrones. Se toman como energías positivas los valores de la energía superiores a la energía de Fermi y como energías negativas los valores inferiores a la energía, $E_F$, de Fermi. Es decir qeu $\epsilon_k$ será negativa si $k$ está dentro de la superficie d Fermi y positiva en caso contrario. Está claro que los únicos electrones que pueden intervernir en todo este proceso son los que tienen energías próximas a la de Fermi dado que solamente estos lectrones pueden tener estados libres accesibles sin violar el principio de exclusión de Pauli. Por lo tanto, estamos solamente considerando como candidatos a electrones superconductores unos pocos de todos los de la esferea de electrons de conducción de Fermi, aquellos que están en una estrecha franja d espesor $\hbar \omega_D$ donde $\omega_D$ es la frecuencia de Debye, por lo tanto del orden de unos pocos milielectrón voltio de la energía de Fermi. En esta situación se conserva el momento $\hbar K$ del centro de masas de los dos electrones. En la figura (FIGURA) se representa gráficamente todo lo qeu acabamoos de decir, además se observa en esta figura qeu son, dentro de l aestrecha franja, muy pocos los electrones qeu cumplen todas las condiciones y qeu por lo tanot pueden ser candidatos a formar pares de Cooper. Solamente los electrones cuyos momentos caen en la intersección de las dos capas esféricas son los posibles electrons superconductores. Es fácil ver qeu si disminuimos el valro del momento del centro de masas del sistema la intersección aumentará y se hará máxima si hacemos $K = 0$, donde toda la franja está formada por posibles pares de Cooper. Es decir los pares de Cooper se forman con electrones de momentos opuestos, como ya habíamos anicipado unas líneas más arriba, esto es $k, -k$.  Además, este argumento que acabamos de desarrollar está apoyado en que la energía del sistema disminuye al ir formándose pares, lo que e puede demostrar, pero no de un a manera fácil.

Otro aspecto de los señalados anteriormente, que podemos abordar ahora, es qeu los pares estaban formados por electrones no sólo con momentos opuestos, sino también con espines opuestos, esta última característica es ahora muy fácil de tratar, dado qeu los electrones son fermiones, es decir cumplen el principio de exclusión de Pauli y tienen funciones de onda antisimétricas. Se puede demostrar que la parte orbital (es decir olvidándose de la parte de espín) de la función de onda de los pares de Cooper, depende tan sólo del módulo del momento. Por lo tanto, frente al intercambio de la posicón de los dos electrones del par se tienen una función simétrica. Luego la parte de espín debe ser antisimétrica, es decir los espines de la pareja son opuestos. Se tiene que un par de Cooper esta formado por dos electrones tal que $k uparrow, -k downarrow$.

En este punto surge uno de los peligros típicos de la teoría BCS, que pone de manifiesto la gran cantidad de sutilezas qeu encierra. Dado que un par de Cooper es una entidad cuyo espin es cero, como los bosones, es fácil caer en la tentación de tratar a los pares de Cooper como bosones. Además hemos indicado qeu un número creciente de pares de Cooper es energéticamente favorable. Ahora bien, el principio de Pauli sigue vigente: Así, el estado formado, por ejemplo, por $k uparraw, -k downarrow$ no uede estar ocupado por más de un par de electrones al mismo tiempo. En el gráfico (FIGURA) se representa la situación del estado fundamental para 3 pares de Cooper. Además, entrando en detalles más técnicos, los operadores con los que se construye el hamiltoniano de la teoría BCS, no siguen las reglas de conmutación de los operadores de bosones. Resulta de todas formas llamativo y conviene resaltarlo que la electrodinámica bosónica reproduce muy bien el comportamiento superconductor, por ejemplo la teoría clásica de la superconductividad de London (que no hemos tratado en este capítulo) se puede deducir de un gas cargado de bosones y de allí se puede extraer de una manera natural el efecto Meissner. Resumiendo, los pares de Cooper no son bosones.

Pasaremos a continuación a describir de la manera más sencilla posible la función de onda del estado fundamenall superconductor y algunas de las expresiones de la teoría BCS.

La función de onda del estado fundamental de $N$ electrones, según la propone la teoría BCS, es el producto de funciones de onda de pares convenientemente antisimetrizadas, que se puede representar por $\phi(1,2,...,N) proportional \phi(1,2)\phi(3,4)...\phi(N-1,N)$ si no escribimos explícitamente la parte de espín y sólo lo hacemos con la parte orbital tendríamos $\phi(1,2,...,N) proportional \sum\limits_{k_1} \sum\limits_{k_2} ... \sum\limits_{k_3} g_{k_1} ... g_{k_{N/2}} e^{i (k_1 r_1 - r_2 k_2 + ... + k_{N/2} r_{N-1} - k_{N/2} r_N}$ donde cada término de esta función de onde describe una configuración donde los N electrones se agrupan en $N/2$ pares que son $(k_1, -k_1) ... (k_{N/2}, -k_{N/2})$ la parte de espín es inmediata cada electrón de cada par tiene espines opuestos. Como vemos la funcion de onda es una función complicada que abarca todos los pares relacionados entre ellos. También se puede escribir de una manera más compacta como $\phi = \prod\limits_k \phi_k$.

Antes de continuar merece la pena señalar algún otro aspecto de los pares de Cooper, estos pares están fuertemente relacionados entre sí, de tal manera qeu se puede decir que del orden de un millón de parestienen sus centros de masa dentro del espacio en el que se extiende un par dado, esto es los pares de electrones qeu forman un par de Cooer están muy alejados uno del otro, estando fuertemente correlacionados unos pares con otros. Se puede demostrar, que la disminución de la energía en la fase superconductora respecto al estado normal, debida a la interacción entre pares, depende de cómo se elijan esos pares. El conjunto de pares de Cooper no son independientes unos d eotros, están muy correlacionados.

Otro punto que hay que aclarar en lo anterior es que estamos considerando el caso en que no tenemos corriente eléctrica neta, ya que los electrones apareados tienen momento toal cero. Los estados portadores  decorriente superconductora son aquellos en que los pares tienen momentos que serán $(k + \frac{q}{2} uparroa, -k + \frac{q}{2} downarrow)$ y los electrones tendrán una velocidad de arrastre neta que será $v_a = \frac{\hbar q}{2m}$.

Otro aspecto importante de la teoria BSC es que predice la existencia de una zanja de energía $\Delta$, zanja que se puede medir experimenalmente y que está relacionada con la temperatura crítica por la ecuación BCS. Este parámetro $\Delta$ es el parámetro crucial de la teoría BCS de que acompaña la aparición del esado superconductor.

Hay que resaltar que esta zanja reúne unas características muy particulares, por lo pronto se diferencia claramente de las zanjas que juen un papel importnate en sólidos, especialmente en los semiconductores. En superconductores tenemos una zanja que está situada en la banda de conducción y que tiene una marcada dependencia con la temperatura. Asimismo, mientras que en un semiconductor se necesita excitar por encima de la zanja a los electrones para tener conducción eléctrica, en un superconductor se tiene la supercorriente sin necesidad de tener estados excitados, se tiene conducción eléctrica por debajo del nivel de Fermi, esto es por debajo de la zanja que existe en la banda de conducción.

Este parámetro $\Delta$ aparece en la sencilla e importante relación que se obtiene en la teoría BCS $2 \Delta(0) = 3.52 k_B T_c$ donde $T_c$ es la temperatura crítica superconductora.

Para finalizar un resumen general de la teoría BCS.

\begin{enumerate}
    \item Interacción atractiva entre electrones.
    \item Mar de Fermi inestable.
    \item Posible formación de estados ligados de dos electrones.
    \item Condensación de parejas de electrones, pares de Cooper $(k uparrow, -k downarrow)$.
    \item Aparición de una zanja de energía (para romper los pares hay que suministrar esa energía).
    \item Temperatura crítica superconductora BCS ligada al valor de la zanja de energía prohibida.
\end{enumerate}


\subsection{Efectos cuánticos macroscópicos en superconductor: Cuantización del flujo magnético y efecto Josephson}

La superconductividad es un campo de la física donde las leyes cuánticas que gobiernan el comportamiento de la naturaleza, se pueden observar a escala macroscópica. El origen de este espectacular efecto veremos que es único, pero existen dos aspectos experimentales en los que se manifiesta. Uno de ellos está ligado a efectos magnéticos y el otro a fenómenos en la densidad de corriente superconductora. El primero de ellso es la cuantización del flujo magnético y el segundo el efecto Josephson. emezando por el primero de estos efetos cuánticos mecroscópicos se tiene qeu el flujo magnético es siempre un número entero de veces el valor de un flujo elemental, conocido con el nombre de fluxoide. Es decir, que el flujo que atraviesa un material superconductor es siempre o uno o 2 o 500 fluxiodes, pero nunca puede ser una cantidad cualquiera, siempre un número entero de fluxoides.

Pasemos a continuacióñ a demostrar este efecto y encontrar el valor del fluxoide.

En la teoría de Ginzburg-Landau de las transiciones de fase se introduce el concepto de parámetro de orde, que es una magnitud qeu aparece acompañando a la transición. Un ejemplo típico de parámetro de orden es la imanación de saturación, que es la magnitud que aparece cuando se tiene la transición de fase del estado paramagnético al ferromagnético. en el caso de una transición de fase del estado paramagnético al ferromagnético. En el caso de una transición superconductora el parámetro de orden es una función completa cuyo módulo al cuadrado nos da la densidad de electrones superconductores que aparecen al pasar el metal del estado normal al superconductor. en el estado normal el parámetro de orden (densidad de electrones superconductores) se desvanece, desaparece.

Esta función compleja se escribe $n_x = \abs{\psi}^2$ donde $n_s$ es la densidad de electrones superconductores.

Se puede escribir explícitamente el módulo y la fase de esta función, $\psi = \abs{\psi} e^{i \psi}$.

Por otro lado, sabemos que la cantidad de movimiento de una partícula de masa $m^*$ y carga $e^*$ en presencia de un campo magnético representado por su potencial vectorial $A$, se escribe como $\mathbf{p} = m* \mathbf{v} + \frac{e^*}{c} \mathbf{A}$.

Si tenemos una densidad de partículas, todas ellas teniendo el mismo momentum $\mathbf{p}$ podemos escribir $n_s \mathbf{p} = n_s (m^* \mathbf{v} + \frac{e^*}{c} \mathbf{A}$.

Recordando el operador momentum en su expresión equivalente $p \rightarrow -\hbar \nabla$ tendremos $p = \hbar \nabla \phi = m^* v + \frac{e^*}{c} A$.

Recordando la expresión general de la densidad de corriente eléctrica en función de la densidad de portadores, de la carga y de la velocidad promedio se pueden escribir $\mathbf{J} = n_s e^* \mathbf{v}$ que en nuestro caso será $\hbar \nabla \phi = \frac{m^*}{n_s e^*} J + \frac{e^*}{c} A$.

Con esta expresión estamos preparados para demostrar la cuantización del flujo magnético en superconductores, para ellos basta con recordar algo trivial, como es que el parámetro de orden superconductor sólo puede tener un único valro en cada punto, es decir, la densidad de electrones superconductores debe ser única en cada punto, esta simple consideración se materializa en que podemos escribir $\phi(2\pi) - \phi(0) = n 2 \pi$.

Recordando la definición de circulación y de gradiente y siendo $C$ un camino cerrado, la expresión anterior la podemos escribir como $\oint\limits_C \nabla \phi dl = n 2 \pi$. Si ahora suponemos que este camino $C$ está en el interior de un superconductor, alejado de los bordes y rodeando a un hueco, como se representa en la figura (FIGURA) y suponemos que tenemos aplicado un campo magnético a este superconductor, se tiene $\oint\limits_C (\frac{m^*}{\hbar n_s e^*} J + \frac{e^*}{\hbar e} A) dl = n 2 \pi$.

Dado que estamos en un camino interior al superconductor y allí no existe corriente (las únicas corrientes que existen en una situación como la que estamos describiendo, están apantallando el campo magnético y están situadas cerca de los bordes tanto de la cavidad como de la superficie del material superconductor), tendremos $\oint A dl = \frac{n h c}{e^*}$.

Recordando el teorema de Stokes se tiene $\oint\limits_C A dl = \iint\limits_S B ds = \Phi$ que en nuestro caso será $\Phi = n \Phi_0$ expresión que nos indica que el flujo magnético que encierra la cavidad es un número entero de un flujo elemental conocido con el nombre de fluxoide y cuyo valor es $\Phi_0 = \frac{h c}{2 e} = 2.07 10^{-7} gauss cm^2$ siendo $e^*$ la carga del portador de corriente, el par de Cooper $e^* = 2e$.

Este efecto fue encontrado experimentalmente de forma simultánea en 1961 por Deaver-Fairbanks y por Döll-Nabauer.

El efecto túnel es un efecto típico del carácter cuántico de los electrones, desde el punto de vista de la física clásica es completamente imposible que se produzca el efecto que vamos a discutir.

Los electrones se pueden representar por funciones de onda, de tal manera que existe una cierta probabilidad de que un electrón pueda ir de un metal a otro atravesando una barrera aislante estrecha, que puede ser vacío o un óxido. La función de onda del electrón decae de una manera exponencial fuera de la superficie del metal, la amplitud de la onda no es totalmetne nula fuera del metal, es como si el electrón se desparramase fuera de la superficie. Si situamos un metal junto al otro, separados tan sólo por una barrera, como puede ser, por ejemplo, el óxido de la superficie, existe una probabilidad pequeña, pero no nula de que el electrón atraviese ese túnel y aparezca al otro lado, en el otro metal.

Antes de seguir hay que hacer notar que la energía necesaria para hacer pasar un electrón que está en el nivel de Fermi de un metal al vacío (función de trabajo del metal) es mayor que la energía necesaria para transferir ese electrón a un aislante.

Por razones de simplicidad vamos a hacer toda la discusión siguiente, salvo cuando se diga expresamente lo contrario, para temperatura de 0K. La figura (FIGURA) nos indica la situación entre dos partes del mismo metal separadas por un aislante. Todos los estados por debajo de la energía de Fermi están ocupados, mientras que todos los estados por encima de $E_F$ están vacíos.

Para que se pueda tener efecto túnel hacen falta dos condiciones. La primera es que, como es lógico, los electrones solamente pueden ir de un estado ocupado a un estado desocupado y la segund aes que se tiene qeu conservar la energía, es decir que las transiciones, en la gráfica, tienen qeu ser horizontales. Por lo tanto en la situación de la gráfica (FIGURA) no tendremos efecto túnel. No se pueden tener transiciones horizontales al no existir estados vacíos. Todos los estados en el mismo nivel de energía, a ambos lados de la barrera, están ocupados.

Si aplicamos una diferencia de potencial constante a la barrera lo que estamso haciendo es aumentando la energía de los electrones de un lado de la barrera respecto al otro y entonces tenemos la posibilidad de que se tenga correitne por efecto túnel, gráfica (FIGURA).

La intensidad de esta correitne túnel depende de varios parámetros. Por ejemplo, a mayor diferencia de potencial aplicada mayor correitne. Está claro que cuantos más estados tengamos en el nivel de Fermi mayor será la probabilidad de tener corriente túnel, lo cual puede indicar que quizáß con experimentos de efecto túnel podemos obtener información sobre este importante parámetro y en general sobre la superficie de Fermi, pero como es de esperar la corriente túnel también depende de la anchura, altura y forma de la barrera y estos parámetros son muy difíciles de determinar, lo cual hace qeu en metales normales del efecto túnel se obtenga una información mucho menos rica de lo que se podía esperar. Ocurre todo lo contrario con el efecto túnel cuando uno de los dos metales está en estado superconductor, como pasaremos a ver a continuacióñ.

La existencia de una zanja de energía en el estado superconductor (FIGURA) hace que el efecto túnel en una estructura formada por superconductor-aislante-metal en estado normal, tenga características especiales (Giaver, 1960). Es claro que necesitamos previamente disponer de electrones normales por encima de la zanja superconductora, esto es hay que romper pares de Cooper, como primera medida, es decir diferencias de potencial aplicadas menores que la zanja no producirán efecto túnel. Esto es, se tiene que diferencias de potencial aplicadas a la barrera no producen corriente túnel, salvo que venzan un valor umbral, qu es precisamente el ancho de la zanja. De entrada ya tenemos una muy importante propiedad del efecto túnel superconductor, que nos permite medir la zanja superconductora y medir la variación de esta zanja con la temperatura.

Otra importante propiedad del efecto túnel es lo que pasa si estudiamos el efecto túnel de dos metales en estado superconductor. Es fácil ver qeu entonces, a 0K, tenemos que aplicar una diferencia de potencial para obtener corriente túnel que será la suma d elas dos zanjas de energía. Pero no es sólo esto lo que ocurre cuando separamos dos supercondcutores por una barrera aislante. Como veremos a continación aparece un nuevo efecto túnel exclusivo de los superconductores. Este es el conocido como efecto Josephson, postulado teóricamente por Josephson (1962) y comprobado experimentalmente por P.W. Anderson y Rowell (1963) y por Shapiro (1963), casi simultáneamente. Este es un efecto túnel de pares de Cooper entre superconductores, mientras que el efecto túnel tratado en las líneas anteriores es túnel de electrones normales entre superconductores o entre un metal normal y un superconductor. En concreto la sugerencia de Josephson es que puede existir efecto túnel entre dos superconductores que se encuentren separados por una barrera aislante (en principio más delgada que las tratadas en el efecto túnel normal, también conocido como túnel Giaver), donde la corriente túnel sea debida exclusivamente a pares de Cooper sin que se tenga una diferencia de potencial a través de la barrera.

Se va a seguir la deducción debida a Feynman, por su sencillez y claridad, porque este efecto enunciado unas lineas arriba y que parece casi mágico, se puede deducir en unas opcas líneas.

Supongamos que tenemos un superconductor separado de dos partes por un aislante lo suficientemente estrecho como para que la función de onda superconductora a un lado de la barrera penetre algo al otro lado. Sean $\psi_1$ y $\psi_2$ las funciones de onda superconductoras a los dos lados de la barrera (1) y (2).

El túnel de pares del lado (2) al (1) aumenta la amplitud $\psi_1$ de la función de onda de los pares en el lado (1). Supongamos que el ritmo de crecimiento de $\psi_1$ es proporcional a $\psi_2$, amplitud de la funcion de onda de los pares en el lado (2). Podemos escribir el ritmo de cambio de $\psi_1$ de la forma $A \psi_2$ donde $A$ es una característica de la barrera y nos da información de la probabilidad de transferencia de pares del lado (2) al lado (1). En realidad lo qeu nos dice esta ecuación es qeu el ritmo de filtración en el lado superconductor (1) es proporcional a $\psi_2$.

Podemos escribir la ecuación de Schrödinger para el lado superconductor (1) teniendo en cuenta lo que acabamos de argumentar, como $\frac{-\hbar}{i} \frac{\partial \psi_1}{\partial t} = E_1 \psi_1 + A \psi_2$ donde $E_1$ es la energía del estado más bajo de energía del superconductor del lado (1).

Análogamente, podemos escribir para el superconductor del lado (2) $\frac{-\hbar}{i} \frac{\partial \psi_2}{\partial t} = E_2 \psi_2 + A \psi_1$

Escribiendo explícitamente la función de onda superconductora, tenemos $\psi = n_s^{1/2} e^{i \varphi}$ donde, como ya vimos, $n_s$ es la densidad de electrones superconductores.

Por lo tanto las ecuacioens anteriores las podemos escribir como $\frac{-\hbar}{i} \frac{1}{2 n_{s1}^{1/2}} \frac{\partial n_{s1}}{\partial t} + \hbar \frac{\partial \varphi_1}{\partial t} n_{s1}^{1/2} = E_1 n_{s1}^{1/2} + A n_{s2}^{1/2} e^{i (\varphi_2 - \varphi_1)}$

$\frac{-\hbar}{i} \frac{1}{2 n_{s2}^{1/2}} \frac{\partial n_{s2}}{\partial t} + \hbar \frac{\partial \varphi_2}{\partial t} n_{s2}^{1/2} = E_2 n_{s2}^{1/2} + A n_{s1}^{1/2} e^{i (\varphi_1 - \varphi_2)}$

Si ahora igualamos las partes reales y las partes imaginarias de estas ecuaciones tenemos

\begin{align*}
    \hbar \frac{\partial \varphi_1}{\partial t} &= A (\frac{n_{s2}}{n_{s1}})^{1/2} \cos(\varphi_2 - \varphi_1) + E_1 \\
    \hbar \frac{\partial \varphi_2}{\partial t} &= A (\frac{n_{s1}}{n_{s2}})^{1/2} \cos(\varphi_1 - \varphi_2) + E_2 \\
    \frac{\partial n_{s1}}{\partial t} &= \frac{-2 A}{\hbar} (n_{s1} n_{s2})^{1/2} \sin(\varphi_2 - \varphi_1) \\
    \frac{\partial n_{s2}}{\partial t} &= \frac{2 A}{\hbar} (n_{s1} n_{s2})^{1/2} \sin(\varphi_2 - \varphi_1)
\end{align*}

Si definimos $\delta = \varphi_2 - \varphi_1$ y $j_0 = \frac{2 A}{\hbar} (n_{s1} n_{s2})^{1/2}$ teniendo en cuenta, que el significado de la variación con el tiempo de la densidad de electrones superconductores, no es ni más ni menos que una corriente eléctrica a través de la barrera $j = j_0 \sin(\delta)$.

Como hemos hecho todo el cálculo para un mismo superconductor podemos simplificar $n_{s1} = n_{s2}$. En realidad todo lo anterior es igualmente válido si tuviésemos a cada lado de la barrera dos superconductores distintos.

Una vez completada la parte de cálculo podemos pasar ahora a considerar aspectos generales de estas uniones Josephson, que es el nombre con el que se las conoce. De momento existe una clara diferencia entre este efecto túnel y el considerado al principio. En este caso de efecto Josephson tenemos túnel de pares de Cooper, mientras que en el caso anterior el túnel era de electrones individuales. En el efecto Josephson son bastante estrechas del orden o menores que la longitud coherente (tamaño de los pares de Cooper), en realidad el aislante actúa como un mal superconductor, las funciones de onda de ambso lados se pueden solapar, existiendo una difrencia de fase a ambos lados de la barrera y como resultado de todo esto se establece una corriente continua a ravés de la barrera. En ralidad, se puede demostrar que el mínimo de enrgía se alcanza cuando las fases se igualan y por lo tanto no se tiene correitne a través de la barrera, pero basta con aplicar una correinte de una fuente externa, siempre menor que $I_0$, para que las fases dejen de ser iguales y si esta desigualdad no varía con el tiempo, se tiene a través de la barrera una correitne constante sin qeu se tenga caída de potencial. no hace falta colocar un óxido, un aislante, para que actúe de barrera se puede utilizar un estrechamiento en el supercondcutor, producido mediante una técnica conocida como fotolitografía, o cualquier otra técnica que nos produzca que una parte del supercondutor esté unida a otra mediante un estrangulamiento lo suficientemente estrecho como para ser una unión débil. También se peuden obtener este tipo de uniones débiles con contactos entre superconductores de tipo puntual, o bien separando dos supercondcutores por una capa delgada de un metal en estado normal. Como acabamos de mencionar el tamaño de los pares de Cooper es una indicación del orden de magnitud de esta unión débil.

Los órdenes de magnitud de los parámetros qeu intervienen en el efecto Josephson son, para una unión de $1mm^2$ de área $R = 1 \Omega$, $I_0 = 1 mA$ y $\Delta = 1meV$. Normalmente las densidades de corriente a través de las uniones son dle orden de un millón de veces menores qeu las densidades de corriente crítica superconductoa en un superconductor. Es decir, en general, en un superconductor se tiene que pasar de la densidad de correinte crítica para hacer desaparecer la superconductividad, pero si el superconductor tiene en algún punto una unión débil densidades de corriente un millón de veces menores hacen que se tenga caída de potencial en el paso de correitne por la unión.

Si hacemos pasar una corriente mayor que $I_0$ aparece una diferencia de potencial y teniendo en cuenta que al mismo tiempo que este efecto túnel (Josephson) de pares podemos tener el efecto túnel normal (Giaver) de electrones, el resultado se puede ver en la gráfica (FIGURA) donde se representa la curva característica I,V. Como sea la conexción en la realidad entre estos dos túneles depende de las características de l aunión y del circuito exterior.

Finalmente hay que considerar lo que ocurre si aplicamos a la unión Josephson una diferencia de potencial externa constante, V, o lo que es lo mismo, tenemos una intensidad pasando por la barrera superor a $I_0$. El cálculo es análogo al anterior, muy sencillo y nos conduce a que la intensidad a través de la barrera tiene la expresión $I = I_0 \sin( (\varphi_2 - \varphi_1) + \omega t) = I_o \sin(\delta + \omega t)$ donde $\omega = \frac{2 ev}{\hbar}$.

En realidad el punto de partida de esta deducción es la dependencia con el tiempo de la fase, una diferencia de potencial aplicada a l aunión lo que hace es variar en el tiempo la fase y el ritmo de variación d ela fase viene dado por la ecuación $\frac{d \delta}{d t} = \frac{E_2 - E_1}{\hbar} = \frac{qV}{\hbar} = \frac{2 eV}{\hbar}$.

Es decir, una diferencia de potencial constante aplicada a una barrera Josephson produce una corriente alterna de pares (una supercorreinte), por ejemplo una diferencia de potencial del orden de $1 \mu V$ da lugar a una corriente que oscila con una frecuencia de 484MHz. Además de esta corriente de pares tenemos la correitne debida al efecto túnel normla de los electrones individuales. Ahora podemos volver a la gráfica (FIGURA) donde tendremos que hasta que se alcanza el valor $I_0$ tenemos paso de correinte por la unión sin caida de potencial. una vez qeu $I > I_0$, entonces aparece un voltaje y nos situamos en un punto de la gráfica del efecto túnel de electrones normales (Giaver), tenemos una diferencia de potencial aplicada, luego existe túnel normal y además, esto no está represntado en la fëáfica, tenemos una supercorriente (corriente de pares de Cooper) que está oscilando.

Antes de terminar esta sección, unas palabaras acerca de este sorprendente descubrimiento. Como dijimos al principio el hecho de que una diferencia de fase de una entidad puramente cuántica peuda determinar un efecto macroscópico tan claro como es la aparición de correitne eléctrica, no fue descubierto experimentalmente. Fue calculado de un manera completa, incluso indicando la forma de hacer la comprobación experimental, por Brian D. Josephson a la edad de 22 años, cuando estaba empezando a trabajas bajo al direccióñ de Brian Pippard en su tesis doctoral en el Cavendish Laboratory de la Universidad de Cambridge. Pocos años después Josephson pasó a engrosar la larga lista de premios Nobel en física por descubrimientos relacionados con la superconductividad. Su descrubrimiento ha tenido numerosas aplicaciones en campos tan distintos como pueden ser desde astronomía a medicina.
