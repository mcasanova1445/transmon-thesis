\chapter{Conclusiones}

En el presente trabajo se estudiaron las bases de información cuántica, superconductividad, computación cuántica con transmones y tres algoritmos cuánticos. Se construyó un simulador de transmones acoplados a un mismo resonador y el set de instrucciones del procesador cuántico formado por estos transmones. Con este simulador se ejecutaron los tres algoritmos estudiados, los cuales son: El algoritmo de búsqueda de Grover, el algoritmo de factorización de Shor y el algoritmo de centralidad PageRank. Además, las simulaciones se realizaron para un sistema cerrado y para un sistema abierto markoviano.

Del algoritmo de Grover, se realizaron simulaciones del algoritmo con tres bases de datos distintas de dieciséis elementos y un estado marcado, una de dos estados marcados y una de cuatro estados marcados. Con el algoritmo de Shor se factorizaron los números quince y ocho. Luego, el algoritmo PageRank se aplicó a cuatro grafos, uno estrella, uno corona, uno árbol y uno aleatorio.

Debido a que el presente trabajo persiguió objetivos que en nuestra universidad no se dictan dentro del contenido programático de la carrera de Ingeniería Electrónica, se decidió hacer una presentación detallada de los conceptos y herramienta necesarias para la comprensión de la teoría de información cuántica, la computación cuántica superconductora y los algoritmos simulados. Hasta donde conocemos, no existe ningún otro trabajo que haya simulado estos algoritmos en un sistema abierto markoviano y es el primer trabajo de computación cuántica en un departamento de ingeniería venezolano.

En el presente trabajo, se desarrollaron las siguientes herramientas y se obtuvieron los siguientes resultados novedosos:

\begin{enumerate}
    \item Una compuerta controlada de fase CP que permita eliminar las fases en las compuertas de negación con dos o más qubits de control, como la de Toffoli.
    \item Un conjunto de instrucciones cuánticas basadas en las compuertas nativas de los transmones y un simulador del sistema físico.
    \item Un operador de multiplicación por 3 módulo 8 sin qubits de ancilla.
    \item La forma explícita del operador de difusión de las caminatas cuánticas de Szegedy para grafos de cuatro nodos, en función de rotaciones en Y controladas.
    \item El efecto de la relajación en los algoritmos de Grover, Shor y PageRank.
\end{enumerate}

En el algoritmo de Grover se ha visto que con relajación, sólo se puede realizar una iteración. En la segunda ya no se puede garantizar que el estado deseado el de probabilidad más alta y en general tiene probabilidad mayor al final de la primera que al final de la segunda. Así que la mejor alternativa en este algoritmo es su variante de un paso. Sin embargo, debido a que se deben procesar $\Omega(N \log(N))$ medidas, se pierde la ventaja de ejecutar una búsqueda cuántica.

El algoritmo de Shor no se pudo simular con relajación, debido a los largos tiempos que esto hubiese tomado. La ejecución del operador de multiplicación modular $MUL(7,15)$ tomó 72 horas y la del operador $MUL(3,8)$ tomó más de 120 horas. Estos operadores se deben aplicar 15 veces en total, así que el tiempo de ejecución de estas simulaciones hubiese sido de al menos 45 y 75 días, para factorizar los números 15 y 8, respectivamente.

En cuanto al algoritmo PageRank, cada iteración del algoritmo es tan larga, que exceden el tiempo de vida de los qubits y al final de cada iteración sólo se tiene el efecto de las últimas compuertas de esta, se pierde el efecto de las primeras. Como la relajación borra toda la información del inicio de la iteración, todas las iteraciones terminan siendo iguales y se miden los mismos valores al final de cada una.

Los resultados de las simulaciones del algoritmo de Grover y PageRank nos indican que para sistemas con tiempos de relajación del orden de $O(10^4 ns)$, los protocolos de corrección de errores son una necesidad. Este es un resultado significante, pues el record actual de tiempo de vida de un qubit superconductor es inferior a 0.1ms, está en el mismo orden de magnitud que el del sistema simulado.
