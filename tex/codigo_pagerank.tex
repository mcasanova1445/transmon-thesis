\chapter{Códigos de la simulación del algoritmo de PageRank}
\label{ch:pagerankcod}

\section{Wolfram Mathematica}

Los grafos que utilizaremos se definen de la siguiente manera:

\begin{enumerate}
    \item Grafo estrella

\begin{doublespace}
\noindent\(\pmb{g=\text{Graph}[\{1\text{-$>$}2,1\text{-$>$}3,1\text{-$>$}4,2\text{-$>$}1,3\text{-$>$}1,4\text{-$>$}1\}];}
\)
\end{doublespace}

    \item Grafo corona

\begin{doublespace}
\noindent\(\pmb{g=\text{Graph}[\{1\text{-$>$}2,2\text{-$>$}3,3\text{-$>$}1, 1\text{-$>$}3,3\text{-$>$}2,2\text{-$>$}1, 1\text{-$>$}4,2\text{-$>$}4,3\text{-$>$}4\}];}
\)
\end{doublespace}

    \item Grafo árbol

\begin{doublespace}
\noindent\(\pmb{g=\text{Graph}[\{1\text{-$>$}2,1\text{-$>$}3,2\text{-$>$}4\}];}\\
\)
\end{doublespace}

    \item Grafo aleatorio

\begin{doublespace}
\noindent\(\pmb{g=\text{Graph}[\{1\text{-$>$}2,1\text{-$>$}3,1\text{-$>$}4,2\text{-$>$}1,2\text{-$>$}4,3\text{-$>$}4,4\text{-$>$}3\}];}
\)
\end{doublespace}
\end{enumerate}

Ahora, con el grafo definido, la matriz de estocástica del grafo se calcula de la siguiente manera a partir de la matriz de adyacencia:

\begin{doublespace}
\noindent\(\pmb{\text{Em}=\text{AdjacencyMatrix}[g]; }\\
\pmb{\text{For}[j=1,j\text{$<$=}4,j\text{++},}\\
\pmb{\quad \text{OutDeg}=\text{Sum}[\text{Em}[[j,i]],\{i,1,4\}];}\\
\pmb{\quad \text{If}[\text{OutDeg}\text{!=}0, \text{Em}[[j]]=\text{Em}[[j]]/\text{OutDeg}, \text{Em}[[j]]=\{1/4,1/4,1/4,1/4\};]]}\\
\pmb{\text{Em}=\text{Transpose}[\text{Em}]; }\\
\)
\end{doublespace}

La matriz de Google con un $\alpha$ arbitrario es la siguiente:

\begin{doublespace}
\noindent\(\pmb{G=\alpha  \text{Em} + \frac{(1-\alpha )}{\text{Dimensions}[\text{Em}][[1]]} \text{Table}[1,\{\text{Dimensions}[\text{Em}][[1]]\},\{\text{Dimensions}[\text{Em}][[1]]\}]; }\)
\end{doublespace}

El PageRank clásico del grafo $g$ se puede calcular de las siguientes dos maneras:

\begin{enumerate}
    \item Cálculo manual multiplicando la matriz de Google $N$ veces a un vector de PageRank inicial. Este método permite conocer el PageRank en cada paso de la iteración, así que se podría utilizar para graficar la evolución de los valores.

\begin{doublespace}
\noindent\(\pmb{\text{Ivl1}=\{1\}; \text{(*PageRank inicial de cada nodo*)}}\\
\pmb{\text{Ivl2}=\{0\};}\\
\pmb{\text{Ivl3}=\{0\};}\\
\pmb{\text{Ivl4}=\{0\};}\\
\pmb{\text{Iv}=\{\{1\},\{0\},\{0\},\{0\}\}; \text{(*Vector de PageRank inicial*)}}\\
\pmb{\text{For}[i=1,i\leq 30,i\text{++},}\\
\pmb{\text{Iv}=(G\text{/.}\{\alpha \to 0.85\}).\text{Iv}; \text{(*Iterar con la matriz de Google*)}}\\
\pmb{\text{(*}\text{Iv}=\text{Iv}/\text{Norm}[\text{Iv}];\text{*)}}\\
\pmb{\text{AppendTo}[\text{Ivl1},\text{Iv}[[1,1]]];}\\
\pmb{\text{AppendTo}[\text{Ivl2},\text{Iv}[[2,1]]];}\\
\pmb{\text{AppendTo}[\text{Ivl3},\text{Iv}[[3,1]]];}\\
\pmb{\text{AppendTo}[\text{Ivl4},\text{Iv}[[4,1]]];]}\)
\end{doublespace}

    \item Cálculo con la función de PageRank incluida en Wolfram Mathematica. Esta función recibe como primer argumento el grafo y como segundo argumento el valor de $\alpha$.

\begin{doublespace}
\noindent\(\pmb{\text{PageRankCentrality}[g,0.85]}\)
\end{doublespace}
\end{enumerate}

Ahora que se han definido los parámetros del grafo, procedemos a definir la parte cuántica del algoritmo. Primero, creamos una función que retorne los kets de los estados número.

\begin{doublespace}
\noindent\(\pmb{\text{ket}[\text{k$\_$},\text{n$\_$}]\text{:=}\text{Table}[\{i==k\},\{i,0,n-1\}]\text{/.}\{\text{True}\to 1,\text{False}\to 0\}; }\)
\end{doublespace}

Utilizaremos la definición del operador de difusión dada por Paparo, así que basta con definir las proyecciones entre las particiones del grafo bipartito en un sólo sentido.

\begin{doublespace}
\noindent\(\pmb{\text{$\phi $x}_{\text{j$\_$}}\text{:=}\text{KroneckerProduct}\left[\text{ket}[j-1,4],\text{Sum}\left[\sqrt{G[[k,j]]} \text{ket}[k-1,4],\{k,1,4\}\right]\right]; }\\
\pmb{\text{$\phi $x0}=\left.\text{Sum}\left[\text{$\phi $x}_j,\{j,1,4\}\right]\right/2;}\\
\pmb{\text{$\Pi $x}=\text{Sum}\left[\text{$\phi $x}_j.\text{$\phi $x}_j\dagger,\{j,1,4\}\right]; \text{(*}\text{Proyector} A \to  B\text{*)}}\\
\pmb{S=\text{Sum}[\text{KroneckerProduct}[\text{ket}[j,4],}\\
\pmb{\qquad\text{ket}[k,4]].\text{KroneckerProduct}[\text{ket}[k,4],\text{ket}[j,4]]\dagger,\{j,0,3\},\{k,0,3\}];}\\
\pmb{\text{Ux}=S.(2\text{$\Pi $x}-\text{IdentityMatrix}[16]); \text{(*}\text{Operador} \text{de} \text{difusi{\' o}n} 1/2\text{*)}}\\
\pmb{W=\text{MatrixPower}[\text{Ux},2]; \text{(*Operador de difusi{\' o}n 1*)}}\\
\)
\end{doublespace}

Finalmente, el PageRank cuántico se calcula de la siguiente manera. Hemos tomado el valor estándar de $\alpha = 0.85$.

\begin{doublespace}
\noindent\(\pmb{a=0.85;}\\
\pmb{\text{qIvl1}=}\\
\pmb{\text{Table}[((\text{$\phi $x0}\dagger\text{/.}\{\alpha \to a\}).\text{MatrixPower}[(W\dagger\text{/.}\{\alpha \to a\}),m].}\\
\pmb{\qquad\text{KroneckerProduct}[\text{IdentityMatrix}[4],\text{ket}[0,4].\text{ket}[0,4]\dagger].}\\
\pmb{\qquad\text{MatrixPower}[(W\text{/.}\{\alpha
\to a\}),m].(\text{$\phi $x0}\text{/.}\{\alpha \to a\}))[[1,1]],\{m,1,30\}];}\\
\pmb{\text{qIvl2}=}\\
\pmb{\text{Table}[((\text{$\phi $x0}\dagger\text{/.}\{\alpha \to a\}).\text{MatrixPower}[(W\dagger\text{/.}\{\alpha \to a\}),m].}\\
\pmb{\qquad\text{KroneckerProduct}[\text{IdentityMatrix}[4],\text{ket}[1,4].\text{ket}[1,4]\dagger].}\\
\pmb{\qquad\text{MatrixPower}[(W\text{/.}\{\alpha
\to a\}),m].(\text{$\phi $x0}\text{/.}\{\alpha \to a\}))[[1,1]],\{m,1,30\}];}\\
\pmb{\text{qIvl3}=}\\
\pmb{\text{Table}[((\text{$\phi $x0}\dagger\text{/.}\{\alpha \to a\}).\text{MatrixPower}[(W\dagger\text{/.}\{\alpha \to a\}),m].}\\
\pmb{\qquad\text{KroneckerProduct}[\text{IdentityMatrix}[4],\text{ket}[2,4].\text{ket}[2,4]\dagger].}\\
\pmb{\qquad\text{MatrixPower}[(W\text{/.}\{\alpha
\to a\}),m].(\text{$\phi $x0}\text{/.}\{\alpha \to a\}))[[1,1]],\{m,1,30\}];}\\
\pmb{\text{qIvl4}=}\\
\pmb{\text{Table}[((\text{$\phi $x0}\dagger\text{/.}\{\alpha \to a\}).\text{MatrixPower}[(W\dagger\text{/.}\{\alpha \to a\}),m].}\\
\pmb{\qquad\text{KroneckerProduct}[\text{IdentityMatrix}[4],\text{ket}[3,4].\text{ket}[3,4]\dagger].}\\
\pmb{\qquad\text{MatrixPower}[(W\text{/.}\{\alpha
\to a\}),m].(\text{$\phi $x0}\text{/.}\{\alpha \to a\}))[[1,1]],\{m,1,30\}];}\)
\end{doublespace}


\section{Python}

Primero, se importan todos los módulos necesarios. Entre ellos, Numpy para variables y funciones numéricas, tgates para las compuertas de transmones.

\begin{verbatim}
import matplotlib.pyplot as plt
import matplotlib as mpl
import numpy as np
from scipy.stats import norm
from qutip import *
import tgates
import time
\end{verbatim}

Las siguientes dos funciones son los operadores de permutación hacia la derecha y hacia la izquierda, controlados. Los argumentos c1 y c2 corresponden a los valores de los qubits del primer y segundo qubit del primer registro, respectivamente, con los que se controla el operador. Si son 0, el operador se activa con el ket $\ket{0}$, si son 1, se activa con el ket $\ket{1}$ y si son -1 se tiene el caso no controlado.

\begin{verbatim}
def R2(psi0, c1, c2):
    if c1 == -1 and c2 == -1:
        res = tgates.CNOT(psi0, 3, 2)
        res = tgates.X(res.states[-1], 3)

    elif c1 == -1:
        psi1 = psi0

        if c2 == 0:
            res = tgates.X(psi1, 1)
            psi1 = res.states[-1]

        res = tgates.CCNOT(psi1, 1, 3, 2)
        res = tgates.CNOT(res.states[-1], 1, 3)
        psi1 = res.states[-1]

        if c2 == 0:
            res = tgates.X(psi1, 1)
            psi1 = res.states[-1]

    elif c2 == -1:
        psi1 = psi0

        if c1 == 0:
            res = tgates.X(psi1, 0)
            psi1 = res.states[-1]

        res = tgates.CCNOT(psi1, 0, 3, 2)
        res = tgates.CNOT(res.states[-1], 0, 3)
        psi1 = res.states[-1]

        if c1 == 0:
            res = tgates.X(psi1, 0)
            psi1 = res.states[-1]

    else:
        psi1 = psi0

        if c1 == 0:
            res = tgates.X(psi1, 0)
            psi1 = res.states[-1]

        if c2 == 0:
            res = tgates.X(psi1, 1)
            psi1 = res.states[-1]

        res = tgates.CCCNOT(psi1, 0, 1, 3, 2)
        res = tgates.CCNOT(res.states[-1], 0, 1, 3)
        psi1 = res.states[-1]

        if c1 == 0:
            res = tgates.X(psi1, 0)
            psi1 = res.states[-1]

        if c2 == 0:
            res = tgates.X(psi1, 1)
            psi1 = res.states[-1]

    return res

def L2(psi0, c1, c2):
    if c1 == -1 and c2 == -1:
        res = tgates.X(psi0, 3)
        res = tgates.CNOT(res.states[-1], 3, 2)

    elif c1 == -1:
        psi1 = psi0

        if c2 == 0:
            res = tgates.X(psi1, 1)
            psi1 = res.states[-1]

        res = tgates.CNOT(psi1, 1, 3)
        res = tgates.CCNOT(res.states[-1], 1, 3, 2)
        psi1 = res.states[-1]

        if c2 == 0:
            res = tgates.X(psi1, 1)
            psi1 = res.states[-1]

    elif c2 == -1:
        psi1 = psi0

        if c1 == 0:
            res = tgates.X(psi1, 0)
            psi1 = res.states[-1]

        res = tgates.CNOT(psi1, 0, 3)
        res = tgates.CCNOT(res.states[-1], 0, 3, 2)
        psi1 = res.states[-1]

        if c1 == 0:
            res = tgates.X(psi1, 0)
            psi1 = res.states[-1]

    else:
        psi1 = psi0

        if c1 == 0:
            res = tgates.X(psi1, 0)
            psi1 = res.states[-1]

        if c2 == 0:
            res = tgates.X(psi1, 1)
            psi1 = res.states[-1]

        res = tgates.CCNOT(psi1, 0, 1, 3)
        res = tgates.CCCNOT(res.states[-1], 0, 1, 3, 2)
        psi1 = res.states[-1]

        if c1 == 0:
            res = tgates.X(psi1, 0)
            psi1 = res.states[-1]

        if c2 == 0:
            res = tgates.X(psi1, 1)
            psi1 = res.states[-1]

    return res
\end{verbatim}

Ahora definimos el operador de reflexión alrededor del estado de referencia de la base computacional. El estado elegido ha sido $\ket{0}$.

\begin{verbatim}
def Dd(psi0):
    return tgates.CP(psi0, 2, 3, np.pi, b=0b00)
\end{verbatim}

El operador de SWAP entre los registros se define de la siguiente manera. Esto es, la compuerta SWAP aplicada entre el i-ésimo qubit de cada uno de los registros.

\begin{verbatim}
def reg_SWAP(psi0):
    res = tgates.SWAP(psi0, 1, 3)
    return tgates.SWAP(res.states[-1], 0, 2)
\end{verbatim}

La parte siguiente es dependiente del grafo. Esto es, los operadores $K_i$ y $T_i$ y la sucesión de estos que generará el operador de difusión de la caminata cuántica de Szegedy en cada uno de los grafos.

\subsection{Grafo estrella}

Los operadores $K_i$ y $T_i$ del grafo estrella se definen de la siguiente manera:

\begin{verbatim}
def Kb1(psi0):
    thetay00 = 1.85806
    thetay10 = 2.48274
    thetay11 = 1.5708

    res = tgates.X(psi0, 0)
    res = tgates.X(res.states[-1], 1)

    res = tgates.X(res.states[-1], 3)
    res = tgates.CCCRy(res.states[-1], 0, 1, 3, 2, thetay00)
    res = tgates.X(res.states[-1], 3)

    res = tgates.X(res.states[-1], 2)
    res = tgates.CCCRy(res.states[-1], 0, 1, 2, 3, thetay10)
    res = tgates.X(res.states[-1], 2)

    res = tgates.CCCRy(res.states[-1], 0, 1, 2, 3, thetay11)

    res = tgates.X(res.states[-1], 1)
    return tgates.X(res.states[-1], 0)

def Kb1d(psi0):
    thetay00 = 1.85806
    thetay10 = 2.48274
    thetay11 = 1.5708

    res = tgates.X(psi0, 0)
    res = tgates.X(res.states[-1], 1)

    res = tgates.CCCRy(res.states[-1], 0, 1, 2, 3, -thetay11)

    res = tgates.X(res.states[-1], 2)
    res = tgates.CCCRy(res.states[-1], 0, 1, 2, 3, -thetay10)
    res = tgates.X(res.states[-1], 2)

    res = tgates.X(res.states[-1], 3)
    res = tgates.CCCRy(res.states[-1], 0, 1, 3, 2, -thetay00)
    res = tgates.X(res.states[-1], 3)

    res = tgates.X(res.states[-1], 1)
    return tgates.X(res.states[-1], 0)

def Kb2(psi0):
    thetay00 = 0.554811
    thetay10 = 0.405465
    thetay11 = 1.5708

    res = tgates.X(psi0, 3)
    res = tgates.CRy(res.states[-1], 3, 2, thetay00)
    res = tgates.X(res.states[-1], 3)

    res = tgates.X(res.states[-1], 2)
    res = tgates.CRy(res.states[-1], 2, 3, thetay10)
    res = tgates.X(res.states[-1], 2)

    res = tgates.CRy(res.states[-1], 2, 3, thetay11)


    res = tgates.X(res.states[-1], 0)
    res = tgates.X(res.states[-1], 1)

    res = tgates.CCCRy(res.states[-1], 0, 1, 2, 3, -thetay11)

    res = tgates.X(res.states[-1], 2)
    res = tgates.CCCRy(res.states[-1], 0, 1, 2, 3, -thetay10)
    res = tgates.X(res.states[-1], 2)

    res = tgates.X(res.states[-1], 3)
    res = tgates.CCCRy(res.states[-1], 0, 1, 3, 2, -thetay00)
    res = tgates.X(res.states[-1], 3)

    res = tgates.X(res.states[-1], 1)
    return tgates.X(res.states[-1], 0)

def Kb2d(psi0):
    thetay00 = 0.554811
    thetay10 = 0.405465
    thetay11 = 1.5708

    res = tgates.X(psi0, 0)
    res = tgates.X(res.states[-1], 1)

    res = tgates.X(res.states[-1], 3)
    res = tgates.CCCRy(res.states[-1], 0, 1, 3, 2, thetay00)
    res = tgates.X(res.states[-1], 3)

    res = tgates.X(res.states[-1], 2)
    res = tgates.CCCRy(res.states[-1], 0, 1, 2, 3, thetay10)
    res = tgates.X(res.states[-1], 2)

    res = tgates.CCCRy(res.states[-1], 0, 1, 2, 3, thetay11)

    res = tgates.X(res.states[-1], 1)
    res = tgates.X(res.states[-1], 0)


    res = tgates.CRy(res.states[-1], 2, 3, -thetay11)

    res = tgates.X(res.states[-1], 2)
    res = tgates.CRy(res.states[-1], 2, 3, -thetay10)
    res = tgates.X(res.states[-1], 2)

    res = tgates.X(res.states[-1], 3)
    res = tgates.CRy(res.states[-1], 3, 2, -thetay00)
    return tgates.X(res.states[-1], 3)
\end{verbatim}

Ahora, creamos el estado inicial, a partir del estado fiducial.

\begin{verbatim}
psi0 = tensor(basis(2,0), basis(2,0), basis(2,0), basis(2,0))

res = tgates.H(psi0,0)
res = tgates.H(res.states[-1],1)
res = Kb1(res.states[-1])
res = Kb2(res.states[-1])
\end{verbatim}

Finalmente, se realizan la caminata de Szegedy y se guarda el estado resultante en cada paso.

\begin{verbatim}
for i in range(60):
    
    res = Kb2d(res.states[-1])
    res = Kb1d(res.states[-1])
    res = Dd(res.states[-1])
    res = Kb1(res.states[-1])
    res = Kb2(res.states[-1])
    res = reg_SWAP(res.states[-1])
    
    qsave(res, 'itj_{}'.format(i+1))
\end{verbatim}


\subsection{Grafo corona}

Los operadores $K_i$ y $T_i$ del grafo corona se definen de la siguiente manera:

\begin{verbatim}
def T1(psi0):
    res = L2(psi0, 0, 1)
    res = L2(res.states[-1], 1, 0)
    return L2(res.states[-1], 1, 0)

def T1d(psi0):
    res = R2(psi0, 1, 0)
    res = R2(res.states[-1], 1, 0)
    return R2(res.states[-1], 0, 1)

def Kb1(psi0):
    thetay00 = 1.85806
    thetay10 = 2.48274
    thetay11 = 1.5708

    res = tgates.X(psi0, 3)
    res = tgates.CRy(res.states[-1], 3, 2, thetay00)
    res = tgates.X(res.states[-1], 3)
    res = tgates.X(res.states[-1], 2)
    res = tgates.CRy(res.states[-1], 2, 3, thetay10)
    res = tgates.X(res.states[-1], 2)
    res = tgates.CRy(res.states[-1], 2, 3, thetay11)

    res = tgates.CCCRy(res.states[-1], 0, 1, 2, 3, -thetay11)
    res = tgates.X(res.states[-1], 2)
    res = tgates.CCCRy(res.states[-1], 0, 1, 2, 3, -thetay10)
    res = tgates.X(res.states[-1], 2)
    res = tgates.X(res.states[-1], 3)
    res = tgates.CCCRy(res.states[-1], 0, 1, 3, 2, -thetay00)
    return tgates.X(res.states[-1], 3)

def Kb1d(psi0):
    thetay00 = 1.85806
    thetay10 = 2.48274
    thetay11 = 1.5708

    res = tgates.X(psi0, 3)
    res = tgates.CCCRy(res.states[-1], 0, 1, 3, 2, thetay00)
    res = tgates.X(res.states[-1], 3)
    res = tgates.X(res.states[-1], 2)
    res = tgates.CCCRy(res.states[-1], 0, 1, 2, 3, thetay10)
    res = tgates.X(res.states[-1], 2)
    res = tgates.CCCRy(res.states[-1], 0, 1, 2, 3, thetay11)

    res = tgates.CRy(res.states[-1], 2, 3, -thetay11)
    res = tgates.X(res.states[-1], 2)
    res = tgates.CRy(res.states[-1], 2, 3, -thetay10)
    res = tgates.X(res.states[-1], 2)
    res = tgates.X(res.states[-1], 3)
    res = tgates.CRy(res.states[-1], 3, 2, -thetay00)
    return tgates.X(res.states[-1], 3)

def Kb2(psi0):
    res = tgates.CCRy(psi0, 0, 1, 2, np.pi/2)
    return tgates.CCRy(res.states[-1], 0, 1, 3, np.pi/2)

def Kb2d(psi0):
    res = tgates.CCRy(psi0, 0, 1, 3, -np.pi/2)
    return tgates.CCRy(res.states[-1], 0, 1, 2, -np.pi/2)
\end{verbatim}

Ahora, creamos el estado inicial, a partir del estado fiducial.

\begin{verbatim}
psi0 = tensor(basis(2,0), basis(2,0), basis(2,0), basis(2,0))

res = tgates.H(psi0,0)
res = tgates.H(res.states[-1],1)
res = Kb1(res.states[-1])
res = Kb2(res.states[-1])
res = T1d(res.states[-1])
\end{verbatim}

Finalmente, se realizan la caminata de Szegedy y se guarda el estado resultante en cada paso.

\begin{verbatim}
for i in range(60):
    
    res = T1(res.states[-1])
    res = Kb2d(res.states[-1])
    res = Kb1d(res.states[-1])
    res = Dd(res.states[-1])
    res = Kb1(res.states[-1])
    res = Kb2(res.states[-1])
    res = T1d(res.states[-1])
    res = reg_SWAP(res.states[-1])
    
    qsave(res, 'itj_{}'.format(i+1))

\end{verbatim}

\subsection{Grafo árbol}

Los operadores $K_i$ y $T_i$ del grafo árbol se definen de la siguiente manera:

\begin{verbatim}
def Kb1(psi0):
    thetay00 = 1.5708
    thetay10 = 2.58678
    thetay11 = 0.554811

    res = tgates.X(psi0, 0)
    res = tgates.X(res.states[-1], 1)

    res = tgates.X(res.states[-1], 3)
    res = tgates.CCCRy(res.states[-1], 0, 1, 3, 2, thetay00)
    res = tgates.X(res.states[-1], 3)

    res = tgates.X(res.states[-1], 2)
    res = tgates.CCCRy(res.states[-1], 0, 1, 2, 3, thetay10)
    res = tgates.X(res.states[-1], 2)

    res = tgates.CCCRy(res.states[-1], 0, 1, 2, 3, thetay11)

    res = tgates.X(res.states[-1], 1)
    return tgates.X(res.states[-1], 0)

def Kb1d(psi0):
    thetay00 = 1.5708
    thetay10 = 2.58678
    thetay11 = 0.554811

    res = tgates.X(psi0, 0)
    res = tgates.X(res.states[-1], 1)

    res = tgates.CCCRy(res.states[-1], 0, 1, 2, 3, -thetay11)

    res = tgates.X(res.states[-1], 2)
    res = tgates.CCCRy(res.states[-1], 0, 1, 2, 3, -thetay10)
    res = tgates.X(res.states[-1], 2)

    res = tgates.X(res.states[-1], 3)
    res = tgates.CCCRy(res.states[-1], 0, 1, 3, 2, -thetay00)
    res = tgates.X(res.states[-1], 3)

    res = tgates.X(res.states[-1], 1)
    return tgates.X(res.states[-1], 0)

def Kb2(psi0):
    thetay00 = 2.58678
    thetay10 = 1.5708
    thetay11 = 2.73613

    res = tgates.X(psi0, 0)

    res = tgates.X(res.states[-1], 3)
    res = tgates.CCCRy(res.states[-1], 0, 1, 3, 2, thetay00)
    res = tgates.X(res.states[-1], 3)

    res = tgates.X(res.states[-1], 2)
    res = tgates.CCCRy(res.states[-1], 0, 1, 2, 3, thetay10)
    res = tgates.X(res.states[-1], 2)

    res = tgates.CCCRy(res.states[-1], 0, 1, 2, 3, thetay11)

    return tgates.X(res.states[-1], 0)

def Kb2d(psi0):
    thetay00 = 2.58678
    thetay10 = 1.5708
    thetay11 = 2.73613

    res = tgates.X(psi0, 0)

    res = tgates.CCCRy(res.states[-1], 0, 1, 2, 3, -thetay11)

    res = tgates.X(res.states[-1], 2)
    res = tgates.CCCRy(res.states[-1], 0, 1, 2, 3, -thetay10)
    res = tgates.X(res.states[-1], 2)

    res = tgates.X(res.states[-1], 3)
    res = tgates.CCCRy(res.states[-1], 0, 1, 3, 2, -thetay00)
    res = tgates.X(res.states[-1], 3)

    return tgates.X(res.states[-1], 0)

def Kb3(psi0):
    res = tgates.CRy(psi0, 0, 2, np.pi/2)
    res = tgates.CRy(res.states[-1], 0, 3, np.pi/2)
    return res

def Kb3d(psi0):
    res = tgates.CRy(psi0, 0, 3, -np.pi/2)
    res = tgates.CRy(res.states[-1], 0, 2, -np.pi/2)
    return res
\end{verbatim}

Ahora, creamos el estado inicial, a partir del estado fiducial.

\begin{verbatim}
psi0 = tensor(basis(2,0), basis(2,0), basis(2,0), basis(2,0))

res = tgates.H(psi0,0)
res = tgates.H(res.states[-1],1)
res = Kb1(res.states[-1])
res = Kb2(res.states[-1])
res = Kb3(res.states[-1])
\end{verbatim}

Finalmente, se realizan la caminata de Szegedy y se guarda el estado resultante en cada paso.

\begin{verbatim}
for i in range(60):
    
    res = Kb3d(res.states[-1])
    res = Kb2d(res.states[-1])
    res = Kb1d(res.states[-1])
    res = Dd(res.states[-1])
    res = Kb1(res.states[-1])
    res = Kb2(res.states[-1])
    res = Kb3(res.states[-1])
    res = reg_SWAP(res.states[-1])
    
    qsave(res, 'itj_{}'.format(i+1))

\end{verbatim}

\subsection{Grafo aleatorio}

Los operadores $K_i$ y $T_i$ del grafo aleatorio se definen de la siguiente manera:

\begin{verbatim}
def T3(psi0):
    return R2(psi0, 1, 1)

def T3d(psi0):
    return L2(psi0, 1, 1)

def Kb1(psi0):
    thetay00 = 1.85806
    thetay10 = 2.48274
    thetay11 = 1.5708

    res = tgates.X(psi0, 0)
    res = tgates.X(res.states[-1], 1)

    res = tgates.X(res.states[-1], 3)
    res = tgates.CCCRy(res.states[-1], 0, 1, 3, 2, thetay00)
    res = tgates.X(res.states[-1], 3)

    res = tgates.X(res.states[-1], 2)
    res = tgates.CCCRy(res.states[-1], 0, 1, 2, 3, thetay10)
    res = tgates.X(res.states[-1], 2)

    res = tgates.CCCRy(res.states[-1], 0, 1, 2, 3, thetay11)

    res = tgates.X(res.states[-1], 1)
    return tgates.X(res.states[-1], 0)

def Kb1d(psi0):
    thetay00 = 1.85806
    thetay10 = 2.48274
    thetay11 = 1.5708

    res = tgates.X(psi0, 0)
    res = tgates.X(res.states[-1], 1)

    res = tgates.CCCRy(res.states[-1], 0, 1, 2, 3, -thetay11)

    res = tgates.X(res.states[-1], 2)
    res = tgates.CCCRy(res.states[-1], 0, 1, 2, 3, -thetay10)
    res = tgates.X(res.states[-1], 2)

    res = tgates.X(res.states[-1], 3)
    res = tgates.CCCRy(res.states[-1], 0, 1, 3, 2, -thetay00)
    res = tgates.X(res.states[-1], 3)

    res = tgates.X(res.states[-1], 1)
    return tgates.X(res.states[-1], 0)

def Kb2(psi0):
    thetay00 = 1.5708
    thetay10 = 0.554811
    thetay11 = 2.58678

    res = tgates.X(psi0, 0)

    res = tgates.X(res.states[-1], 3)
    res = tgates.CCCRy(res.states[-1], 0, 1, 3, 2, thetay00)
    res = tgates.X(res.states[-1], 3)

    res = tgates.X(res.states[-1], 2)
    res = tgates.CCCRy(res.states[-1], 0, 1, 2, 3, thetay10)
    res = tgates.X(res.states[-1], 2)

    res = tgates.CCCRy(res.states[-1], 0, 1, 2, 3, thetay11)

    return tgates.X(res.states[-1], 0)

def Kb2d(psi0):
    thetay00 = 1.5708
    thetay10 = 0.554811
    thetay11 = 2.58678

    res = tgates.X(psi0, 0)

    res = tgates.CCCRy(res.states[-1], 0, 1, 2, 3, -thetay11)

    res = tgates.X(res.states[-1], 2)
    res = tgates.CCCRy(res.states[-1], 0, 1, 2, 3, -thetay10)
    res = tgates.X(res.states[-1], 2)

    res = tgates.X(res.states[-1], 3)
    res = tgates.CCCRy(res.states[-1], 0, 1, 3, 2, -thetay00)
    res = tgates.X(res.states[-1], 3)

    return tgates.X(res.states[-1], 0)

def Kb3(psi0):
    thetay00 = 2.58678
    thetay10 = 1.5708
    thetay11 = 2.73613

    res = tgates.X(psi0, 3)
    res = tgates.CCRy(res.states[-1], 0, 3, 2, thetay00)
    res = tgates.X(res.states[-1], 3)

    res = tgates.X(res.states[-1], 2)
    res = tgates.CCRy(res.states[-1], 0, 2, 3, thetay10)
    res = tgates.X(res.states[-1], 2)

    res = tgates.CCRy(res.states[-1], 0, 2, 3, thetay11)

    return res

def Kb3d(psi0):
    thetay00 = 2.58678
    thetay10 = 1.5708
    thetay11 = 2.73613

    res = tgates.CCRy(psi0, 0, 2, 3, -thetay11)

    res = tgates.X(res.states[-1], 2)
    res = tgates.CCRy(res.states[-1], 0, 2, 3, -thetay10)
    res = tgates.X(res.states[-1], 2)

    res = tgates.X(res.states[-1], 3)
    res = tgates.CCRy(res.states[-1], 0, 3, 2, -thetay00)
    res = tgates.X(res.states[-1], 3)

    return res
\end{verbatim}

Ahora, creamos el estado inicial, a partir del estado fiducial.

\begin{verbatim}
psi0 = tensor(basis(2,0), basis(2,0), basis(2,0), basis(2,0))

res = tgates.H(psi0,0)
res = tgates.H(res.states[-1],1)
res = Kb1(res.states[-1])
res = Kb2(res.states[-1])
res = Kb3(res.states[-1])
res = T3d(res.states[-1])
\end{verbatim}

Finalmente, se realizan la caminata de Szegedy y se guarda el estado resultante en cada paso.

\begin{verbatim}
for i in range(60):
    
    res = T3(res.states[-1])
    res = Kb3d(res.states[-1])
    res = Kb2d(res.states[-1])
    res = Kb1d(res.states[-1])
    res = Dd(res.states[-1])
    res = Kb1(res.states[-1])
    res = Kb2(res.states[-1])
    res = Kb3(res.states[-1])
    res = T3d(res.states[-1])
    res = reg_SWAP(res.states[-1])
    
    qsave(res, 'itj_{}'.format(i+1))
\end{verbatim}


