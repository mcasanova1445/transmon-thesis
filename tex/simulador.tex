\chapter{El simulador}

El simulador se construyó utilizando la librería Qutip 4.2 de Python 3.6. Esta es una librería que incluye varias herramientas para realizar simulaciones de sistemas mecánico cuánticos, entre ellas, un solucionador de ecuaciones maestras. El funcionamiento básico del simulador desarrollado es el siguiente:

\begin{enumerate}
    \item Leer estado inicial
    \item Construir Hamiltoniano del sistema
    \item Introducir Hamiltoniano y estado inicial en el solucionador de ecuaciones maestras.
    \item Retornar solución
\end{enumerate}

De esta manera se simulan las compuertas naturales de los transmones. Luego, a partir de estas se construyen todas las demás compuertas que se necesitaran para construir un set de instrucciones cuántico con el cual poder ejecutar los algoritmos de Grover, Shor y PageRank.

Se simularon dos sistemas distintos, uno de cuatro qubits y otro de ocho qubits. El diseño original era el de cuatro qubits, con él se realizaron las simulaciones del algoritmo de Grover y del PageRank. Sin embargo, el algoritmo de Shor requiere de al menos ocho qubits para factorizar el número compuesto impar más pequeño: el número 15. Posteriormente también se realizo una generalización del simulador para poder trabajar con sistemas de n qubits.

El tipo de acoplamiento entre los qubits elegido es el acoplamiento de tipo bus. De esta manera trabajamos con un único resonador, el cual se puede tracear. Esto reduce significativamente la dimensión del sistema a simular y nos permite tener más qubits. Además, de esta forma, la interación es más directa y basta con la compuerta iSWAP para construir cualquier otra compuerta multiqubits, el cual no sería el caso con qubits acoplados a distintos resonadores, pues se necesitarían compuertas de interacción entre resonadores.

\section{Parámetros de los sistemas simulados}

Se han elegido parámetros típicos de los sistemas de qubits\cite{blais}.

\begin{enumerate}
    \item Frecuencias de resonancia:
        \begin{enumerate}
            \item Resonador: 10GHz
            \item Qubit 0: 5GHz
            \item Qubit 1: 6GHz
            \item Qubit 2: 7GHz
            \item Qubit 3: 8GHz
            \item *Qubit 4: 11GHz
            \item *Qubit 5: 12GHz
            \item *Qubit 6: 13GHz
            \item *Qubit 7: 14GHz
        \end{enumerate}
    \item Constante de acoplamiento: Todas iguales a $0.1  2 \pi$ (Unidades?)
    \item Tasas de decaimiento: Todas iguales a 5e-6 (Unidades?)
    \item Frecuencia de resonancia para iSWAP: 9GHz
\end{enumerate}

*Sólo aplica para el caso del sistema de 8 qubits

