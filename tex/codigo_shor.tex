\chapter{Códigos de la simulación del algoritmo de Shor}
\label{ch:shorcod}

\section{Wolfram Mathematica}

Primero definimos el número a factorizar y número que se utilizará para el operador de multiplicación modular. También verificamos que estos dos números sean coprimos.

\begin{doublespace}
\noindent\(
\pmb{M=15; (*13;*)}\\
\pmb{x=\text{RandomInteger}[\{2,M-2\}]}\\
\pmb{\text{GCD}[x,M]==1}\)
\end{doublespace}

Ahora se definen las matrices necesarias para el algoritmo, como los kets de la base computacional, el operador de multiplicación, la compuerta de fase, entre otras.

\begin{doublespace}
\noindent\(
\pmb{\text{ket}_0=\{\{1\},\{0\}\};}\\
\pmb{\text{ket}_1=\{\{0\},\{1\}\};}\\
\pmb{\text{Id}=\text{IdentityMatrix}[2];}\\
\pmb{X=\text{PauliMatrix}[1];}\\
\pmb{Y=\text{PauliMatrix}[2];}\\
\pmb{Z=\text{PauliMatrix}[3];}\\
\pmb{H=\text{HadamardMatrix}[2];}\\
\pmb{P_0=\text{ket}_0.\text{ConjugateTranspose}\left[\text{ket}_0\right];}\\
\pmb{P_1=\text{ket}_1.\text{ConjugateTranspose}\left[\text{ket}_1\right];}\\
\pmb{R[\phi \_]\text{:=}P_0 + e^{i \phi } P_1;}\\
\pmb{\text{CNOT}=\text{KroneckerProduct}\left[P_0,\text{Id}\right]+\text{KroneckerProduct}\left[P_1,X\right];}\\
\pmb{\text{CR}[\phi \_]\text{:=}\text{KroneckerProduct}\left[P_0,\text{Id}\right]+\text{KroneckerProduct}\left[P_1,R[\phi ]\right];}\\
\pmb{\text{CR13}[\phi \_]\text{:=}\text{KroneckerProduct}\left[P_0,\text{Id},\text{Id}\right]+\text{KroneckerProduct}\left[P_1,\text{Id},R[\phi ]\right];}\\
\pmb{\text{ket}[\text{n$\_$}]\text{:=}\text{Table}[\{n==i-1\},\{i,1,16\}]\text{/.}\{\text{True}\to 1,\text{False}\to 0\};}\\
\pmb{\text{Urp}[\text{x$\_$},\text{N$\_$}]\text{:=}\text{Sum}[\text{ket}[\text{Mod}[x i,N]].\text{ConjugateTranspose}[\text{ket}[i]],\{i,0,N-1\}]}\\
\pmb{\qquad+\text{Sum}[\text{ket}[i].\text{ConjugateTranspose}[\text{ket}[i]],\{i,N,15\}];}\\
\pmb{\rho [\psi \_]\text{:=}\psi .\text{ConjugateTranspose}[\psi ];}\\
\pmb{U=\text{Urp}[x,M];}\\
\)
\end{doublespace}

Ahora se prepara el estado inicial y se ejecuta el algoritmo.


\begin{doublespace}
\noindent\(
\pmb{\text{$\psi $0}=\text{KroneckerProduct}\left[\text{ket}_0,\text{ket}_0,\text{ket}_0,\text{ket}_0\right];}\\
\pmb{\text{$\psi $0}=\text{KroneckerProduct}[H,H,H,H].\text{$\psi $0};}\\
\pmb{\psi =\text{KroneckerProduct}\left[\text{ket}_0,\text{ket}_0,\text{ket}_0,\text{ket}_1\right];}\\
\pmb{\text{$\psi $t}=\text{KroneckerProduct}[\text{$\psi $0},\psi ];}\\
\pmb{\text{$\psi $t}=(\text{KroneckerProduct}[\text{Id},\text{Id},\text{Id},P_0,\text{Id},\text{Id},\text{Id},\text{Id}]}\\
\pmb{\qquad+\text{KroneckerProduct}[\text{Id},\text{Id},\text{Id},P_1,U]).\text{$\psi
$t};}\\
\pmb{\text{$\psi $t}=(\text{KroneckerProduct}[\text{Id},\text{Id},P_0,\text{Id},\text{Id},\text{Id},\text{Id},\text{Id}]}\\
\pmb{\qquad+\text{KroneckerProduct}[\text{Id},\text{Id},P_1,\text{Id},\text{MatrixPower}[U,2]]).\text{$\psi
$t};}\\
\pmb{\text{$\psi $t}=(\text{KroneckerProduct}[\text{Id},P_0,\text{Id},\text{Id},\text{Id},\text{Id},\text{Id},\text{Id}]}\\
\pmb{\qquad+\text{KroneckerProduct}[\text{Id},P_1,\text{Id},\text{Id},\text{MatrixPower}[U,2^2]]).\text{$\psi
$t};}\\
\pmb{\text{$\psi $t}=(\text{KroneckerProduct}[P_0,\text{Id},\text{Id},\text{Id},\text{Id},\text{Id},\text{Id},\text{Id}]}\\
\pmb{\qquad+\text{KroneckerProduct}[P_1,\text{Id},\text{Id},\text{Id},\text{MatrixPower}[U,2^3]]).\text{$\psi
$t};}\\
\pmb{\text{$\psi $t}=\text{KroneckerProduct}[\text{ConjugateTranspose}[\text{FourierMatrix}[2^4]],\text{Id},\text{Id},\text{Id},\text{Id}].\text{$\psi
$t};}\\
\pmb{\text{For}[i=0,i<32,i\text{++},\text{sub0}=\text{Mod}[i,2];}\\
\pmb{\text{sub1}=\text{Which}[i==0,0,\text{Mod}[i,2]==0}\\
\pmb{\qquad\&\&\text{sub1}==0,1,\text{Mod}[i,2]==0}\\
\pmb{\qquad\&\&\text{sub1}==1,0,\text{True},\text{sub1}];}\\
\pmb{\text{sub2}=\text{Which}[i==0,0,\text{Mod}[i,2^2]==0}\\
\pmb{\qquad\&\&\text{sub2}==0,1,\text{Mod}[i,2^2]==0}\\
\pmb{\qquad\&\&\text{sub2}==1,0,\text{True},\text{sub2}];}\\
\pmb{\text{sub3}=\text{Which}[i==0,0,\text{Mod}[i,2^3]==0}\\
\pmb{\qquad\&\&\text{sub3}==0,1,\text{Mod}[i,2^3]==0}\\
\pmb{\qquad\&\&\text{sub3}==1,0,\text{True},\text{sub3}];}\\
\pmb{\text{$\psi $t}_i=\text{KroneckerProduct}\left[P_{\text{sub3}},P_{\text{sub2}},P_{\text{sub1}},P_{\text{sub0}},\text{Id},\text{Id},\text{Id},\text{Id}\right].\text{$\psi
$t};}\\
\pmb{\left.\text{$\psi $t}_i=\text{ConjugateTranspose}\left[\text{$\psi $t}_i\right].\text{$\psi $t}_i\right];}\\
\pmb{}\\
\pmb{\text{(*Organizar resultados de las medidas y graficar*)}}\\
\pmb{L=\text{Table}\left[\left\{i,\frac{i}{2^4},\text{$\psi $t}_i[[1,1]]\right\},\{i,0,31\}\right];}\\
\pmb{\text{ListPlot}\left[\text{Table}\left[\left\{\frac{i}{2^4}\right],\text{$\psi $t}_i[[1,1]]\right\},\{i,0,16\}\right],\text{PlotRange}\to
\text{All}}\)
\end{doublespace}

\begin{doublespace}
\noindent\(\pmb{\text{(*Procesar medidas para tener la factorizaci{\' o}n*)}}\\
\pmb{\text{(*Fracciones continuas*)}}\\
\pmb{\text{L1}=\text{Table}[\{\text{Denominator}[\text{FromContinuedFraction}[}\\
\pmb{\qquad\text{ContinuedFraction}[\text{Transpose}[L][[2,i]]]]],L[[i,3]]\},\{i,1,16\}];}\\
\pmb{\text{For}[i=1,i\leq \text{Dimensions}[\text{L1}][[1]],i\text{++},}\\
\pmb{\qquad\text{For}[j=1,j\leq \text{Dimensions}[\text{L1}][[1]],j\text{++},}\\
\pmb{\qquad\text{If}[\text{L1}[[i,1]]==\text{L1}[[j,1]]\&\&i\neq
j,\text{L1}[[i,2]]\text{+=}\text{L1}[[j,2]];}\\
\pmb{\text{L1}=\text{Drop}[\text{L1},\{j\}];]]]}\\
\pmb{\text{For}[i=1,i\leq \text{Dimensions}[\text{L1}][[1]],i\text{++},\text{If}[\text{OddQ}[\text{L1}[[i,1]]],\text{L1}=\text{Drop}[\text{L1},\{i\}];]]}\\
\pmb{\text{For}\left[i=1,i\leq \text{Dimensions}[\text{L1}][[1]],i\text{++},\text{If}\left[\text{Mod}\left[x^{\text{L1}[[i,1]]},M\right]\neq 1,\text{L1}=\text{Drop}[\text{L1},\{i\}];\right]\right]}\\
\pmb{\text{For}[i=1,i\leq \text{Dimensions}[\text{L1}][[1]],i\text{++},}\\
\pmb{\qquad\text{If}[\text{Mod}[x^{\frac{\text{L1}[[1,1]]}{2}},M]==M-1,\text{L1}=\text{Drop}[\text{L1},\{i\}];]]}\\
\pmb{\text{For}\left[i=1,i\leq \text{Dimensions}[\text{L1}][[1]],i\text{++},\text{L1}[[i,2]]=\frac{\text{L1}[[i,2]]}{\text{Sum}[\text{L1}[[j,2]],\{j,1,\text{Dimensions}[\text{L1}][[1]]\}]};\right]}\\
\pmb{\text{L2}=\text{Table}\left[\left\{\text{GCD}\left[x^{\frac{\text{L1}[[i,1]]}{2}}+1,M\right],\text{L1}[[i,2]]\right\},\{i,1,\text{Dimensions}[\text{L1}][[1]]\}\right];}\\
\pmb{\text{L3}=\text{Table}\left[\left\{\text{GCD}\left[x^{\frac{\text{L1}[[i,1]]}{2}}-1,M\right],\text{L1}[[i,2]]\right\},\{i,1,\text{Dimensions}[\text{L1}][[1]]\}\right];}\)
\end{doublespace}

\section{Python}

Primero, se importan todos los módulos necesarios. Entre ellos, Numpy para variables y funciones numéricas, tgates8 para las compuertas de transmones.

\begin{verbatim}
import matplotlib.pyplot as plt
import matplotlib as mpl
import numpy as np
from scipy.stats import norm
from qutip import *
import tgates8
import time
\end{verbatim}

Ahora se definen las operaciones que son explícitas para este algoritmo. Ellas son la transformada de Hadamard y los operadores de reflexión.

\begin{verbatim}
def QFT4d(psi0, target1, target2, target3, target4):
    res = tgates8.SWAP(psi0, target2, target3)
    res = tgates8.SWAP(res.states[-1], target1, target4)
    res = tgates8.H(res.states[-1], target4)
    res = tgates8.CP(res.states[-1], target4, target3, -2*np.pi/2**2)
    res = tgates8.H(res.states[-1], target3)
    res = tgates8.CP(res.states[-1], target4, target2, -2*np.pi/2**3)
    res = tgates8.CP(res.states[-1], target3, target2, -2*np.pi/2**2)
    res = tgates8.H(res.states[-1], target2)
    res = tgates8.CP(res.states[-1], target4, target1, -2*np.pi/2**4)
    res = tgates8.CP(res.states[-1], target3, target1, -2*np.pi/2**3)
    res = tgates8.CP(res.states[-1], target2, target1, -2*np.pi/2**2)
    return tgates8.H(res.states[-1], target1)

def CSWAP(psi0, control, target1, target2):
    res = tgates8.CCNOT(psi0, control, target1, target2)
    res = tgates8.CCNOT(psi0, control, target2, target1)
    return tgates8.CCNOT(psi0, control, target1, target2)

def CMUL7(psi0, control, target1, target2, target3, target4):
    res = tgates8.CNOT(psi0, control, target1)
    res = tgates8.CNOT(res.states[-1], control, target2)
    res = tgates8.CNOT(res.states[-1], control, target3)
    res = tgates8.CNOT(res.states[-1], control, target4)
    res = CSWAP(res.states[-1], control, target2, target3)
    res = CSWAP(res.states[-1], control, target1, target2)
    return CSWAP(res.states[-1], control, target1, target4)

qN = 2**8
\end{verbatim}

Finalmente, se define el estado inicial y se ejecuta el algoritmo. Se realizan el doble de interaciones de las necesarias, igual que en el código de Wolfram Mathematica, para analizar lo que ocurre si se realizan más rotaciones de las necesarias.

\begin{verbatim}
psi0 = tensor(basis(2,0), basis(2,0), basis(2,0), basis(2,0),
              basis(2,0), basis(2,0), basis(2,0), basis(2,0))

res = tgates8.X(psi0, 7)

qsave(res, 'rp_0')

res = tgates8.H(res.states[-1], 0)
res = tgates8.H(res.states[-1], 1)
res = tgates8.H(res.states[-1], 2)
res = tgates8.H(res.states[-1], 3)

qsave(res, 'rp_1')

res = CMUL7(res.states[-1], 3, 4, 5, 6, 7)

qsave(res, 'rp_2')

res = CMUL7(res.states[-1], 2, 4, 5, 6, 7)
res = CMUL7(res.states[-1], 2, 4, 5, 6, 7)

qsave(res, 'rp_3')

res = CMUL7(res.states[-1], 1, 4, 5, 6, 7)
res = CMUL7(res.states[-1], 1, 4, 5, 6, 7)
res = CMUL7(res.states[-1], 1, 4, 5, 6, 7)
res = CMUL7(res.states[-1], 1, 4, 5, 6, 7)

qsave(res, 'rp_4')

res = CMUL7(res.states[-1], 0, 4, 5, 6, 7)
res = CMUL7(res.states[-1], 0, 4, 5, 6, 7)
res = CMUL7(res.states[-1], 0, 4, 5, 6, 7)
res = CMUL7(res.states[-1], 0, 4, 5, 6, 7)
res = CMUL7(res.states[-1], 0, 4, 5, 6, 7)
res = CMUL7(res.states[-1], 0, 4, 5, 6, 7)
res = CMUL7(res.states[-1], 0, 4, 5, 6, 7)
res = CMUL7(res.states[-1], 0, 4, 5, 6, 7)

qsave(res, 'rp_5')

res = QFT4d(res.states[-1], 0, 1, 2, 3)

qsave(res, 'rp_6')

\end{verbatim}

