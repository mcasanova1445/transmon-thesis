\begin{abstract}
El computador cuántico es la promesa tecnológica más relevante del presente siglo, siendo los circuitos cuánticos superconductores los candidatos idóneos para la construcción escalable de procesadores cuánticos eficientes. Un procesador cuántico ejecuta algoritmos cuánticos aplicando secuencias programables de compuertas cuánticas a un registro de entrada conformado por qubits, los cuales evolucionan a un estado cuántico final, el cual es la solución de un algoritmo. Dicho estado cuántico final se caracteriza por tener una familia de correlaciones cuánticas tales como entrelazamiento, discordia cuántica y medida geométrica de la discordia. Experimentalmente, se han logrado implementar en la última década procesadores cuánticos superconductores en base a arquitecturas de transmones, con qubits y qutrits, para los algoritmos cuánticos de Deutsch, Deutsch-Jozsa y Grover. En el presente trabajo se diseñaron y simularon, en base a arquitecturas superconductoras de transmones, procesadores cuánticos que implementen eficientemente los algoritmos cuánticos de Grover y Shor. Además, el internet cuántico permite la comunicación cuántica entre procesadores cuánticos remotos, utilizando procesadores cuánticos locales, que están interconectados por canales de comunicación cuántica que permiten la transmisión de qubits entre los diferentes procesadores, pra resolver problemas que son inviables clásicamente, razón por la cual se estudia la posibilidad de construir circuitos cuánticos del algoritmo de Google Cuántico, así como su representación en una arquitectura superconductora de transmones. En cada uno de los casos estudiados se diseñaron los circuitos cuánticos asociados, su simulación en Python y Mathematica, y se estudió de la información cuántica generada respectivamente.
\end{abstract}

