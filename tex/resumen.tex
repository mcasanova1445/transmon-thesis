\begin{abstract}
El computador cuántico es la promesa tecnológica más relevante del presente siglo, siendo los circuitos cuánticos superconductores los candidatos idóneos para la construcción escalable de procesadores cuánticos eficientes. Un procesador cuántico ejecuta algoritmos cuánticos aplicando secuencias programables de compuertas cuánticas a un registro de entrada conformado por qubits, los cuales evolucionan a un estado cuántico final, el cual es la solución de un algoritmo. Dicho estado cuántico final se caracteriza por tener una familia de correlaciones cuánticas tales como entrelazamiento, discordia cuántica y medida geométrica de la discordia. Experimentalmente, se han logrado implementar en la última década procesadores cuánticos superconductores en base a arquitecturas de transmones, con qubits y qutrits, para los algoritmos cuánticos de Deutsch, Deutsch-Jozsa y Grover. En el presente trabajo deseamos diseñar y simular, en base a arquitecturas superconductoras de transmones, procesadores cuánticos que implementen más eficientemente los algoritmos cuánticos de Grover y Shor. Además, usando algoritmos cuánticos que resuelven sistemas de ecuaciones lineales, se desea explorar la posibilidad de construir el circuito cuántico del algoritmo de Google Cuántico, así como su representación en una arquitectura superconductora de transmones. En cada uno de los casos mencionados se harán el diseño de los circuitos cuánticos asociados, su simulación en Python y Mathematica, y el estudio de la información cuántica generada respectivamente. Se estudiarán las potenciales aplicaciones que presentan estos procesadores cuánticos para el desarrollo de redes cuánticas que conduzcan en un Internet Cuántico eficiente.
\end{abstract}

