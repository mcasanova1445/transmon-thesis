\chapter{Cálculos de Hamiltonianos}

\section{Hamiltoniano de Jaynes-Cummings}

El Hamiltoniano de Jaynes-Cummings es un Hamiltoniano diagonal que representa un
sistema de dos niveles interactuando con un modo cuantizado de una cavidad óptica.

\begin{equation}
    \hat{H}_{JC} = \hat{H}_r + \hat{H}_q + \hat{H}_{qr} = \omega_r a^\dag a - \frac{1}{2} \omega_q \sigma_z + g(a \sigma_+ + a^\dag \sigma_-)
\end{equation}

\section{Hamiltoniano multiquibit}

El modelo de Jaynes-Cummings para varios qubits sin el término de la energía de la cavidad es el siguiente:

\begin{equation}
    \hat{H} = \hat{H}_q + \hat{H}_{qr} = - \frac{1}{2} \sum\limits_i \omega_{qi} \sigma_{zi} + \sum\limits_i g_i (a \sigma_{+ i} + a^\dagger \sigma_{- i})
\end{equation}

\section{Pulsos de microondas}

Para operar sobre los qubits se aplican pulsos de microondas.

\begin{equation}
    \hat{H}_d = \sum\limits_k (a+a^\dagger) (\xi_k e^{-i\omega_d^{(k)}t} + \xi_k^*e^{i\omega_d^{(k)}t})
\end{equation}

RWA:

\begin{equation}
    \hat{H}_d=\sum\limits_k a\xi_k^*e^{i\omega_d^{(k)}t}+ a^\dagger\xi_ke^{-i\omega_d^{(k)}t}
\end{equation}

\section{Régimen rotacional del pulso}

Partiendo del Hamiltoniano de Jaynes-Cummings para un sistema multiqubit con pulsos de microondas bajo la aproximación de onda rotacional:

\begin{equation}
    \hat{H}_1 = \hat{H}_{syst} + \hat{H}_d = \omega_r a^\dag a - \frac{1}{2} \sum\limits_i \omega_{qi} \sigma_{zi} + \sum\limits_i g_i (a \sigma_{+ i} + a^\dagger \sigma_{- i}) + \sum\limits_k a\xi_k^*e^{i\omega_d^{(k)}t}+ a^\dagger\xi_ke^{-i\omega_d^{(k)}t}
\end{equation}

Aplicamos la siguiente transformación unitaria para entrar en el regimen rotacional del pulso aplicado

\begin{equation}
    U(t) = exp[\sum\limits_n-i \omega_d^{(n)} t(a^\dagger a - \frac{1}{2} \sum\limits_i \sigma_{z i})]
\end{equation}

De esta manera, el Hamiltoniano en el régimen rotacional del pulso tendrá la siguiente forma:

\begin{equation}
    \hat{H}_2 = U^\dagger (\hat{H}_{syst} + \hat{H}_d) U - i U^\dagger \dot{U}
\end{equation}

Donde $\dot{U}$ representa la derivada temporal del operador unitario $U$.

Utilizaremos la formula de Baker-Campbell-Hausdorff para calcular este Hamiltoniano, ya que esta nos permite realizar el producto con los exponenciales de operadores calculando sólo conmutadores.

\begin{equation}
    e^{-\lambda X} H e^{\lambda X} = H + \lambda [H,X] + \frac{\lambda^2}{2!}[[H,X],X] + ...
\end{equation}

En cuanto a los conmutadores, podemos utilizar las siguientes identidades:

\begin{align}
    [a, a^\dagger] &= 1
    \label{eq:a_ad} \\
    [a, a^\dagger a] &= a a^\dagger a - a^\dagger a a = (a a^\dagger - a^\dagger a) a = [a, a^\dagger] a = a
    \label{eq:a_ada} \\
    [a^\dagger, a^\dagger a] &= a^\dagger a^\dagger a - a^\dagger a a^\dagger = a^\dagger (a^\dagger a - a a^\dagger) = a^\dagger [a^\dagger, a] = -a^\dagger
    \label{eq:ad_ada} \\
    [\sigma_+, \sigma_z] &= 2 \sigma_+
    \label{eq:sp_sz} \\
    [\sigma_-, \sigma_z] &= - 2 \sigma_-
    \label{eq:sm_sz} \\
    [\sigma_-, \sigma_+] &= \sigma_z
    \label{eq:sm_sp}
\end{align}

Para que el cálculo sea visualmente más manejable, aprovechamos la propiedad distributiva de los conmutadores y separaremos el Hamiltoniano $\hat{H}_1$ en los siguientes términos:

\begin{enumerate}
    \item $\hat{H}_r = \omega_r a^\dag a$
    \item $\hat{H}_q = - \frac{1}{2} \sum\limits_i \omega_{qi} \sigma_{zi}$
    \item $\hat{H}_{qr} = \sum\limits_i g_i (a \sigma_{+ i} + a^\dagger \sigma_{- i})$
    \item $\hat{H}_d = \sum\limits_k (a\xi_k^*e^{i\sum\limits_k \omega_d^{(k)}t}+ a^\dagger\xi_ke^{-i\sum\limits_k \omega_d^{(k)}t})$
\end{enumerate}


Con el primer término tenemos el siguiente conmutador:

\begin{equation}
    [\omega_r a^\dag a, \sum\limits_n-i \omega_d^{(n)} t(a^\dagger a - \frac{1}{2} \sum\limits_i \sigma_{z i})] = 0
\end{equation}

El cual es igual a cero, ya que $a^\dagger a$ conmuta con sí mismo, como todo operador, y con $\sigma_{z_i}$, ya que actúan sobre particiones distintas del sistema.

Con el segundo término tenemos el siguiente conmutador:

\begin{equation}
    [- \frac{1}{2} \sum\limits_i \omega_{qi} \sigma_{zi} , \sum\limits_n-i \omega_d^{(n)} t(a^\dagger a - \frac{1}{2} \sum\limits_i \sigma_{z i})] = 0
\end{equation}

El cual es igual a cero, ya que $\sigma_{z_i}$ conmuta consigo mismo y con $a^\dagger a$.

Con el tercer término tenemos el siguiente conmutador:

\begin{multline}
\left[\sum\limits_i g_i (a \sigma_{+ i} + a^\dagger \sigma_{- i}), \sum\limits_n-i \omega_d^{(n)} t(a^\dagger a - \frac{1}{2} \sum\limits_i \sigma_{z i})\right] = \\
\sum\limits_n-i \omega_d^{(n)} t \left[\sum\limits_i g_i (a \sigma_{+ i} + a^\dagger \sigma_{- i}), (a^\dagger a)\right] \\
+ \sum\limits_n-i \omega_d^{(n)} t \left[\sum\limits_i g_i (a \sigma_{+ i} + a^\dagger \sigma_{- i}), (- \frac{1}{2} \sum\limits_i \sigma_{z i})\right] = \\
\sum\limits_n-i \omega_d^{(n)} t \sum\limits_i g_i (a \sigma_{+ i} - a^\dagger \sigma_{- i})
- \sum\limits_n-i \omega_d^{(n)} t \sum\limits_i g_i (a \sigma_{+ i} - a^\dagger \sigma_{- i}) = 0
\end{multline}

En este caso, utilizamos el hecho de que $a$, $a^\dagger$ y $a^\dagger a$ conmutan con $\sigma_{z_i}$, $\sigma_{+_i}$ y $\sigma_{-_i}$, igual que $\sigma_{z_i}$, $\sigma_{+_i}$ y $\sigma_{-_i}$ con $\sigma_{z_j}$, $\sigma_{+_j}$ y $\sigma_{-_j}$, donde $i \neq j$, ya que actuan sobre particiones distintas. También utilizamos las identidades \ref{eq:a_ad} - \ref{eq:sm_sp} y el hecho de que todo operador conmuta con sí mismo. De esta manera llegamos a una resta de dos términos iguales, así que este conmutador también es cero.

Con el cuarto término tenemos el siguiente conmutador:

\begin{multline}
    \left[\sum\limits_k \left(a\xi_k^*e^{i\sum\limits_k \omega_d^{(k)}t}+ a^\dagger\xi_ke^{-i\sum\limits_k \omega_d^{(k)}t}\right), \sum\limits_n-i \omega_d^{(n)} t\left(a^\dagger a - \frac{1}{2} \sum\limits_i \sigma_{z i}\right)\right] = \\
    \left(\sum\limits_n-i \omega_d^{(n)} t\right) \sum\limits_k \left(a\xi_k^*e^{i\sum\limits_k \omega_d^{(k)}t} - a^\dagger\xi_ke^{-i\sum\limits_k \omega_d^{(k)}t}\right)
\end{multline}

Aquí hemos utilzado el hecho de que $a$ y $a^\dagger$ conmutan con $\sigma_{z_i}$ y las identidades \ref{eq:a_ada} y \ref{eq:ad_ada}. Este conmutador no es igual a cero como los anteriores, así que tenemos que utilizar este resultado para calcular $[[H,X],X]$.

\begin{multline}
    \left[\left(\sum\limits_n-i \omega_d^{(n)} t\right) \sum\limits_k \left(a\xi_k^*e^{i\sum\limits_k \omega_d^{(k)}t} - a^\dagger\xi_ke^{-i\sum\limits_k \omega_d^{(k)}t}\right), \sum\limits_n-i \omega_d^{(n)} t\left(a^\dagger a - \frac{1}{2} \sum\limits_i \sigma_{z i}\right)\right] = \\
    \left(\sum\limits_n-i \omega_d^{(n)} t\right)^2 \sum\limits_k \left(a\xi_k^*e^{i\sum\limits_k \omega_d^{(k)}t} + a^\dagger\xi_ke^{-i\sum\limits_k \omega_d^{(k)}t}\right)
\end{multline}
 
De igual manera que en el conmutador anterior, hemos utilzado el hecho de que $a$ y $a^\dagger$ conmutan con $\sigma_{z_i}$ y las identidades \ref{eq:a_ada} y \ref{eq:ad_ada}. Este conmutador no es igual a cero como los anteriores, así que tenemos que utilizar este resultado para calcular $[[[H,X],X],X]$.

\begin{multline}
    \left[\left(\sum\limits_n-i \omega_d^{(n)} t\right)^2 \sum\limits_k \left(a\xi_k^*e^{i\sum\limits_k \omega_d^{(k)}t}+ a^\dagger\xi_ke^{-i\sum\limits_k \omega_d^{(k)}t}\right), \left(\sum\limits_n-i \omega_d^{(n)} t\right)\left(a^\dagger a - \frac{1}{2} \sum\limits_i \sigma_{z i}\right)\right] = \\
    \left(\sum\limits_n-i \omega_d^{(n)} t\right)^3 \sum\limits_k \left(a\xi_k^*e^{i\sum\limits_k \omega_d^{(k)}t} - a^\dagger\xi_ke^{-i\sum\limits_k \omega_d^{(k)}t}\right)
\end{multline}

De igual manera que en el conmutador anterior, hemos utilzado el hecho de que $a$ y $a^\dagger$ conmutan con $\sigma_{z_i}$ y las identidades \ref{eq:a_ada} y \ref{eq:ad_ada}. Este conmutador no es igual a cero como los anteriores, así que tenemos que utilizar este resultado para calcular $[[[[H,X],X],X],X]$.

\begin{multline}
    \left[\left(\sum\limits_n-i \omega_d^{(n)} t\right)^3 \sum\limits_k \left(a\xi_k^*e^{i\sum\limits_k \omega_d^{(k)}t} - a^\dagger\xi_ke^{-i\sum\limits_k \omega_d^{(k)}t}\right), \left(\sum\limits_n-i \omega_d^{(n)} t\right)\left(a^\dagger a - \frac{1}{2} \sum\limits_i \sigma_{z i}\right)\right] = \\
    \left(\sum\limits_n-i \omega_d^{(n)} t\right)^4 \sum\limits_k \left(a\xi_k^*e^{i\sum\limits_k \omega_d^{(k)}t} + a^\dagger\xi_ke^{-i\sum\limits_k \omega_d^{(k)}t}\right)
\end{multline}

En este punto podemos notar cierto patrón. Esta serie de conmutadores nos recuerda a la serie de Taylor de la función exponencial con argumento $\sum\limits_n-i \omega_d^{(n)} t$. Al sustituirlos en la fórmula de Baker-Campbell-Hausdorff, vemos que efectivamente se trata de este exponencial. Entonces, el primer término del Hamiltoniano $\hat{H}_2$ es:

\begin{equation}
    U^\dagger (\hat{H}_1) U = \hat{H}_{syst} + \sum\limits_k (e^{\sum\limits_n-i \omega_d^{(n)} t} a\xi_k^*e^{\sum\limits_k i\omega_d^{(k)}t}+ e^{-\sum\limits_n-i \omega_d^{(n)} t} a^\dagger\xi_ke^{-\sum\limits_k i\omega_d^{(k)}t}) = \hat{H}_{syst} + \sum\limits_k (a\xi_k^* + a^\dagger\xi_k)
\end{equation}

Por otro lado, el segundo término es:

\begin{equation}
    - i U^\dagger \dot{U} = -i U^\dagger (-i \sum\limits_n \omega_d^{(n)} (a^\dagger a - \frac{1}{2} \sum\limits_i \sigma_{z i})) U = - \sum\limits_n \omega_d^{(n)}(a^\dagger a - \frac{1}{2} \sum\limits_i \sigma_{z i})
\end{equation}

Es decir, $-i$ por la derivada temporal del argumento del exponencial en el que consiste $U$. Esto es porque todo exponencial conmuta con su argumento y en este caso, la derivada interna de U es igual al argumento del exponencial entre el escalar $t$, por lo que también conmuta con U. Además de que como $U$ es unitario, se cumple que $U^\dagger U = \mathds{1}$.

Finalmente, sumando y agrupando términos, nos queda que el Hamiltoniano en el régimen rotacional del pulso es:

\begin{equation}
    \hat{H}_2 = (\omega_r - \sum\limits_n \omega_d^{(n)}) a^\dag a - \frac{1}{2} \sum\limits_i (\omega_{qi} - \sum\limits_n \omega_d^{(n)}) \sigma_{zi} + \sum\limits_i g_i (a \sigma_{+ i} + a^\dagger \sigma_{- i}) + \sum\limits_k (a\xi_k^* + a^\dagger\xi_k)
\end{equation}

En el caso de un pulso de un sólo modo, este Hamiltoniano toma la forma:

\begin{equation}
    \hat{H} = \Delta_r a^\dagger a - \frac{1}{2} \sum\limits_i \Delta_{qi} \sigma_{zi} + \sum\limits_i g_i (a \sigma_{+ i} + a^\dagger \sigma_{- i}) + (a\xi^*+a^\dagger\xi )
\end{equation}

Donde $\Delta_r = \omega_r - \omega_d$ es la diferencia entre la frecuencia de resonancia del resonador y la frecuencia central del pulso, y $\Delta_{qi} = \omega_{qi} - \omega_d$ es la diferencia entre la frecuencia de resonancia de cada qubit y la frecuencia central del pulso.

\section{Efecto del pulso sobre el qubit}

Ahora desplazaremos el campo $a$, aplicando el operador de desplazamineto

\begin{equation}
    D(\alpha) = exp[\alpha a^\dagger - \alpha^* a]
\end{equation}

Al Hamiltoniano $\hat{H}_2$ monomodo, con $\dot{\alpha} = -i \Delta_r \alpha -i \xi$, para eliminar el efecto directo del pulso sobre el resonador y ver cómo afecta a los qubits.

De esta manera, el nuevo Hamiltoniano será:

\begin{equation}
    \hat{H}_3 = D^\dagger (\alpha) \hat{H}_2 D(\alpha) -i D^\dagger(\alpha) \dot{D}(\alpha)
\end{equation}

Los operadores de desplazamiento cumplen con las siguientes propiedades:

\begin{align}
    D(-\alpha) &= D^\dagger(\alpha) = D^{-1}(\alpha) \\
    D^\dagger(\alpha) a D(\alpha) &= a + \alpha \\
    D^\dagger(\alpha) a^\dagger D(\alpha) &= a^\dagger + \alpha^*
\end{align}

Utilizando estas dos propiedes, podemos calcular $D^\dagger(\alpha) \hat{H}_2 D(\alpha)$ directamente sin utilizar la expansión de Baker-Campbell-Hausdorff, pues basta con sustituir $a$ por $a+\alpha$ y $a^\dagger$ por $a^\dagger+\alpha^*$.

\begin{multline}
  D^\dagger(\alpha) \hat{H}_2 D(\alpha) = \Delta_r (a^\dagger + \alpha^*) (a + \alpha) - \frac{1}{2} \sum\limits_i \Delta_{qi} \sigma_{zi} + \sum\limits_i g_i [(a+\alpha) \sigma_{+ i} + (a^\dagger+\alpha^*) \sigma_{- i}] \\
  + [(a+\alpha)\xi^*+(a^\dagger+\alpha^*)\xi]
\end{multline}

Estas mismas propiedades también se utilizan para calcular el otro término de $\hat{H}_3$, de la siguiente manera:

\begin{equation}
    -i D^\dagger(\alpha) \dot{D}(\alpha) = -i D^\dagger(\alpha) (\dot{\alpha} a^\dagger - \dot{\alpha}^* a) D(\alpha) = -i[\dot{\alpha} (a^\dagger + \alpha^*) - \dot{\alpha}^* (a + \alpha)]
\end{equation}

Finalemente, sumando, sustituyendo $\dot{\alpha}$ y agrupando, nos queda:

\begin{equation}
    \hat{H}_3 = \Delta_r a^\dagger a - \frac{1}{2} \sum\limits_i \Delta_{qi} \sigma_{zi} + \sum\limits_i g_i (a \sigma_{+i} + a^\dagger \sigma_{-i}) + \sum\limits_i g_i (\alpha \sigma_{+i} + \alpha^* \sigma_{-i}) - \Delta_c \alpha \alpha^*
\end{equation}

El término $-\Delta_r \alpha \alpha^*$ se desprecia, ya que sólo representa una fase global en la evolución del sistema.

\section{Régimen dispersivo}

Finalmente, aplicamos la transformación

\begin{equation}
    U = exp[\sum\limits_i \frac{g_i} {\Delta_i} (a^\dagger \sigma_{-i} - a \sigma_{+i})]
\end{equation}

Donde $\Delta_i = \omega_{qi} - \omega_r$ y realizamos la expansión de Baker-Campbell-Hausdorff de segundo grado sobre los términos $\frac{g_i}{\Delta_i} \ll 1$. El Hamiltoniano efectivo $\hat{H}_{eff}$ será la aproximación del Hamiltoniano $\hat{H}_4$ resultante de esta expansión.

\begin{equation}
    \hat{H}_{eff} \approx \hat{H}_4 = U^\dagger \hat{H}_3 U
\end{equation}

Para resolver los conmutadores seguiremos el esquema utilizado anteriormente y separaremos el Hamiltoniano $\hat{H}_3$ en los siguientes términos:

\begin{enumerate}
    \item $\hat{H}_r = \Delta_r a^\dagger a$
    \item $\hat{H}_q = - \frac{1}{2} \sum\limits_i \Delta_{qi} \sigma_{zi}$
    \item $\hat{H}_{qr} = \sum\limits_i g_i (a \sigma_{+i} + a^\dagger \sigma_{-i})$
    \item $\hat{H}_d = \sum\limits_i g_i (\alpha \sigma_{+i} + \alpha^* \sigma_{-i})$
\end{enumerate}

Con el primer término tenemos el siguiente conmutador:

\begin{equation}
    [\Delta_r a^\dagger a, \sum\limits_i \frac{g_i} {\Delta_i} (a^\dagger \sigma_{-i} - a \sigma_{+i})] = \Delta_r \sum\limits_i \frac{g_i} {\Delta_i} (a^\dagger \sigma_{-i} + a \sigma_{+i})
\end{equation}

Con el segundo término tenemos el siguiente conmutador:

\begin{equation}
    [- \frac{1}{2} \sum\limits_i \Delta_{qi} \sigma_{zi}, \sum\limits_i \frac{g_i} {\Delta_i} (a^\dagger \sigma_{-i} - a \sigma_{+i})] = - \sum\limits_i \frac{g_i} {\Delta_i} \Delta_{qi} (a^\dagger \sigma_{-i} + a \sigma_{+i})
\end{equation}

Con el tercer término tenemos el siguiente conmutador:

\begin{multline}
[\sum\limits_i g_i (a \sigma_{+i} + a^\dagger \sigma_{-i}), \sum\limits_j \frac{g_j} {\Delta_j} (a^\dagger \sigma_{-j} - a \sigma_{+j})] = \\
\sum\limits_{ij} g_i \frac{g_j}{\Delta_j} \left([ a \sigma_{+i}, a^\dagger \sigma_{-j}] +
[a \sigma_{+i}, - a \sigma_{+j}] +
[a^\dagger \sigma_{-i}, a^\dagger \sigma_{-j}] +
[a^\dagger \sigma_{-i}, - a \sigma_{+j}]\right) =\\
\sum\limits_{ij} g_i \frac{g_j}{\Delta_j} \left([ a \sigma_{+i}, a^\dagger \sigma_{-j}] +
[a^\dagger \sigma_{-i}, - a \sigma_{+j}]\right) =\\
\sum\limits_{ij} g_i \frac{g_j}{\Delta_j} \left(
    a \sigma_{+i} a^\dagger \sigma_{-j} - a^\dagger \sigma_{-j} a \sigma_{+i}
    - a^\dagger \sigma_{-i} a \sigma_{+j} + a \sigma_{+j} a^\dagger \sigma_{-i}\right) =\\
\sum\limits_{ij} g_i \frac{g_j}{\Delta_j} \left(
    a a^\dagger \sigma_{+i} \sigma_{-j} - a^\dagger a \sigma_{-j} \sigma_{+i}
    - a^\dagger a \sigma_{-i} \sigma_{+j} + a a^\dagger \sigma_{+j} \sigma_{-i}\right) =\\
\sum\limits_{ij} g_i \frac{g_j}{\Delta_j} \left(
    (1 + a^\dagger a) \sigma_{+i} \sigma_{-j} - a^\dagger a \sigma_{-j} \sigma_{+i}
    - a^\dagger a \sigma_{-i} \sigma_{+j} + (1 + a^\dagger a) \sigma_{+j} \sigma_{-i}\right) =\\
\sum\limits_{ij} g_i \frac{g_j}{\Delta_j} \left(
    \sigma_{+i} \sigma_{-j} + a^\dagger a \sigma_{+i} \sigma_{-j} - a^\dagger a \sigma_{-j} \sigma_{+i}
    - a^\dagger a \sigma_{-i} \sigma_{+j} + \sigma_{+j} \sigma_{-i} + a^\dagger a \sigma_{+j} \sigma_{-i}\right) =\\
\sum\limits_{ij} g_i \frac{g_j}{\Delta_j} \left(\sigma_{+i} \sigma_{-j} + \sigma_{+j} \sigma_{-i}\right) +
\sum\limits_{ij} g_i \frac{g_j}{\Delta_j} a^\dagger a \left(
    \sigma_{+i} \sigma_{-j} - \sigma_{-j} \sigma_{+i}
    - \sigma_{-i} \sigma_{+j} + \sigma_{+j} \sigma_{-i}\right) =\\
\sum\limits_{ij} g_i \frac{g_j}{\Delta_j} \left(\sigma_{+i} \sigma_{-j} + \sigma_{+j} \sigma_{-i}\right) -
2 \sum\limits_{i} \frac{g_i^2}{\Delta_i} a^\dagger a \sigma_{zi}
\end{multline}

Con el cuarto término tenemos el siguiente conmutador:

\begin{multline}
[\sum\limits_i g_i (\alpha \sigma_{+i} + \alpha^* \sigma_{-i}), \sum\limits_i \frac{g_i} {\Delta_i} (a^\dagger \sigma_{-i} - a \sigma_{+i})] =\\
\sum\limits_{ij} g_i \frac{g_j} {\Delta_j} \left([\alpha \sigma_{+i}, a^\dagger \sigma_{-j}] - [\alpha^* \sigma_{-i}, a \sigma_{+j}]\right) =
- \sum\limits_i \frac{g_i^2} {\Delta_i} (\alpha a^\dagger + \alpha^* a) \sigma_{zi}
\end{multline}

Sumando y agrupando se tiene

\begin{multline}
\Delta_r a^\dagger a - \frac{1}{2} \sum\limits_i \Delta_{qi} \sigma_{zi} + \sum\limits_i g_i (a \sigma_{+i} + a^\dagger \sigma_{-i}) + \sum\limits_i g_i (\alpha \sigma_{+i} + \alpha^* \sigma_{-i}) \\
+ \Delta_r \sum\limits_i \frac{g_i} {\Delta_i} (a^\dagger \sigma_{-i} + a \sigma_{+i}) - \sum\limits_i \frac{g_i} {\Delta_i} \Delta_{qi} (a^\dagger \sigma_{-i} + a \sigma_{+i}) \\
+ \sum\limits_{ij} g_i \frac{g_j}{\Delta_j} \left(\sigma_{+i} \sigma_{-j} + \sigma_{+j} \sigma_{-i}\right) - 2 \sum\limits_{i} \frac{g_i^2}{\Delta_i} a^\dagger a \sigma_{zi} - \sum\limits_i \frac{g_i^2} {\Delta_i} (\alpha a^\dagger + \alpha^* a) \sigma_{zi}
\end{multline}


\begin{multline}
    (\Delta_r - \sum\limits_i \frac{g_i^2}{\Delta_i} \sigma_{zi}) a^\dagger a - \frac{1}{2} \sum\limits_i (\Delta_{qi} + 2 \frac{g_i^2}{\Delta_i} a^\dagger a) \sigma_{zi} + \sum\limits_i g_i (a \sigma_{+i} + a^\dagger \sigma_{-i}) + \sum\limits_i g_i (\alpha \sigma_{+i} + \alpha^* \sigma_{-i}) \\
- \sum\limits_i \frac{g_i \Delta_i} {\Delta_i} (a^\dagger \sigma_{-i} + a \sigma_{+i}) + \sum\limits_{ij} g_i \frac{g_j}{\Delta_j} \left(\sigma_{+i} \sigma_{-j} + \sigma_{+j} \sigma_{-i}\right) - \sum\limits_i \frac{g_i^2}{\Delta_i} (\alpha a^\dagger + \alpha^* a) \sigma_{zi}
\end{multline}


\begin{multline}
    (\Delta_r - \sum\limits_i \frac{g_i^2}{\Delta_i} \sigma_{zi}) a^\dagger a - \frac{1}{2} \sum\limits_i (\Delta_{qi} + 2 \frac{g_i^2}{\Delta_i} a^\dagger a) \sigma_{zi} + \sum\limits_i g_i (\alpha \sigma_{+i} + \alpha^* \sigma_{-i}) \\
+ \sum\limits_{ij} g_i \frac{g_j}{\Delta_j} \left(\sigma_{+i} \sigma_{-j} + \sigma_{+j} \sigma_{-i}\right) - \sum\limits_i \frac{g_i^2}{\Delta_i} (\alpha a^\dagger + \alpha^* a) \sigma_{zi}
\end{multline}


% Dr = wr - wd
% Dqi = wqi - wd
% Di = wqi - wr
% Desaparece el acoplamiento resonador-qubits
% Aparece acoplamiento qubit-qubit


\begin{multline}
\hat{H} = \Delta_r a^\dagger a - \sum\limits_i \Delta_{qi} (\frac{1}{2} + 2 \frac{g_i^2} {\Delta_i} (a^\dagger a + \frac{1}{2})) \sigma_{zi} +
\sum\limits_{i \neq j} \frac{g_i g_j} {\Delta_j} (\sigma_{+i} \sigma_{-j} +\\
\sigma_{-i} \sigma_{+j}) +
\sum\limits_i g_i (\alpha \sigma_{+i} + \alpha^* \sigma_{-i}) -\\
\sum\limits_i \frac{g_i^2} {\Delta_i} (\alpha a^\dagger + \alpha^* a) \sigma_{zi}
\end{multline}

------------------------------

$$\hat{H} \approx \tilde{\Delta}_c a^\dagger a - \frac{1}{2} \sum\limits_i \tilde{\Delta}_{qi} \sigma_{zi} + \sum\limits_i (\Omega_i \sigma_{+i} + \Omega_i^* \sigma_{-i})$$
$$+ \sum\limits_{i \neq j} \frac{g_i g_j}{2 \Delta_i} (\sigma_{-i} \sigma_{+j}+\sigma_{+i} \sigma_{-j})$$

$\tilde{\Delta}_c = (\omega_c + \sum\limits_i \chi_i \sigma_{zi}) - \omega_d \qquad \tilde{\Delta}_{qi} = (\omega_{qi} + \chi_i) - \omega_d \qquad \chi_i = \frac{g_i^2}{\Delta_i}$

\section{Rotaciones X-Y}

Tomando $\Omega(t) = \Omega^x(t) \cos(\omega_d t) + \Omega^y \sin(\omega_d t)$, donde $\omega_d$ es igual a la frecuencia de resonancia de uno de los qubits logramos rotaciones sobre los ejes X e Y. Las amplitudes de estas rotaciones vienen dadas por $\int_0^{t_0} \Omega^x(t) dt$ y $\int_0^{t_0} \Omega^y(t) dt$, respectivamente, donde $t_0$ es la duración del pulso.

------------------------
$$ \Omega \sigma_+ + \Omega^* \sigma_- $$

$$ e^{i(x+\pi/2)}-e^{-i(x+\pi/2)} = e^{i\pi/2}e^{ix}-e^{-i\pi/2}e^{-ix} =
e^{i\pi/2}e^{ix}+e^{i\pi}e^{-i\pi/2}e^{-ix} = e^{i\pi/2}e^{ix}+e^{i\pi/2}e^{-ix} = e^{i\pi/2}(e^{ix}+e^{-ix})$$
------------------------

$$\hat{H} \approx \tilde{\Delta}_c a^\dagger a + \frac{1}{2} \tilde{\Delta}_q \sigma_z + \frac{1}{2} (\Omega^x(t) \sigma_x + \Omega^y(t) \sigma_y)$$

\section{Compuerta de entrelazamiento}

Ejemplo con sólo dos qubits

$$\hat{H} \approx \frac{1}{2} \tilde{\Delta}_{q_1} \sigma_{z_1} + \frac{1}{2} \tilde{\Delta}_{q_2} \sigma_{z_2} + \frac{g_1 g_2 (\Delta_1 + \Delta_2)}{2 \Delta_1 \Delta_2} (\sigma_{-_1} \sigma_{+_2} + \sigma_{+_1} \sigma_{-_2})$$

Variando la frecuencia de resonacia de los qubit, se puede variar el acoplamiento entre estos. 


