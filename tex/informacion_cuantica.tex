\chapter{Información cuántica}

\section{Kets, bras y operadores}

La notación bra-ket es la notación estándar en la mecánica cuántica para describir estados cuánticos. En el caso de la computación cuántica, se utilizan los kets $\ket{0}$ y $\ket{1}$ para describir los qubits en la base computacional. Este par de estados sería el equivalente a los bits 0 y 1 en la computación clásica. En su representación matricial, los kets $\ket{0}$ y $\ket{1}$ se representan de la siguiente manera:

\[\ket{0} = \begin{pmatrix} 1 \\ 0 \end{pmatrix}\]

\[\ket{1} = \begin{pmatrix} 0 \\ 1 \end{pmatrix}\]

Un bra es el operador adjunto de un ket. Los bras de la base computacional son $\bra{0}$ y $\bra{1}$. En la representación matricial estos son la transpuesta conjugada de los kets y se representan de la siguiente manera:

\[\bra{0} = \begin{pmatrix} 1 & 0 \end{pmatrix}\]

\[\bra{1} = \begin{pmatrix} 0 & 1 \end{pmatrix}\]

\section{Postulados de la mecánica cuántica}
\begin{enumerate}
\item Un estado puro en mecánica cuántica se representa en términos de
  un vector normalizado $\ket{\psi}$ en un espacio de Hilbert
  $\mathcal{H}$.
\item Si $\mathcal{H}_1$ y $\mathcal{H}_2$ son los espacios de Hilbert asociados a dos sistemas físicos, entonces el espacio del sistema
  compuesto $\mathcal{H}$ estará dado por el producto tensoral de los
  dos espacios de Hilbert
  $\mathcal{H} = \mathcal{H}_1 \otimes \mathcal{H}_2$.
\item Para todo observable $a$, existe un operador hermítico
  correspondiente $A$ que actua sobre el espacio de Hilbert
  $\mathcal{H}$, cuyos autovalores son los posibles resultados de una
  medida de este observable.
\item La evolución temporal del sistema sigue la ecuación de
  Schrödinger
  $i \hbar \frac{\partial \ket{\psi}}{\partial t} = H
  \ket{\psi}$. Donde $\hbar$ es la constante de Planck reducida y $H$
  es el Hamiltoniano del sistema, el cuál es el operador hermítico
  correspondiente a la energía.
\item Después de realizar una medida del observable $a$, el estado
  $\ket{\psi}$ del sistema colapsa a al autoestado de $A$
  correspondiente a la medida.
\end{enumerate}

\section{Computación cuántica}
This section's content...

\subsection{Qubits}
Un qubit es un sistema físico de dos niveles, es decir, es un objeto cuyo estado es un elemento del espacio de Hilbert de dimensión $\dim (\mathcal{H})=2$ y puede ser escrito de la siguiente manera: $ \ket{\psi} = \alpha \ket{0} + \beta \ket{1} $, donde $ \{ \ket{0},\ket{1} \} $ forma una base de $\mathcal{H}$ y donde $ \alpha $ y $ \beta $ son números complejos, tales que $ | \alpha |^2 + | \beta |^2 = 1 $, conocidos como amplitudes de probabilidad.
\vspace{0.5cm}

El qubit se puede pensar como el equivalente en IC del bit, el cual, por sus propiedad cuánticas, puede estar no sólo puede estar en el estado $\ket{0}$ y en el estado $\ket{1}$, sino también en superposiciones de estos dos.
\vspace{0.5cm}

El estado de un qubit también se puede escribir de la siguiente manera: $ \ket{\psi} = e^{i \phi_0} \cos ( \theta ) \ket{0} + e^{i \phi_1} \sin ( \theta ) \ket{1}  = e^{i \phi_0} (\cos ( \theta ) \ket{0} + e^{i ( \phi1 - \phi_0 )} \sin ( \theta ) \ket{1}) $, donde $ \theta $, $\phi_0$ y $\phi_1$ son números reales. La fase global $\phi_0$ es ignorable, pues no tiene ningún efecto sobre las probabilidades. Entonces, sin pérdida de generalidad, $ \ket{\psi} = \cos ( \theta ) \ket{0} + \sin ( \theta ) e^{i \phi} \ket{1} $, donde $ \theta \in [0, \pi ] $ y $ \phi \in [0, 2 \pi ] $. De esta manera, podemos representar los qubits en una esfera unitaria, conocida como esfera de Bloch.

\subsection{Compuertas cuánticas}
Las operaciones unitarias con las que se opera sobre los qubits reciben el nombre de compuertas cuánticas.
\vspace{0.5cm}

Las compuertas de un sólo qubit pueden ser vistas como rotaciones en la esfera de Bloch.

\subsubsection{Compuerta identidad}

Esta operación es equivalente a \textit{no-operation} en una computadora clásica.

\begin{minipage}{0.5\textwidth}
\[
\Qcircuit @C=1.4em @R=1.8em {
& \gate{I} & \qw
}
\]
\end{minipage}
\begin{minipage}{0.5\textwidth}
\[
\begin{pmatrix}
1 & 0 \\
0 & 1
\end{pmatrix}
\]
\end{minipage}

\subsubsection{Compuerta X}
Este es el equivalente al NOT clásico, pues tránsforma los $\ket{0}$ en $\ket{1}$ y viceversa, ya que realiza una rotación de $\pi$ sobre el eje X en la esfera de Bloch. Su forma matricial viene dada por la matriz de Pauli $\sigma_x$
\vspace{0.25cm}

\begin{minipage}{0.5\textwidth}
\[
\Qcircuit @C=1.4em @R=1.8em {
& \gate{X} & \qw
}
\]
\end{minipage}
\begin{minipage}{0.5\textwidth}
\[
\begin{pmatrix}
0 & 1 \\
1 & 0
\end{pmatrix}
\]
\end{minipage}

\subsubsection{Compuerta Z}
Esta compuerta no tiene análogo clásico, pues lo que realiza es un cambio de fase. Esto equivale a una rotación de $\pi$ sobre el eje Z en la esfera de Bloch. Su forma matricial viene dada por la matriz de Pauli $\sigma_z$
\vspace{0.25cm}

\begin{minipage}{0.5\textwidth}
\[
\Qcircuit @C=1.4em @R=1.8em {
& \gate{Z} & \qw
}
\]
\end{minipage}
\begin{minipage}{0.5\textwidth}
\[
\begin{pmatrix}
1 & 0 \\
0 & -1
\end{pmatrix}
\]
\end{minipage}

\subsubsection{Compuerta Y}
Esta compuerta realiza una rotación de $\pi$ sobre el eje y de la esfera de Bloch. Su forma matricial viene dada por la matriz de Pauli $\sigma_y$
\vspace{0.25cm}

\begin{minipage}{0.5\textwidth}
\[
\Qcircuit @C=1.4em @R=1.8em {
& \gate{Y} & \qw
}
\]
\end{minipage}
\begin{minipage}{0.5\textwidth}
\[
\begin{pmatrix}
0 & -i \\
i & 0
\end{pmatrix}
\]
\end{minipage}

\subsubsection{Compuerta de Hadamard}
%Esta compuerta realiza una rotación de $\frac{\pi}{2}$ sobre el eje y de la esfera de Bloch. Ella es de especial importancia, pues transforma los estados de la base computacional $\ket{0}$ y $\ket{1}$ en estados de superposiciones uniformes ($\ket{+}$ y $\ket{-}$). También se puede interpretar como el mapa de la base Z a la base X.
Esta compuerta transforma los estados de la base computacional $\ket{0}$ y $\ket{1}$ en estados de superposiciones uniformes ($\ket{+}$ y $\ket{-}$). También se puede interpretar como el mapa de la base Z a la base X.
\vspace{0.25cm}

\begin{minipage}{0.5\textwidth}
\[
\Qcircuit @C=1.4em @R=1.8em {
& \gate{H} & \qw
}
\]
\end{minipage}
\begin{minipage}{0.5\textwidth}
\[
\frac{1}{\sqrt{2}}
\begin{pmatrix}
1 & 1 \\
1 & -1
\end{pmatrix}
\]
\end{minipage}

\subsubsection{Compuerta S}
Esta compuerta es la raiz cuadrada de Z.
\vspace{0.25cm}

\begin{minipage}{0.5\textwidth}
\[
\Qcircuit @C=1.4em @R=1.8em {
& \gate{S} & \qw
}
\]
\end{minipage}
\begin{minipage}{0.5\textwidth}
\[
\begin{pmatrix}
1 & 0 \\
0 & i
\end{pmatrix}
\]
\end{minipage}

\subsubsection{Compuerta T}
Esta compuerta es la raiz cuadrada de S.
\vspace{0.25cm}

\begin{minipage}{0.5\textwidth}
\[
\Qcircuit @C=1.4em @R=1.8em {
& \gate{T} & \qw
}
\]
\end{minipage}
\begin{minipage}{0.5\textwidth}
\[
\begin{pmatrix}
1 & 0 \\
0 & e^{\frac{i \pi}{4}}
\end{pmatrix}
\]
\end{minipage}

\subsubsection{Compuerta de cambio de fase}

\begin{minipage}{0.5\textwidth}
\[
\Qcircuit @C=1.4em @R=1.8em {
& \gate{R_{\phi}} & \qw
}
\]
\end{minipage}
\begin{minipage}{0.5\textwidth}
\[
\begin{pmatrix}
1 & 0 \\
0 & e^{i \phi}
\end{pmatrix}
\]
\end{minipage}

\subsubsection{Compuertas de rotación}

\[
R(\theta,\vec{r}) = e^{i \frac{\theta}{2} \vec{\sigma} \cdot \vec{r}} =
\begin{pmatrix}
\cos(\frac{\theta}{2}) + i z \sin(\frac{\theta}{2}) & \sin(\frac{\theta}{2}) (i x + y) \\
\sin(\frac{\theta}{2}) (i x - y) & \cos(\frac{\theta}{2}) - i z \sin(\frac{\theta}{2})
\end{pmatrix}
\]

\[
R_y(\theta) =
\begin{pmatrix}
\cos(\frac{\theta}{2}) & \sin(\frac{\theta}{2}) \\
-\sin(\frac{\theta}{2}) & \cos(\frac{\theta}{2})
\end{pmatrix}
\]

\[
R_z(\theta) =
\begin{pmatrix}
e^{i \frac{\theta}{2}} & 0 \\
0 & e^{-i \frac{\theta}{2}}
\end{pmatrix}
\]

\[
R_x(\theta) =
\begin{pmatrix}
\cos(\frac{\theta}{2}) & i \sin(\frac{\theta}{2}) \\
i\sin(\frac{\theta}{2}) & \cos(\frac{\theta}{2})
\end{pmatrix}
\]

\[
R_x(\theta) = R_z(\frac{\pi}{2}) R_y(\theta) R_z(\frac{-\pi}{2})
\]

\subsubsection{Compuerta CNOT}

\begin{minipage}{0.5\textwidth}
\[
\Qcircuit @C=1.4em @R=1.8em {
& \ctrl{1} & \qw \\
& \targ & \qw \\
}
\]
\end{minipage}
\begin{minipage}{0.5\textwidth}
\[
\begin{pmatrix}
1 & 0 & 0 & 0 \\
0 & 1 & 0 & 0 \\
0 & 0 & 0 & 1 \\
0 & 0 & 1 & 0
\end{pmatrix}
\]
\end{minipage}

\subsubsection{Compuerta SWAP}

\begin{minipage}{0.5\textwidth}
\[
\Qcircuit @C=1.4em @R=1.8em {
& \qswap & \qw \\
& \qswap \qwx & \qw \\
}
\]
\end{minipage}
\begin{minipage}{0.5\textwidth}
\[
\begin{pmatrix}
1 & 0 & 0 & 0 \\
0 & 0 & 1 & 0 \\
0 & 1 & 0 & 0 \\
0 & 0 & 0 & 1
\end{pmatrix}
\]
\end{minipage}

\subsubsection{Compuerta $\sqrt{\text{SWAP}}$}

\begin{minipage}{0.5\textwidth}
\[
\Qcircuit @C=1.4em @R=1.8em {
& \qswap & \qw \\
& \qswap\qwxo{\scalebox{0.5}{$1\hspace{-1pt}/\hspace{-1pt}2$}} & \qw
}
\]
\end{minipage}
\begin{minipage}{0.5\textwidth}
\[
\begin{pmatrix}
1 & 0 & 0 & 0 \\
0 & \frac{1}{2} (1+i) & \frac{1}{2} (1-i) & 0 \\
0 & \frac{1}{2} (1-i) & \frac{1}{2} (1+i) & 0 \\
0 & 0 & 0 & 1
\end{pmatrix}
\]
\end{minipage}

\subsubsection{Compuerta de Ising}

\begin{minipage}{0.5\textwidth}
\[
\Qcircuit @C=1.4em @R=1.8em {
& \multigate{1}{\mathit{XX}_{\phi}} & \qw \\
& \ghost{\mathit{XX}_{\phi}} & \qw
}
\]
\end{minipage}
\begin{minipage}{0.5\textwidth}
\[
\frac{1}{\sqrt{2}}
\begin{pmatrix}
1 & 0 & 0 & -i e^{i \phi} \\
0 & 1 & -i & 0 \\
0 & -i & 1 & 0 \\
-i e^{-i \phi} & 0 & 0 & 1
\end{pmatrix}
\]
\end{minipage}

\subsubsection{Compuerta de Toffoli}

\begin{minipage}{0.5\textwidth}
\[
\Qcircuit @C=1.4em @R=1.8em {
& \ctrl{1} & \qw \\
& \ctrl{1} & \qw \\
& \targ & \qw \\
}
\]
\end{minipage}
\begin{minipage}{0.5\textwidth}
\[
\begin{pmatrix}
1 & 0 & 0 & 0 & 0 & 0 & 0 & 0 \\
0 & 1 & 0 & 0 & 0 & 0 & 0 & 0 \\
0 & 0 & 1 & 0 & 0 & 0 & 0 & 0 \\
0 & 0 & 0 & 1 & 0 & 0 & 0 & 0 \\
0 & 0 & 0 & 0 & 1 & 0 & 0 & 0 \\
0 & 0 & 0 & 0 & 0 & 1 & 0 & 0 \\
0 & 0 & 0 & 0 & 0 & 0 & 0 & 1 \\
0 & 0 & 0 & 0 & 0 & 0 & 1 & 0
\end{pmatrix}
\]
\end{minipage}

\subsubsection{Compuerta de Fredkin}

\subsubsection{Compuerta de Deutsch}

\begin{minipage}{0.5\textwidth}
\[
\Qcircuit @C=1.4em @R=1.8em {
& \multigate{2}{D(\theta)} & \qw \\
& \ghost{D(\theta)} & \qw \\
& \ghost{D(\theta)} & \qw
}
\]
\end{minipage}
\begin{minipage}{0.5\textwidth}
\[
\begin{pmatrix}
1 & 0 & 0 & 0 & 0 & 0 & 0 & 0 \\
0 & 1 & 0 & 0 & 0 & 0 & 0 & 0 \\
0 & 0 & 1 & 0 & 0 & 0 & 0 & 0 \\
0 & 0 & 0 & 1 & 0 & 0 & 0 & 0 \\
0 & 0 & 0 & 0 & 1 & 0 & 0 & 0 \\
0 & 0 & 0 & 0 & 0 & 1 & 0 & 0 \\
0 & 0 & 0 & 0 & 0 & 0 & i \cos(\theta) & \sin(\theta) \\
0 & 0 & 0 & 0 & 0 & 0 & \sin(\theta) & i \cos(\theta)
\end{pmatrix}
\]
\end{minipage}

\[
\ket{a,b,c} \rightarrow
\begin{cases}
i \cos(\theta) \ket{a,b,c} + \sin(\theta) \ket{a,b,c \oplus 1} & \text{si } a=b=1 \\
\ket{a,b,c} & \text{en otro caso}
\end{cases}
\]

\subsection{Correspondencia entre compuertas clásicas y cuánticas}

\subsection{Conjuntos universales de compuertas cuánticas}
Un conjunto universal de compuertas cuánticas (CUCC) es un conjunto finito de compuertas cuánticas con el cuál se puede aproximar cualquier operación unitaria arbitrariamente bien.
\vspace{0.5cm}

Cualquier operador unitario puede ser escrito en función de compuertas de uno y dos qubits [Barenco et al. 1995].
\vspace{0.5cm}

Un CUCC simple es $\{H,T,\mathit{CNOT}\}$.
\vspace{0.5cm}

Existe un CUCC de una sóla compuerta, la compuerta de Deutsch, $D(\theta)$.
\vspace{0.5cm}

La compuerta de Toffoli es un caso especial de la compuerta de Deutsch, $D(\frac{\pi}{2})$.
\vspace{0.5cm}

Otro CUCC consiste en la compuerta de Ising y la compuerta de cambio de fase, \{$\mathit{XX}_\phi,R_z(\theta)$\}. Este conjunto es nativo en algunas computadoras cuánticas de trampas de iones.

\subsection{Compuertas no cliffordianas}
Las compuertas cliffor


\subsection{Circuitos cuánticos}

\subsection{Paralelismo cuántico}

\subsection{Algoritmos cuánticos}

\subsection{Criterios de DiVincenzo}
Para construir un computador cuántico, se deben cumplir las siguientes condiciones experimentales:

\begin{enumerate}
\item Un sistema físico escalable con qubits bien caracterizados.
\item La habilidad de inicializar el estado de los qubits en un estado fiducial simple.
\item Tiempos de coherencia relevantes largos.
\item Un conjunto universal de compuertas cuánticas.
\item La capacidad de medir qubits en específico.
\end{enumerate}


