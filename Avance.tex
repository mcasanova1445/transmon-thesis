\begin{frame}{Bracket de Poisson y coordenadas canónicas}
\protect\hypertarget{bracket-de-poisson-y-coordenadas-canuxf3nicas}{}

El bracket de Poisson de dos funciones en coordenadas canónicas es

\[\{f,g\} = \sum\limits_{i=1}^N (\frac{\partial f}{\partial q_i}
 \frac{\partial g}{\partial p_i} - \frac{\partial f}{\partial p_i}
 \frac{\partial g}{\partial q_i})\]

Donde p y q cumplen que:

\(\{q_i,q_j\}=0; \qquad \quad \{p_i,p_j\}=0; \qquad \quad  \{q_i,p_j\}=\delta_{i,j}\)

Tradicionalmente, p y q representan posición y momentum,
respectivamente.

\end{frame}

\begin{frame}{Mecánica Hamiltoniana}
\protect\hypertarget{mecuxe1nica-hamiltoniana}{}

En la mecánica Hamiltoniana, la evolución dinámica de un sistema clásico
está descrito por un conjunto de variables canónicas conjugadas
\(r=(q,p)\) de la siguiente manera:

\[\frac{dp}{dt}=-\frac{\partial H}{\partial q} \qquad \quad 
 \frac{dq}{dt}=+\frac{\partial H}{\partial p}\]

Donde el Hamiltoniano \(H\) es la suma de la energía potencial \(V\) y
la energía cinética \(T\), \(H=T+V\).

\end{frame}

\begin{frame}{Ejemplo: Oscilador armónico masa-resorte}
\protect\hypertarget{ejemplo-oscilador-armuxf3nico-masa-resorte}{}

\[H=T+V=\frac{p^2}{2m}+\frac{k q^2}{2}\]

\(p=mv=m \frac{dq}{dt}; \qquad \quad \omega=\sqrt{\frac{k}{m}}; \qquad \quad  m \frac{d^2q}{dt^2} + k q=0\)

\(-\frac{\partial H}{\partial q} = -k q = m \frac{d^2q}{dt^2} = \frac{dp}{dt};  \qquad \quad \frac{\partial H}{\partial p} = \frac{p}{m} = \frac{dq}{dt}\)

\(\{q,q\} = \frac{\partial q}{\partial q_i}  \frac{\partial q}{\partial p_i} - \frac{\partial q}{\partial p_i}  \frac{\partial q}{\partial q_i} = 0\)

\(\{p,p\} = \frac{\partial p}{\partial q_i}  \frac{\partial p}{\partial p_i} - \frac{\partial p}{\partial p_i}  \frac{\partial p}{\partial q_i} = 0\)

\(\{q,p\} = \frac{\partial q}{\partial q_i}  \frac{\partial p}{\partial p_i} - \frac{\partial q}{\partial p_i}  \frac{\partial p}{\partial q_i} = 1\)

\end{frame}

\begin{frame}{Cuantización canónica}
\protect\hypertarget{cuantizaciuxf3n-canuxf3nica}{}

Paul Dirac introdujo en su tesis doctoral en 1926 el “método de analogía
clásica” para la cuantización de sistemas físicos.

Este método consiste en cuantizar las teorías clásicas intentando
mantener las estructuras formales de la teoría clásica en la mayor
medida posible.

Dirac propuso la siguiente relación:
\(\{q,p\} \rightarrow \frac{1}{i\hbar} [\hat{q},\hat{p}]\)

Luego se demostró que esta relación no se para todo sistema físico, sino
que es un caso especial de una relación más general. Sin embargo, para
los casos a estudiar en esta presentación, sí se cumple.

\end{frame}

\begin{frame}{Cuantización canónica: La receta}
\protect\hypertarget{cuantizaciuxf3n-canuxf3nica-la-receta}{}

El procedimientro a seguir será:

\begin{itemize}
\tightlist
\item
  Construir el Hamiltoniano clásico del sistema en términos de
  coordenadas canónicas
\item
  Sustituir los observables por operadores cuánticos
\item
  Forzar la relación de conmutación canónica
\end{itemize}

\end{frame}

\begin{frame}{Oscilador armónico LC: Hamiltoniano clásico}
\protect\hypertarget{oscilador-armuxf3nico-lc-hamiltoniano-cluxe1sico}{}

Un ejemplo de variables canónicamente conjugades son la carga y el flujo
magnético. Por esto, podemos cuántizar un oscilador armónico LC con la
cuantización canónica.

La energía en un capacitor es \(\frac{Q^2}{2C}\)

La energía en un inductor es \(\frac{\phi^2}{2L}\)

Por lo tanto, el Hamiltoniano clásico del sistema es:

\[H = \frac{Q^2}{2C} + \frac{\phi^2}{2L}\]

\end{frame}

\begin{frame}{Oscilador armónico LC: Cuantización canónica}
\protect\hypertarget{oscilador-armuxf3nico-lc-cuantizaciuxf3n-canuxf3nica}{}

\(\frac{\partial H}{\partial Q} = \frac{Q}{C} = -L \frac{dI}{dt} =  -\frac{d\phi}{dt}; \qquad \quad  \frac{\partial H}{\partial \phi} = \frac{\phi}{L} = I = \frac{dQ}{dt}\)

Esto demuestra que \(\phi\) y \(Q\) son coordenadas canónicas. Donde
\(\phi\) y \(Q\) son posición y momentum generalizados, respectivamente.
Así que procedemos a sustituirlas por los operadores \(\hat{\phi}\) y
\(\hat{q}\) y escribimos el Hamiltoniano cuántico

\[\hat{H} = \frac{\hat{q}^2}{2C} + \frac{\hat{\phi}^2}{2L}\]

Donde \([\hat{\phi},\hat{q}]= i \hbar\)

Esto, además, significa que la carga en el capacitor y el flujo en el
inductor no pueden ser medidos simultaneamente.

Finalmente, la frecuencia de resonancia de este oscilador es
\(\omega = \frac{1}{\sqrt{L C}}\) y los niveles de energía se encuentran
separados por \(\hbar \omega\)

\end{frame}

\begin{frame}{Efecto Josephson}
\protect\hypertarget{efecto-josephson}{}

Se tienen dos placas superconductoras A y B, separadas por un aislante.
Las funciones de onda de las placas superconductoras son:
\(\psi_A = \sqrt{\rho_1} e^{i \phi_1}, \psi_B = \sqrt{\rho_2} e^{i \phi_2}\)

En principio, no debería haber una supercorriente entre ambas placas.
Sin embargo, ese no es el caso. Por el efecto tunel, pares de electrones
(pares de Cooper) pueden pasar de una placa a la otra sin disipación.

\[V_J = \frac{\hbar}{2e} \frac{d\delta}{dt}\]

\[I_J = I_0 \sin(\delta)\]

Donde \(\delta=\phi_2-\phi_1\) es la diferencia de fase entre las dos
placas superconductoras.

\end{frame}

\begin{frame}{Efecto Josephson DC y AC}
\protect\hypertarget{efecto-josephson-dc-y-ac}{}

\begin{itemize}
\tightlist
\item
  Efecto Josephson DC
\end{itemize}

Si las placas se encuentran sin alimentación, entonces correrá una
supercorriente constante a través de ellas.

\begin{itemize}
\tightlist
\item
  Efecto Josephson AC
\end{itemize}

Si las placas se alimentan con un voltaje DC externo, entonces la
diferencia de fase entre ellas variará linealmente con el tiempo y habrá
una corriente AC a través de ellas.

\end{frame}

\begin{frame}{Energía e inductancia de Josephson}
\protect\hypertarget{energuxeda-e-inductancia-de-josephson}{}

\[E_J = \int\limits_0^t I_0 \sin(\delta) \frac{\hbar}{2e} \frac{d\delta}{dt} dt
 = \frac{\hbar I_0}{2e} \int\limits_0^\delta \sin(\delta) dt
 = \frac{\hbar I_0}{2e} (1 - \cos(\delta))\]

\(\frac{dI_J}{dt} = I_0 \cos(\delta) \frac{d\delta){dt} = I_0 \cos(\delta)  \frac{2e}{\hbar} V_J\)

\[L_J = \frac{\hbar}{2e I_0 \cos(\delta)}\]

\(E_C = \frac{(2e)^2}{2C}\)

\(E_L = \frac{\hbar^2}{4e^2L}\)

\end{frame}

\begin{frame}{The end}
\protect\hypertarget{the-end}{}

Has ended sooner

\end{frame}
